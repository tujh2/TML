\documentclass[11pt]{report}
\usepackage[T2A,T1]{fontenc}
\usepackage[utf8]{inputenc}
\usepackage{blindtext}
\usepackage[russian]{babel}
\usepackage{mwe}
\usepackage{graphbox}
\usepackage[document]{ragged2e}
\usepackage[margin=50pt]{geometry}
\usepackage{longtable}
\usepackage{fontspec}
\usepackage{float}
\usepackage{titlesec}
\usepackage{setspace}

\usepackage{hyperref}
\hypersetup{
    colorlinks,
    citecolor=black,
    filecolor=black,
    linkcolor=black,
    urlcolor=black
}

\setstretch{1}
\graphicspath{{./images/}}
\setmainfont{LiberationSerif}
\setmonofont{Hack}
\titleformat{\chapter}{\normalfont\LARGE\bfseries}{\thechapter}{1em}{}
\titleclass{\chapter}{straight}
\titlespacing{\chapter}{0pt}{0pt}{5pt}[25pt]


    \usepackage[breakable]{tcolorbox}
    \usepackage{parskip} % Stop auto-indenting (to mimic markdown behaviour)
    
    \usepackage{iftex}
    \ifPDFTeX
    	\usepackage[T1]{fontenc}
    	\usepackage{mathpazo}
    \else
    	\usepackage{fontspec}
    \fi

    % Basic figure setup, for now with no caption control since it's done
    % automatically by Pandoc (which extracts ![](path) syntax from Markdown).
    \usepackage{graphicx}
    % Maintain compatibility with old templates. Remove in nbconvert 6.0
    \let\Oldincludegraphics\includegraphics
    % Ensure that by default, figures have no caption (until we provide a
    % proper Figure object with a Caption API and a way to capture that
    % in the conversion process - todo).
    \usepackage{caption}
    \DeclareCaptionFormat{nocaption}{}
    \captionsetup{format=nocaption,aboveskip=0pt,belowskip=0pt}

    \usepackage{float}
    \floatplacement{figure}{H} % forces figures to be placed at the correct location
    \usepackage{xcolor} % Allow colors to be defined
    \usepackage{enumerate} % Needed for markdown enumerations to work
    \usepackage{geometry} % Used to adjust the document margins
    \usepackage{amsmath} % Equations
    \usepackage{amssymb} % Equations
    \usepackage{textcomp} % defines textquotesingle
    % Hack from http://tex.stackexchange.com/a/47451/13684:
    \AtBeginDocument{%
        \def\PYZsq{\textquotesingle}% Upright quotes in Pygmentized code
    }
    \usepackage{upquote} % Upright quotes for verbatim code
    \usepackage{eurosym} % defines \euro
    \usepackage[mathletters]{ucs} % Extended unicode (utf-8) support
    \usepackage{fancyvrb} % verbatim replacement that allows latex
    \usepackage{grffile} % extends the file name processing of package graphics 
                         % to support a larger range
    \makeatletter % fix for old versions of grffile with XeLaTeX
    \@ifpackagelater{grffile}{2019/11/01}
    {
      % Do nothing on new versions
    }
    {
      \def\Gread@@xetex#1{%
        \IfFileExists{"\Gin@base".bb}%
        {\Gread@eps{\Gin@base.bb}}%
        {\Gread@@xetex@aux#1}%
      }
    }
    \makeatother
    \usepackage[Export]{adjustbox} % Used to constrain images to a maximum size
    \adjustboxset{max size={0.9\linewidth}{0.9\paperheight}}

    % The hyperref package gives us a pdf with properly built
    % internal navigation ('pdf bookmarks' for the table of contents,
    % internal cross-reference links, web links for URLs, etc.)
    \usepackage{hyperref}
    % The default LaTeX title has an obnoxious amount of whitespace. By default,
    % titling removes some of it. It also provides customization options.
    \usepackage{titling}
    \usepackage{longtable} % longtable support required by pandoc >1.10
    \usepackage{booktabs}  % table support for pandoc > 1.12.2
    \usepackage[inline]{enumitem} % IRkernel/repr support (it uses the enumerate* environment)
    \usepackage[normalem]{ulem} % ulem is needed to support strikethroughs (\sout)
                                % normalem makes italics be italics, not underlines
    \usepackage{mathrsfs}
    

    
    % Colors for the hyperref package
    \definecolor{urlcolor}{rgb}{0,.145,.698}
    \definecolor{linkcolor}{rgb}{.71,0.21,0.01}
    \definecolor{citecolor}{rgb}{.12,.54,.11}

    % ANSI colors
    \definecolor{ansi-black}{HTML}{3E424D}
    \definecolor{ansi-black-intense}{HTML}{282C36}
    \definecolor{ansi-red}{HTML}{E75C58}
    \definecolor{ansi-red-intense}{HTML}{B22B31}
    \definecolor{ansi-green}{HTML}{00A250}
    \definecolor{ansi-green-intense}{HTML}{007427}
    \definecolor{ansi-yellow}{HTML}{DDB62B}
    \definecolor{ansi-yellow-intense}{HTML}{B27D12}
    \definecolor{ansi-blue}{HTML}{208FFB}
    \definecolor{ansi-blue-intense}{HTML}{0065CA}
    \definecolor{ansi-magenta}{HTML}{D160C4}
    \definecolor{ansi-magenta-intense}{HTML}{A03196}
    \definecolor{ansi-cyan}{HTML}{60C6C8}
    \definecolor{ansi-cyan-intense}{HTML}{258F8F}
    \definecolor{ansi-white}{HTML}{C5C1B4}
    \definecolor{ansi-white-intense}{HTML}{A1A6B2}
    \definecolor{ansi-default-inverse-fg}{HTML}{FFFFFF}
    \definecolor{ansi-default-inverse-bg}{HTML}{000000}

    % common color for the border for error outputs.
    \definecolor{outerrorbackground}{HTML}{FFDFDF}

    % commands and environments needed by pandoc snippets
    % extracted from the output of `pandoc -s`
    \providecommand{\tightlist}{%
      \setlength{\itemsep}{0pt}\setlength{\parskip}{0pt}}
    \DefineVerbatimEnvironment{Highlighting}{Verbatim}{commandchars=\\\{\}}
    % Add ',fontsize=\small' for more characters per line
    \newenvironment{Shaded}{}{}
    \newcommand{\KeywordTok}[1]{\textcolor[rgb]{0.00,0.44,0.13}{\textbf{{#1}}}}
    \newcommand{\DataTypeTok}[1]{\textcolor[rgb]{0.56,0.13,0.00}{{#1}}}
    \newcommand{\DecValTok}[1]{\textcolor[rgb]{0.25,0.63,0.44}{{#1}}}
    \newcommand{\BaseNTok}[1]{\textcolor[rgb]{0.25,0.63,0.44}{{#1}}}
    \newcommand{\FloatTok}[1]{\textcolor[rgb]{0.25,0.63,0.44}{{#1}}}
    \newcommand{\CharTok}[1]{\textcolor[rgb]{0.25,0.44,0.63}{{#1}}}
    \newcommand{\StringTok}[1]{\textcolor[rgb]{0.25,0.44,0.63}{{#1}}}
    \newcommand{\CommentTok}[1]{\textcolor[rgb]{0.38,0.63,0.69}{\textit{{#1}}}}
    \newcommand{\OtherTok}[1]{\textcolor[rgb]{0.00,0.44,0.13}{{#1}}}
    \newcommand{\AlertTok}[1]{\textcolor[rgb]{1.00,0.00,0.00}{\textbf{{#1}}}}
    \newcommand{\FunctionTok}[1]{\textcolor[rgb]{0.02,0.16,0.49}{{#1}}}
    \newcommand{\RegionMarkerTok}[1]{{#1}}
    \newcommand{\ErrorTok}[1]{\textcolor[rgb]{1.00,0.00,0.00}{\textbf{{#1}}}}
    \newcommand{\NormalTok}[1]{{#1}}
    
    % Additional commands for more recent versions of Pandoc
    \newcommand{\ConstantTok}[1]{\textcolor[rgb]{0.53,0.00,0.00}{{#1}}}
    \newcommand{\SpecialCharTok}[1]{\textcolor[rgb]{0.25,0.44,0.63}{{#1}}}
    \newcommand{\VerbatimStringTok}[1]{\textcolor[rgb]{0.25,0.44,0.63}{{#1}}}
    \newcommand{\SpecialStringTok}[1]{\textcolor[rgb]{0.73,0.40,0.53}{{#1}}}
    \newcommand{\ImportTok}[1]{{#1}}
    \newcommand{\DocumentationTok}[1]{\textcolor[rgb]{0.73,0.13,0.13}{\textit{{#1}}}}
    \newcommand{\AnnotationTok}[1]{\textcolor[rgb]{0.38,0.63,0.69}{\textbf{\textit{{#1}}}}}
    \newcommand{\CommentVarTok}[1]{\textcolor[rgb]{0.38,0.63,0.69}{\textbf{\textit{{#1}}}}}
    \newcommand{\VariableTok}[1]{\textcolor[rgb]{0.10,0.09,0.49}{{#1}}}
    \newcommand{\ControlFlowTok}[1]{\textcolor[rgb]{0.00,0.44,0.13}{\textbf{{#1}}}}
    \newcommand{\OperatorTok}[1]{\textcolor[rgb]{0.40,0.40,0.40}{{#1}}}
    \newcommand{\BuiltInTok}[1]{{#1}}
    \newcommand{\ExtensionTok}[1]{{#1}}
    \newcommand{\PreprocessorTok}[1]{\textcolor[rgb]{0.74,0.48,0.00}{{#1}}}
    \newcommand{\AttributeTok}[1]{\textcolor[rgb]{0.49,0.56,0.16}{{#1}}}
    \newcommand{\InformationTok}[1]{\textcolor[rgb]{0.38,0.63,0.69}{\textbf{\textit{{#1}}}}}
    \newcommand{\WarningTok}[1]{\textcolor[rgb]{0.38,0.63,0.69}{\textbf{\textit{{#1}}}}}
    
    
    % Define a nice break command that doesn't care if a line doesn't already
    % exist.
    \def\br{\hspace*{\fill} \\* }
    % Math Jax compatibility definitions
    \def\gt{>}
    \def\lt{<}
    \let\Oldtex\TeX
    \let\Oldlatex\LaTeX
    \renewcommand{\TeX}{\textrm{\Oldtex}}
    \renewcommand{\LaTeX}{\textrm{\Oldlatex}}
    % Document parameters
    % Document title
    \title{rk2}
    
    
    
    
    
% Pygments definitions
\makeatletter
\def\PY@reset{\let\PY@it=\relax \let\PY@bf=\relax%
    \let\PY@ul=\relax \let\PY@tc=\relax%
    \let\PY@bc=\relax \let\PY@ff=\relax}
\def\PY@tok#1{\csname PY@tok@#1\endcsname}
\def\PY@toks#1+{\ifx\relax#1\empty\else%
    \PY@tok{#1}\expandafter\PY@toks\fi}
\def\PY@do#1{\PY@bc{\PY@tc{\PY@ul{%
    \PY@it{\PY@bf{\PY@ff{#1}}}}}}}
\def\PY#1#2{\PY@reset\PY@toks#1+\relax+\PY@do{#2}}

\@namedef{PY@tok@w}{\def\PY@tc##1{\textcolor[rgb]{0.73,0.73,0.73}{##1}}}
\@namedef{PY@tok@c}{\let\PY@it=\textit\def\PY@tc##1{\textcolor[rgb]{0.25,0.50,0.50}{##1}}}
\@namedef{PY@tok@cp}{\def\PY@tc##1{\textcolor[rgb]{0.74,0.48,0.00}{##1}}}
\@namedef{PY@tok@k}{\let\PY@bf=\textbf\def\PY@tc##1{\textcolor[rgb]{0.00,0.50,0.00}{##1}}}
\@namedef{PY@tok@kp}{\def\PY@tc##1{\textcolor[rgb]{0.00,0.50,0.00}{##1}}}
\@namedef{PY@tok@kt}{\def\PY@tc##1{\textcolor[rgb]{0.69,0.00,0.25}{##1}}}
\@namedef{PY@tok@o}{\def\PY@tc##1{\textcolor[rgb]{0.40,0.40,0.40}{##1}}}
\@namedef{PY@tok@ow}{\let\PY@bf=\textbf\def\PY@tc##1{\textcolor[rgb]{0.67,0.13,1.00}{##1}}}
\@namedef{PY@tok@nb}{\def\PY@tc##1{\textcolor[rgb]{0.00,0.50,0.00}{##1}}}
\@namedef{PY@tok@nf}{\def\PY@tc##1{\textcolor[rgb]{0.00,0.00,1.00}{##1}}}
\@namedef{PY@tok@nc}{\let\PY@bf=\textbf\def\PY@tc##1{\textcolor[rgb]{0.00,0.00,1.00}{##1}}}
\@namedef{PY@tok@nn}{\let\PY@bf=\textbf\def\PY@tc##1{\textcolor[rgb]{0.00,0.00,1.00}{##1}}}
\@namedef{PY@tok@ne}{\let\PY@bf=\textbf\def\PY@tc##1{\textcolor[rgb]{0.82,0.25,0.23}{##1}}}
\@namedef{PY@tok@nv}{\def\PY@tc##1{\textcolor[rgb]{0.10,0.09,0.49}{##1}}}
\@namedef{PY@tok@no}{\def\PY@tc##1{\textcolor[rgb]{0.53,0.00,0.00}{##1}}}
\@namedef{PY@tok@nl}{\def\PY@tc##1{\textcolor[rgb]{0.63,0.63,0.00}{##1}}}
\@namedef{PY@tok@ni}{\let\PY@bf=\textbf\def\PY@tc##1{\textcolor[rgb]{0.60,0.60,0.60}{##1}}}
\@namedef{PY@tok@na}{\def\PY@tc##1{\textcolor[rgb]{0.49,0.56,0.16}{##1}}}
\@namedef{PY@tok@nt}{\let\PY@bf=\textbf\def\PY@tc##1{\textcolor[rgb]{0.00,0.50,0.00}{##1}}}
\@namedef{PY@tok@nd}{\def\PY@tc##1{\textcolor[rgb]{0.67,0.13,1.00}{##1}}}
\@namedef{PY@tok@s}{\def\PY@tc##1{\textcolor[rgb]{0.73,0.13,0.13}{##1}}}
\@namedef{PY@tok@sd}{\let\PY@it=\textit\def\PY@tc##1{\textcolor[rgb]{0.73,0.13,0.13}{##1}}}
\@namedef{PY@tok@si}{\let\PY@bf=\textbf\def\PY@tc##1{\textcolor[rgb]{0.73,0.40,0.53}{##1}}}
\@namedef{PY@tok@se}{\let\PY@bf=\textbf\def\PY@tc##1{\textcolor[rgb]{0.73,0.40,0.13}{##1}}}
\@namedef{PY@tok@sr}{\def\PY@tc##1{\textcolor[rgb]{0.73,0.40,0.53}{##1}}}
\@namedef{PY@tok@ss}{\def\PY@tc##1{\textcolor[rgb]{0.10,0.09,0.49}{##1}}}
\@namedef{PY@tok@sx}{\def\PY@tc##1{\textcolor[rgb]{0.00,0.50,0.00}{##1}}}
\@namedef{PY@tok@m}{\def\PY@tc##1{\textcolor[rgb]{0.40,0.40,0.40}{##1}}}
\@namedef{PY@tok@gh}{\let\PY@bf=\textbf\def\PY@tc##1{\textcolor[rgb]{0.00,0.00,0.50}{##1}}}
\@namedef{PY@tok@gu}{\let\PY@bf=\textbf\def\PY@tc##1{\textcolor[rgb]{0.50,0.00,0.50}{##1}}}
\@namedef{PY@tok@gd}{\def\PY@tc##1{\textcolor[rgb]{0.63,0.00,0.00}{##1}}}
\@namedef{PY@tok@gi}{\def\PY@tc##1{\textcolor[rgb]{0.00,0.63,0.00}{##1}}}
\@namedef{PY@tok@gr}{\def\PY@tc##1{\textcolor[rgb]{1.00,0.00,0.00}{##1}}}
\@namedef{PY@tok@ge}{\let\PY@it=\textit}
\@namedef{PY@tok@gs}{\let\PY@bf=\textbf}
\@namedef{PY@tok@gp}{\let\PY@bf=\textbf\def\PY@tc##1{\textcolor[rgb]{0.00,0.00,0.50}{##1}}}
\@namedef{PY@tok@go}{\def\PY@tc##1{\textcolor[rgb]{0.53,0.53,0.53}{##1}}}
\@namedef{PY@tok@gt}{\def\PY@tc##1{\textcolor[rgb]{0.00,0.27,0.87}{##1}}}
\@namedef{PY@tok@err}{\def\PY@bc##1{{\setlength{\fboxsep}{\string -\fboxrule}\fcolorbox[rgb]{1.00,0.00,0.00}{1,1,1}{\strut ##1}}}}
\@namedef{PY@tok@kc}{\let\PY@bf=\textbf\def\PY@tc##1{\textcolor[rgb]{0.00,0.50,0.00}{##1}}}
\@namedef{PY@tok@kd}{\let\PY@bf=\textbf\def\PY@tc##1{\textcolor[rgb]{0.00,0.50,0.00}{##1}}}
\@namedef{PY@tok@kn}{\let\PY@bf=\textbf\def\PY@tc##1{\textcolor[rgb]{0.00,0.50,0.00}{##1}}}
\@namedef{PY@tok@kr}{\let\PY@bf=\textbf\def\PY@tc##1{\textcolor[rgb]{0.00,0.50,0.00}{##1}}}
\@namedef{PY@tok@bp}{\def\PY@tc##1{\textcolor[rgb]{0.00,0.50,0.00}{##1}}}
\@namedef{PY@tok@fm}{\def\PY@tc##1{\textcolor[rgb]{0.00,0.00,1.00}{##1}}}
\@namedef{PY@tok@vc}{\def\PY@tc##1{\textcolor[rgb]{0.10,0.09,0.49}{##1}}}
\@namedef{PY@tok@vg}{\def\PY@tc##1{\textcolor[rgb]{0.10,0.09,0.49}{##1}}}
\@namedef{PY@tok@vi}{\def\PY@tc##1{\textcolor[rgb]{0.10,0.09,0.49}{##1}}}
\@namedef{PY@tok@vm}{\def\PY@tc##1{\textcolor[rgb]{0.10,0.09,0.49}{##1}}}
\@namedef{PY@tok@sa}{\def\PY@tc##1{\textcolor[rgb]{0.73,0.13,0.13}{##1}}}
\@namedef{PY@tok@sb}{\def\PY@tc##1{\textcolor[rgb]{0.73,0.13,0.13}{##1}}}
\@namedef{PY@tok@sc}{\def\PY@tc##1{\textcolor[rgb]{0.73,0.13,0.13}{##1}}}
\@namedef{PY@tok@dl}{\def\PY@tc##1{\textcolor[rgb]{0.73,0.13,0.13}{##1}}}
\@namedef{PY@tok@s2}{\def\PY@tc##1{\textcolor[rgb]{0.73,0.13,0.13}{##1}}}
\@namedef{PY@tok@sh}{\def\PY@tc##1{\textcolor[rgb]{0.73,0.13,0.13}{##1}}}
\@namedef{PY@tok@s1}{\def\PY@tc##1{\textcolor[rgb]{0.73,0.13,0.13}{##1}}}
\@namedef{PY@tok@mb}{\def\PY@tc##1{\textcolor[rgb]{0.40,0.40,0.40}{##1}}}
\@namedef{PY@tok@mf}{\def\PY@tc##1{\textcolor[rgb]{0.40,0.40,0.40}{##1}}}
\@namedef{PY@tok@mh}{\def\PY@tc##1{\textcolor[rgb]{0.40,0.40,0.40}{##1}}}
\@namedef{PY@tok@mi}{\def\PY@tc##1{\textcolor[rgb]{0.40,0.40,0.40}{##1}}}
\@namedef{PY@tok@il}{\def\PY@tc##1{\textcolor[rgb]{0.40,0.40,0.40}{##1}}}
\@namedef{PY@tok@mo}{\def\PY@tc##1{\textcolor[rgb]{0.40,0.40,0.40}{##1}}}
\@namedef{PY@tok@ch}{\let\PY@it=\textit\def\PY@tc##1{\textcolor[rgb]{0.25,0.50,0.50}{##1}}}
\@namedef{PY@tok@cm}{\let\PY@it=\textit\def\PY@tc##1{\textcolor[rgb]{0.25,0.50,0.50}{##1}}}
\@namedef{PY@tok@cpf}{\let\PY@it=\textit\def\PY@tc##1{\textcolor[rgb]{0.25,0.50,0.50}{##1}}}
\@namedef{PY@tok@c1}{\let\PY@it=\textit\def\PY@tc##1{\textcolor[rgb]{0.25,0.50,0.50}{##1}}}
\@namedef{PY@tok@cs}{\let\PY@it=\textit\def\PY@tc##1{\textcolor[rgb]{0.25,0.50,0.50}{##1}}}

\def\PYZbs{\char`\\}
\def\PYZus{\char`\_}
\def\PYZob{\char`\{}
\def\PYZcb{\char`\}}
\def\PYZca{\char`\^}
\def\PYZam{\char`\&}
\def\PYZlt{\char`\<}
\def\PYZgt{\char`\>}
\def\PYZsh{\char`\#}
\def\PYZpc{\char`\%}
\def\PYZdl{\char`\$}
\def\PYZhy{\char`\-}
\def\PYZsq{\char`\'}
\def\PYZdq{\char`\"}
\def\PYZti{\char`\~}
% for compatibility with earlier versions
\def\PYZat{@}
\def\PYZlb{[}
\def\PYZrb{]}
\makeatother


    % For linebreaks inside Verbatim environment from package fancyvrb. 
    \makeatletter
        \newbox\Wrappedcontinuationbox 
        \newbox\Wrappedvisiblespacebox 
        \newcommand*\Wrappedvisiblespace {\textcolor{red}{\textvisiblespace}} 
        \newcommand*\Wrappedcontinuationsymbol {\textcolor{red}{\llap{\tiny$\m@th\hookrightarrow$}}} 
        \newcommand*\Wrappedcontinuationindent {3ex } 
        \newcommand*\Wrappedafterbreak {\kern\Wrappedcontinuationindent\copy\Wrappedcontinuationbox} 
        % Take advantage of the already applied Pygments mark-up to insert 
        % potential linebreaks for TeX processing. 
        %        {, <, #, %, $, ' and ": go to next line. 
        %        _, }, ^, &, >, - and ~: stay at end of broken line. 
        % Use of \textquotesingle for straight quote. 
        \newcommand*\Wrappedbreaksatspecials {% 
            \def\PYGZus{\discretionary{\char`\_}{\Wrappedafterbreak}{\char`\_}}% 
            \def\PYGZob{\discretionary{}{\Wrappedafterbreak\char`\{}{\char`\{}}% 
            \def\PYGZcb{\discretionary{\char`\}}{\Wrappedafterbreak}{\char`\}}}% 
            \def\PYGZca{\discretionary{\char`\^}{\Wrappedafterbreak}{\char`\^}}% 
            \def\PYGZam{\discretionary{\char`\&}{\Wrappedafterbreak}{\char`\&}}% 
            \def\PYGZlt{\discretionary{}{\Wrappedafterbreak\char`\<}{\char`\<}}% 
            \def\PYGZgt{\discretionary{\char`\>}{\Wrappedafterbreak}{\char`\>}}% 
            \def\PYGZsh{\discretionary{}{\Wrappedafterbreak\char`\#}{\char`\#}}% 
            \def\PYGZpc{\discretionary{}{\Wrappedafterbreak\char`\%}{\char`\%}}% 
            \def\PYGZdl{\discretionary{}{\Wrappedafterbreak\char`\$}{\char`\$}}% 
            \def\PYGZhy{\discretionary{\char`\-}{\Wrappedafterbreak}{\char`\-}}% 
            \def\PYGZsq{\discretionary{}{\Wrappedafterbreak\textquotesingle}{\textquotesingle}}% 
            \def\PYGZdq{\discretionary{}{\Wrappedafterbreak\char`\"}{\char`\"}}% 
            \def\PYGZti{\discretionary{\char`\~}{\Wrappedafterbreak}{\char`\~}}% 
        } 
        % Some characters . , ; ? ! / are not pygmentized. 
        % This macro makes them "active" and they will insert potential linebreaks 
        \newcommand*\Wrappedbreaksatpunct {% 
            \lccode`\~`\.\lowercase{\def~}{\discretionary{\hbox{\char`\.}}{\Wrappedafterbreak}{\hbox{\char`\.}}}% 
            \lccode`\~`\,\lowercase{\def~}{\discretionary{\hbox{\char`\,}}{\Wrappedafterbreak}{\hbox{\char`\,}}}% 
            \lccode`\~`\;\lowercase{\def~}{\discretionary{\hbox{\char`\;}}{\Wrappedafterbreak}{\hbox{\char`\;}}}% 
            \lccode`\~`\:\lowercase{\def~}{\discretionary{\hbox{\char`\:}}{\Wrappedafterbreak}{\hbox{\char`\:}}}% 
            \lccode`\~`\?\lowercase{\def~}{\discretionary{\hbox{\char`\?}}{\Wrappedafterbreak}{\hbox{\char`\?}}}% 
            \lccode`\~`\!\lowercase{\def~}{\discretionary{\hbox{\char`\!}}{\Wrappedafterbreak}{\hbox{\char`\!}}}% 
            \lccode`\~`\/\lowercase{\def~}{\discretionary{\hbox{\char`\/}}{\Wrappedafterbreak}{\hbox{\char`\/}}}% 
            \catcode`\.\active
            \catcode`\,\active 
            \catcode`\;\active
            \catcode`\:\active
            \catcode`\?\active
            \catcode`\!\active
            \catcode`\/\active 
            \lccode`\~`\~ 	
        }
    \makeatother

    \let\OriginalVerbatim=\Verbatim
    \makeatletter
    \renewcommand{\Verbatim}[1][1]{%
        %\parskip\z@skip
        \sbox\Wrappedcontinuationbox {\Wrappedcontinuationsymbol}%
        \sbox\Wrappedvisiblespacebox {\FV@SetupFont\Wrappedvisiblespace}%
        \def\FancyVerbFormatLine ##1{\hsize\linewidth
            \vtop{\raggedright\hyphenpenalty\z@\exhyphenpenalty\z@
                \doublehyphendemerits\z@\finalhyphendemerits\z@
                \strut ##1\strut}%
        }%
        % If the linebreak is at a space, the latter will be displayed as visible
        % space at end of first line, and a continuation symbol starts next line.
        % Stretch/shrink are however usually zero for typewriter font.
        \def\FV@Space {%
            \nobreak\hskip\z@ plus\fontdimen3\font minus\fontdimen4\font
            \discretionary{\copy\Wrappedvisiblespacebox}{\Wrappedafterbreak}
            {\kern\fontdimen2\font}%
        }%
        
        % Allow breaks at special characters using \PYG... macros.
        \Wrappedbreaksatspecials
        % Breaks at punctuation characters . , ; ? ! and / need catcode=\active 	
        \OriginalVerbatim[#1,codes*=\Wrappedbreaksatpunct]%
    }
    \makeatother

    % Exact colors from NB
    \definecolor{incolor}{HTML}{303F9F}
    \definecolor{outcolor}{HTML}{D84315}
    \definecolor{cellborder}{HTML}{CFCFCF}
    \definecolor{cellbackground}{HTML}{F7F7F7}
    
    % prompt
    \makeatletter
    \newcommand{\boxspacing}{\kern\kvtcb@left@rule\kern\kvtcb@boxsep}
    \makeatother
    \newcommand{\prompt}[4]{
        {\ttfamily\llap{{\color{#2}[#3]:\hspace{3pt}#4}}\vspace{-\baselineskip}}
    }
    

    
    % Prevent overflowing lines due to hard-to-break entities
    \sloppy 
    % Setup hyperref package
    \hypersetup{
      breaklinks=true,  % so long urls are correctly broken across lines
      colorlinks=true,
      urlcolor=urlcolor,
      linkcolor=linkcolor,
      citecolor=citecolor,
      }
    % Slightly bigger margins than the latex defaults
    
    \geometry{verbose,tmargin=1in,bmargin=1in,lmargin=1in,rmargin=1in}
    
    

\begin{document}
    
	 \begin{titlepage}
		\begin{minipage}{0.3\textwidth}
		\includegraphics{logo.png}
		\end{minipage}
		\begin{minipage}{0.6\textwidth}\centering
			\textbf{
				Министерство науки и высшего образования Российской Федерации
				Федеральное государственное бюджетное образовательное
				учреждение высшего образования
				«Московский государственный технический университет
				имени Н.Э. Баумана (национальный исследовательский университет)»
				(МГТУ им. Н.Э. Баумана)
			}
		\end{minipage}

		\vspace{4cm}
		\centering
		\Large
		\textbf{
			Рубежный контроль №2 \\
			по курсу «Теория машинного обучения» \\
			Вариант 16
		}

		\vspace{4cm}
		\begin{flushright}
			Выполнил \\
			студент группы ИУ5-64Б \\
			Сысойкин Е.М.
		\end{flushright}
		\vspace{4cm}
		Москва, 2020
	\end{titlepage}
    
    

    
    \hypertarget{ux440ux43a1-ux442ux43cux43e-ux441ux44bux441ux43eux439ux43aux438ux43d-ux435ux433ux43eux440-ux432ux430ux440ux438ux430ux43dux442-16-ux438ux4435-64ux431}{%
\section{РК1; ТМО; Сысойкин Егор; Вариант 16;
ИУ5-64Б}\label{ux440ux43a1-ux442ux43cux43e-ux441ux44bux441ux43eux439ux43aux438ux43d-ux435ux433ux43eux440-ux432ux430ux440ux438ux430ux43dux442-16-ux438ux4435-64ux431}}

    \hypertarget{ux437ux430ux434ux430ux43dux438ux435.}{%
\subsection{Задание.}\label{ux437ux430ux434ux430ux43dux438ux435.}}

Для заданного набора данных (по Вашему варианту) постройте модели
классификации или регрессии (в зависимости от конкретной задачи,
рассматриваемой в наборе данных). Для построения моделей используйте
методы линейная/логистическая регрессия и градиентный бустинг. Оцените
качество моделей на основе подходящих метрик качества (не менее двух
метрик). Какие метрики качества Вы использовали и почему? Какие выводы
Вы можете сделать о качестве построенных моделей? Для построения моделей
необходимо выполнить требуемую предобработку данных: заполнение
пропусков, кодирование категориальных признаков, и т.д.

    \begin{itemize}
\tightlist
\item
  При решении задач можно выбирать любое подмножество признаков из
  приведенного набора данных.
\item
  Для сокращения времени построения моделей можно использовать фрагмент
  набора данных (например, первые 200-500 строк).
\end{itemize}

    Датасет:
https://www.kaggle.com/san-francisco/sf-restaurant-scores-lives-standard

    \begin{tcolorbox}[breakable, size=fbox, boxrule=1pt, pad at break*=1mm,colback=cellbackground, colframe=cellborder]
\prompt{In}{incolor}{1}{\boxspacing}
\begin{Verbatim}[commandchars=\\\{\}]
\PY{k+kn}{import} \PY{n+nn}{numpy} \PY{k}{as} \PY{n+nn}{np}
\PY{k+kn}{import} \PY{n+nn}{pandas} \PY{k}{as} \PY{n+nn}{pd}
\PY{k+kn}{import} \PY{n+nn}{seaborn} \PY{k}{as} \PY{n+nn}{sns}
\PY{k+kn}{import} \PY{n+nn}{matplotlib}\PY{n+nn}{.}\PY{n+nn}{pyplot} \PY{k}{as} \PY{n+nn}{plt}

\PY{k+kn}{from} \PY{n+nn}{sklearn}\PY{n+nn}{.}\PY{n+nn}{linear\PYZus{}model} \PY{k+kn}{import} \PY{n}{LinearRegression}
\PY{k+kn}{from} \PY{n+nn}{sklearn}\PY{n+nn}{.}\PY{n+nn}{linear\PYZus{}model} \PY{k+kn}{import} \PY{n}{LogisticRegression}
\end{Verbatim}
\end{tcolorbox}

    \begin{tcolorbox}[breakable, size=fbox, boxrule=1pt, pad at break*=1mm,colback=cellbackground, colframe=cellborder]
\prompt{In}{incolor}{2}{\boxspacing}
\begin{Verbatim}[commandchars=\\\{\}]
\PY{n}{data} \PY{o}{=} \PY{n}{pd}\PY{o}{.}\PY{n}{read\PYZus{}csv}\PY{p}{(}\PY{l+s+s2}{\PYZdq{}}\PY{l+s+s2}{data/restaurant\PYZhy{}scores\PYZhy{}lives\PYZhy{}standard.csv}\PY{l+s+s2}{\PYZdq{}}\PY{p}{,} \PY{n}{sep}\PY{o}{=}\PY{l+s+s1}{\PYZsq{}}\PY{l+s+s1}{,}\PY{l+s+s1}{\PYZsq{}}\PY{p}{)}
\PY{n}{data}\PY{o}{.}\PY{n}{dtypes}
\end{Verbatim}
\end{tcolorbox}

            \begin{tcolorbox}[breakable, size=fbox, boxrule=.5pt, pad at break*=1mm, opacityfill=0]
\prompt{Out}{outcolor}{2}{\boxspacing}
\begin{Verbatim}[commandchars=\\\{\}]
business\_id                    int64
business\_name                 object
business\_address              object
business\_city                 object
business\_state                object
business\_postal\_code          object
business\_latitude            float64
business\_longitude           float64
business\_location             object
business\_phone\_number        float64
inspection\_id                 object
inspection\_date               object
inspection\_score             float64
inspection\_type               object
violation\_id                  object
violation\_description         object
risk\_category                 object
Neighborhoods (old)          float64
Police Districts             float64
Supervisor Districts         float64
Fire Prevention Districts    float64
Zip Codes                    float64
Analysis Neighborhoods       float64
dtype: object
\end{Verbatim}
\end{tcolorbox}
        
    \begin{tcolorbox}[breakable, size=fbox, boxrule=1pt, pad at break*=1mm,colback=cellbackground, colframe=cellborder]
\prompt{In}{incolor}{3}{\boxspacing}
\begin{Verbatim}[commandchars=\\\{\}]
\PY{n}{data}\PY{o}{.}\PY{n}{isnull}\PY{p}{(}\PY{p}{)}\PY{o}{.}\PY{n}{sum}\PY{p}{(}\PY{p}{)}
\end{Verbatim}
\end{tcolorbox}

            \begin{tcolorbox}[breakable, size=fbox, boxrule=.5pt, pad at break*=1mm, opacityfill=0]
\prompt{Out}{outcolor}{3}{\boxspacing}
\begin{Verbatim}[commandchars=\\\{\}]
business\_id                      0
business\_name                    0
business\_address                 0
business\_city                    0
business\_state                   0
business\_postal\_code          1018
business\_latitude            19556
business\_longitude           19556
business\_location            19556
business\_phone\_number        36938
inspection\_id                    0
inspection\_date                  0
inspection\_score             13610
inspection\_type                  0
violation\_id                 12870
violation\_description        12870
risk\_category                12870
Neighborhoods (old)          19594
Police Districts             19594
Supervisor Districts         19594
Fire Prevention Districts    19646
Zip Codes                    19576
Analysis Neighborhoods       19594
dtype: int64
\end{Verbatim}
\end{tcolorbox}
        
    \begin{tcolorbox}[breakable, size=fbox, boxrule=1pt, pad at break*=1mm,colback=cellbackground, colframe=cellborder]
\prompt{In}{incolor}{4}{\boxspacing}
\begin{Verbatim}[commandchars=\\\{\}]
\PY{n}{data}\PY{o}{.}\PY{n}{shape}
\end{Verbatim}
\end{tcolorbox}

            \begin{tcolorbox}[breakable, size=fbox, boxrule=.5pt, pad at break*=1mm, opacityfill=0]
\prompt{Out}{outcolor}{4}{\boxspacing}
\begin{Verbatim}[commandchars=\\\{\}]
(53973, 23)
\end{Verbatim}
\end{tcolorbox}
        
    \begin{tcolorbox}[breakable, size=fbox, boxrule=1pt, pad at break*=1mm,colback=cellbackground, colframe=cellborder]
\prompt{In}{incolor}{5}{\boxspacing}
\begin{Verbatim}[commandchars=\\\{\}]
\PY{n}{data}\PY{o}{.}\PY{n}{head}\PY{p}{(}\PY{p}{)}
\end{Verbatim}
\end{tcolorbox}

            \begin{tcolorbox}[breakable, size=fbox, boxrule=.5pt, pad at break*=1mm, opacityfill=0]
\prompt{Out}{outcolor}{5}{\boxspacing}
\begin{Verbatim}[commandchars=\\\{\}]
   business\_id          business\_name          business\_address  \textbackslash{}
0       101192           Cochinita \#2  2 Marina Blvd Fort Mason
1        97975             BREADBELLY           1408 Clement St
2        92982  Great Gold Restaurant             3161 24th St.
3       101389                 HOMAGE         214 CALIFORNIA ST
4        85986           Pronto Pizza               798 Eddy St

   business\_city business\_state business\_postal\_code  business\_latitude  \textbackslash{}
0  San Francisco             CA                  NaN                NaN
1  San Francisco             CA                94118                NaN
2  San Francisco             CA                94110                NaN
3  San Francisco             CA                94111                NaN
4  San Francisco             CA                94109                NaN

   business\_longitude business\_location  business\_phone\_number  {\ldots}  \textbackslash{}
0                 NaN               NaN           1.415043e+10  {\ldots}
1                 NaN               NaN           1.415724e+10  {\ldots}
2                 NaN               NaN                    NaN  {\ldots}
3                 NaN               NaN           1.415488e+10  {\ldots}
4                 NaN               NaN                    NaN  {\ldots}

         inspection\_type           violation\_id  \textbackslash{}
0          New Ownership                    NaN
1  Routine - Unscheduled  97975\_20190725\_103124
2          New Ownership                    NaN
3       New Construction                    NaN
4          New Ownership  85986\_20161011\_103114

                               violation\_description  risk\_category  \textbackslash{}
0                                                NaN            NaN
1  Inadequately cleaned or sanitized food contact{\ldots}  Moderate Risk
2                                                NaN            NaN
3                                                NaN            NaN
4                       High risk vermin infestation      High Risk

  Neighborhoods (old) Police Districts Supervisor Districts  \textbackslash{}
0                 NaN              NaN                  NaN
1                 NaN              NaN                  NaN
2                 NaN              NaN                  NaN
3                 NaN              NaN                  NaN
4                 NaN              NaN                  NaN

   Fire Prevention Districts  Zip Codes  Analysis Neighborhoods
0                        NaN        NaN                     NaN
1                        NaN        NaN                     NaN
2                        NaN        NaN                     NaN
3                        NaN        NaN                     NaN
4                        NaN        NaN                     NaN

[5 rows x 23 columns]
\end{Verbatim}
\end{tcolorbox}
        
    \begin{tcolorbox}[breakable, size=fbox, boxrule=1pt, pad at break*=1mm,colback=cellbackground, colframe=cellborder]
\prompt{In}{incolor}{6}{\boxspacing}
\begin{Verbatim}[commandchars=\\\{\}]
\PY{n}{data2} \PY{o}{=} \PY{n}{data}\PY{o}{.}\PY{n}{copy}\PY{p}{(}\PY{p}{)}\PY{o}{.}\PY{n}{dropna}\PY{p}{(}\PY{n}{axis}\PY{o}{=}\PY{l+m+mi}{0}\PY{p}{,} \PY{n}{how}\PY{o}{=}\PY{l+s+s1}{\PYZsq{}}\PY{l+s+s1}{any}\PY{l+s+s1}{\PYZsq{}}\PY{p}{)}
\PY{n}{data2}\PY{o}{.}\PY{n}{drop\PYZus{}duplicates}\PY{p}{(}\PY{n}{keep}\PY{o}{=}\PY{k+kc}{False}\PY{p}{,}\PY{n}{inplace}\PY{o}{=}\PY{k+kc}{True}\PY{p}{)}
\end{Verbatim}
\end{tcolorbox}

    \begin{tcolorbox}[breakable, size=fbox, boxrule=1pt, pad at break*=1mm,colback=cellbackground, colframe=cellborder]
\prompt{In}{incolor}{7}{\boxspacing}
\begin{Verbatim}[commandchars=\\\{\}]
\PY{k}{for} \PY{n}{col} \PY{o+ow}{in} \PY{n}{data2}\PY{o}{.}\PY{n}{columns}\PY{p}{:}
    \PY{n}{unique\PYZus{}nums} \PY{o}{=} \PY{n}{data2}\PY{p}{[}\PY{n}{col}\PY{p}{]}\PY{o}{.}\PY{n}{unique}\PY{p}{(}\PY{p}{)}
    \PY{k}{if} \PY{n}{unique\PYZus{}nums}\PY{o}{.}\PY{n}{size} \PY{o}{\PYZlt{}} \PY{l+m+mi}{10}\PY{p}{:}
        \PY{n+nb}{print}\PY{p}{(}\PY{l+s+s2}{\PYZdq{}}\PY{l+s+si}{\PYZob{}\PYZcb{}}\PY{l+s+s2}{: }\PY{l+s+si}{\PYZob{}\PYZcb{}}\PY{l+s+s2}{\PYZdq{}}\PY{o}{.}\PY{n}{format}\PY{p}{(}\PY{n}{col}\PY{p}{,} \PY{n}{unique\PYZus{}nums}\PY{p}{)}\PY{p}{)}
\end{Verbatim}
\end{tcolorbox}

    \begin{Verbatim}[commandchars=\\\{\}]
business\_city: ['San Francisco']
business\_state: ['CA']
inspection\_type: ['Routine - Unscheduled']
risk\_category: ['Low Risk' 'High Risk' 'Moderate Risk']
    \end{Verbatim}

    business\_city: {[}`San Francisco'{]}, business\_state: {[}`CA'{]},
inspection\_type: {[}`Routine - Unscheduled'{]} - имеют 1 уникальное
значение. Можно убрать.

    \begin{tcolorbox}[breakable, size=fbox, boxrule=1pt, pad at break*=1mm,colback=cellbackground, colframe=cellborder]
\prompt{In}{incolor}{8}{\boxspacing}
\begin{Verbatim}[commandchars=\\\{\}]
\PY{n}{fig}\PY{p}{,} \PY{n}{ax} \PY{o}{=} \PY{n}{plt}\PY{o}{.}\PY{n}{subplots}\PY{p}{(}\PY{n}{figsize}\PY{o}{=}\PY{p}{(}\PY{l+m+mi}{15}\PY{p}{,}\PY{l+m+mi}{7}\PY{p}{)}\PY{p}{)}
\PY{n}{sns}\PY{o}{.}\PY{n}{heatmap}\PY{p}{(}\PY{n}{data2}\PY{o}{.}\PY{n}{corr}\PY{p}{(}\PY{n}{method}\PY{o}{=}\PY{l+s+s1}{\PYZsq{}}\PY{l+s+s1}{pearson}\PY{l+s+s1}{\PYZsq{}}\PY{p}{)}\PY{p}{,} \PY{n}{ax}\PY{o}{=}\PY{n}{ax}\PY{p}{,} \PY{n}{annot}\PY{o}{=}\PY{k+kc}{True}\PY{p}{,} \PY{n}{fmt}\PY{o}{=}\PY{l+s+s1}{\PYZsq{}}\PY{l+s+s1}{.2f}\PY{l+s+s1}{\PYZsq{}}\PY{p}{)}
\end{Verbatim}
\end{tcolorbox}

            \begin{tcolorbox}[breakable, size=fbox, boxrule=.5pt, pad at break*=1mm, opacityfill=0]
\prompt{Out}{outcolor}{8}{\boxspacing}
\begin{Verbatim}[commandchars=\\\{\}]
<AxesSubplot:>
\end{Verbatim}
\end{tcolorbox}
        
    \begin{center}
    \adjustimage{max size={0.9\linewidth}{0.9\paperheight}}{output_12_1.png}
    \end{center}
    { \hspace*{\fill} \\}
    
    \begin{tcolorbox}[breakable, size=fbox, boxrule=1pt, pad at break*=1mm,colback=cellbackground, colframe=cellborder]
\prompt{In}{incolor}{9}{\boxspacing}
\begin{Verbatim}[commandchars=\\\{\}]
\PY{n}{data2}\PY{p}{[}\PY{l+s+s2}{\PYZdq{}}\PY{l+s+s2}{risk\PYZus{}category}\PY{l+s+s2}{\PYZdq{}}\PY{p}{]} \PY{o}{=} \PY{n}{data2}\PY{p}{[}\PY{l+s+s2}{\PYZdq{}}\PY{l+s+s2}{risk\PYZus{}category}\PY{l+s+s2}{\PYZdq{}}\PY{p}{]}\PY{o}{.}\PY{n}{astype}\PY{p}{(}\PY{l+s+s1}{\PYZsq{}}\PY{l+s+s1}{category}\PY{l+s+s1}{\PYZsq{}}\PY{p}{)}
\PY{n}{data2}\PY{p}{[}\PY{l+s+s2}{\PYZdq{}}\PY{l+s+s2}{risk\PYZus{}category\PYZus{}cat}\PY{l+s+s2}{\PYZdq{}}\PY{p}{]} \PY{o}{=} \PY{n}{data2}\PY{p}{[}\PY{l+s+s2}{\PYZdq{}}\PY{l+s+s2}{risk\PYZus{}category}\PY{l+s+s2}{\PYZdq{}}\PY{p}{]}\PY{o}{.}\PY{n}{cat}\PY{o}{.}\PY{n}{codes}

\PY{n}{data2}\PY{o}{.}\PY{n}{drop}\PY{p}{(}\PY{p}{[}\PY{l+s+s2}{\PYZdq{}}\PY{l+s+s2}{business\PYZus{}city}\PY{l+s+s2}{\PYZdq{}}\PY{p}{,} \PY{l+s+s2}{\PYZdq{}}\PY{l+s+s2}{business\PYZus{}state}\PY{l+s+s2}{\PYZdq{}}\PY{p}{,} \PY{l+s+s2}{\PYZdq{}}\PY{l+s+s2}{business\PYZus{}location}\PY{l+s+s2}{\PYZdq{}}\PY{p}{,} \PY{l+s+s2}{\PYZdq{}}\PY{l+s+s2}{business\PYZus{}name}\PY{l+s+s2}{\PYZdq{}}\PY{p}{,} 
            \PY{l+s+s2}{\PYZdq{}}\PY{l+s+s2}{business\PYZus{}address}\PY{l+s+s2}{\PYZdq{}}\PY{p}{,} \PY{l+s+s2}{\PYZdq{}}\PY{l+s+s2}{violation\PYZus{}description}\PY{l+s+s2}{\PYZdq{}}\PY{p}{,} \PY{l+s+s2}{\PYZdq{}}\PY{l+s+s2}{risk\PYZus{}category}\PY{l+s+s2}{\PYZdq{}}\PY{p}{,} \PY{l+s+s2}{\PYZdq{}}\PY{l+s+s2}{Neighborhoods (old)}\PY{l+s+s2}{\PYZdq{}}\PY{p}{,} \PY{l+s+s2}{\PYZdq{}}\PY{l+s+s2}{inspection\PYZus{}type}\PY{l+s+s2}{\PYZdq{}}\PY{p}{,}
            \PY{l+s+s2}{\PYZdq{}}\PY{l+s+s2}{inspection\PYZus{}id}\PY{l+s+s2}{\PYZdq{}}\PY{p}{,} \PY{l+s+s2}{\PYZdq{}}\PY{l+s+s2}{violation\PYZus{}id}\PY{l+s+s2}{\PYZdq{}}\PY{p}{,} \PY{l+s+s2}{\PYZdq{}}\PY{l+s+s2}{inspection\PYZus{}date}\PY{l+s+s2}{\PYZdq{}}
           \PY{p}{]}\PY{p}{,} 
           \PY{n}{axis}\PY{o}{=}\PY{l+m+mi}{1}\PY{p}{,} \PY{n}{inplace}\PY{o}{=}\PY{k+kc}{True}\PY{p}{)}
\end{Verbatim}
\end{tcolorbox}

    \begin{tcolorbox}[breakable, size=fbox, boxrule=1pt, pad at break*=1mm,colback=cellbackground, colframe=cellborder]
\prompt{In}{incolor}{10}{\boxspacing}
\begin{Verbatim}[commandchars=\\\{\}]
\PY{n}{data2}\PY{p}{[}\PY{l+s+s2}{\PYZdq{}}\PY{l+s+s2}{business\PYZus{}postal\PYZus{}code}\PY{l+s+s2}{\PYZdq{}}\PY{p}{]}\PY{o}{.}\PY{n}{unique}\PY{p}{(}\PY{p}{)}
\end{Verbatim}
\end{tcolorbox}

            \begin{tcolorbox}[breakable, size=fbox, boxrule=.5pt, pad at break*=1mm, opacityfill=0]
\prompt{Out}{outcolor}{10}{\boxspacing}
\begin{Verbatim}[commandchars=\\\{\}]
array(['94107', '94131', '94112', '94121', '94110', '94109', '94115',
       '94111', '94118', '94103', '94134', '94117', '94114', '94123',
       '94124', '94104', '94122', '94108', '94133', '94132', '941102019',
       '94127', '94102', '92672', '94105', '94116', '94158'], dtype=object)
\end{Verbatim}
\end{tcolorbox}
        
    \begin{tcolorbox}[breakable, size=fbox, boxrule=1pt, pad at break*=1mm,colback=cellbackground, colframe=cellborder]
\prompt{In}{incolor}{11}{\boxspacing}
\begin{Verbatim}[commandchars=\\\{\}]
\PY{n}{data2}\PY{p}{[}\PY{l+s+s2}{\PYZdq{}}\PY{l+s+s2}{business\PYZus{}postal\PYZus{}code}\PY{l+s+s2}{\PYZdq{}}\PY{p}{]} \PY{o}{=} \PY{n}{data2}\PY{p}{[}\PY{l+s+s2}{\PYZdq{}}\PY{l+s+s2}{business\PYZus{}postal\PYZus{}code}\PY{l+s+s2}{\PYZdq{}}\PY{p}{]}\PY{o}{.}\PY{n}{astype}\PY{p}{(}\PY{n+nb}{int}\PY{p}{)}
\end{Verbatim}
\end{tcolorbox}

    \begin{tcolorbox}[breakable, size=fbox, boxrule=1pt, pad at break*=1mm,colback=cellbackground, colframe=cellborder]
\prompt{In}{incolor}{12}{\boxspacing}
\begin{Verbatim}[commandchars=\\\{\}]
\PY{n}{data2}\PY{p}{[}\PY{l+s+s2}{\PYZdq{}}\PY{l+s+s2}{Police Districts}\PY{l+s+s2}{\PYZdq{}}\PY{p}{]} \PY{o}{=} \PY{n}{data2}\PY{p}{[}\PY{l+s+s2}{\PYZdq{}}\PY{l+s+s2}{Police Districts}\PY{l+s+s2}{\PYZdq{}}\PY{p}{]}\PY{o}{.}\PY{n}{astype}\PY{p}{(}\PY{n+nb}{int}\PY{p}{)}
\PY{n}{data2}\PY{p}{[}\PY{l+s+s2}{\PYZdq{}}\PY{l+s+s2}{inspection\PYZus{}score}\PY{l+s+s2}{\PYZdq{}}\PY{p}{]} \PY{o}{=} \PY{n}{data2}\PY{p}{[}\PY{l+s+s2}{\PYZdq{}}\PY{l+s+s2}{inspection\PYZus{}score}\PY{l+s+s2}{\PYZdq{}}\PY{p}{]}\PY{o}{.}\PY{n}{astype}\PY{p}{(}\PY{n+nb}{int}\PY{p}{)}
\PY{n}{data2}\PY{p}{[}\PY{l+s+s2}{\PYZdq{}}\PY{l+s+s2}{Supervisor Districts}\PY{l+s+s2}{\PYZdq{}}\PY{p}{]} \PY{o}{=} \PY{n}{data2}\PY{p}{[}\PY{l+s+s2}{\PYZdq{}}\PY{l+s+s2}{Supervisor Districts}\PY{l+s+s2}{\PYZdq{}}\PY{p}{]}\PY{o}{.}\PY{n}{astype}\PY{p}{(}\PY{n+nb}{int}\PY{p}{)}
\PY{n}{data2}\PY{p}{[}\PY{l+s+s2}{\PYZdq{}}\PY{l+s+s2}{Fire Prevention Districts}\PY{l+s+s2}{\PYZdq{}}\PY{p}{]} \PY{o}{=} \PY{n}{data2}\PY{p}{[}\PY{l+s+s2}{\PYZdq{}}\PY{l+s+s2}{Fire Prevention Districts}\PY{l+s+s2}{\PYZdq{}}\PY{p}{]}\PY{o}{.}\PY{n}{astype}\PY{p}{(}\PY{n+nb}{int}\PY{p}{)}
\PY{n}{data2}\PY{p}{[}\PY{l+s+s2}{\PYZdq{}}\PY{l+s+s2}{Zip Codes}\PY{l+s+s2}{\PYZdq{}}\PY{p}{]} \PY{o}{=} \PY{n}{data2}\PY{p}{[}\PY{l+s+s2}{\PYZdq{}}\PY{l+s+s2}{Zip Codes}\PY{l+s+s2}{\PYZdq{}}\PY{p}{]}\PY{o}{.}\PY{n}{astype}\PY{p}{(}\PY{n+nb}{int}\PY{p}{)}
\PY{n}{data2}\PY{p}{[}\PY{l+s+s2}{\PYZdq{}}\PY{l+s+s2}{Analysis Neighborhoods}\PY{l+s+s2}{\PYZdq{}}\PY{p}{]} \PY{o}{=} \PY{n}{data2}\PY{p}{[}\PY{l+s+s2}{\PYZdq{}}\PY{l+s+s2}{Analysis Neighborhoods}\PY{l+s+s2}{\PYZdq{}}\PY{p}{]}\PY{o}{.}\PY{n}{astype}\PY{p}{(}\PY{n+nb}{int}\PY{p}{)}
\end{Verbatim}
\end{tcolorbox}

    \begin{tcolorbox}[breakable, size=fbox, boxrule=1pt, pad at break*=1mm,colback=cellbackground, colframe=cellborder]
\prompt{In}{incolor}{13}{\boxspacing}
\begin{Verbatim}[commandchars=\\\{\}]
\PY{n}{data2}\PY{o}{.}\PY{n}{isnull}\PY{p}{(}\PY{p}{)}\PY{o}{.}\PY{n}{sum}\PY{p}{(}\PY{p}{)}
\end{Verbatim}
\end{tcolorbox}

            \begin{tcolorbox}[breakable, size=fbox, boxrule=.5pt, pad at break*=1mm, opacityfill=0]
\prompt{Out}{outcolor}{13}{\boxspacing}
\begin{Verbatim}[commandchars=\\\{\}]
business\_id                  0
business\_postal\_code         0
business\_latitude            0
business\_longitude           0
business\_phone\_number        0
inspection\_score             0
Police Districts             0
Supervisor Districts         0
Fire Prevention Districts    0
Zip Codes                    0
Analysis Neighborhoods       0
risk\_category\_cat            0
dtype: int64
\end{Verbatim}
\end{tcolorbox}
        
    \begin{tcolorbox}[breakable, size=fbox, boxrule=1pt, pad at break*=1mm,colback=cellbackground, colframe=cellborder]
\prompt{In}{incolor}{14}{\boxspacing}
\begin{Verbatim}[commandchars=\\\{\}]
\PY{n}{data2}\PY{o}{.}\PY{n}{head}\PY{p}{(}\PY{p}{)}
\end{Verbatim}
\end{tcolorbox}

            \begin{tcolorbox}[breakable, size=fbox, boxrule=.5pt, pad at break*=1mm, opacityfill=0]
\prompt{Out}{outcolor}{14}{\boxspacing}
\begin{Verbatim}[commandchars=\\\{\}]
     business\_id  business\_postal\_code  business\_latitude  business\_longitude  \textbackslash{}
11          4794                 94107          37.778634         -122.393089
372         2684                 94131          37.746759         -122.426995
464         3256                 94112          37.709737         -122.450070
484         3951                 94121          37.779962         -122.485087
496         4864                 94110          37.759174         -122.419066

     business\_phone\_number  inspection\_score  Police Districts  \textbackslash{}
11            1.415561e+10                71                 2
372           1.415528e+10                87                 7
464           1.415534e+10                94                 7
484           1.415539e+10                77                 6
496           1.415583e+10                78                 4

     Supervisor Districts  Fire Prevention Districts  Zip Codes  \textbackslash{}
11                      9                          6      28856
372                     5                          2         63
464                     6                          9      28861
484                     2                         11         55
496                     7                          2      28859

     Analysis Neighborhoods  risk\_category\_cat
11                       34                  1
372                      22                  0
464                      28                  1
484                      29                  1
496                      20                  1
\end{Verbatim}
\end{tcolorbox}
        
    \begin{tcolorbox}[breakable, size=fbox, boxrule=1pt, pad at break*=1mm,colback=cellbackground, colframe=cellborder]
\prompt{In}{incolor}{15}{\boxspacing}
\begin{Verbatim}[commandchars=\\\{\}]
\PY{n}{data2}\PY{o}{.}\PY{n}{dtypes}
\end{Verbatim}
\end{tcolorbox}

            \begin{tcolorbox}[breakable, size=fbox, boxrule=.5pt, pad at break*=1mm, opacityfill=0]
\prompt{Out}{outcolor}{15}{\boxspacing}
\begin{Verbatim}[commandchars=\\\{\}]
business\_id                    int64
business\_postal\_code           int64
business\_latitude            float64
business\_longitude           float64
business\_phone\_number        float64
inspection\_score               int64
Police Districts               int64
Supervisor Districts           int64
Fire Prevention Districts      int64
Zip Codes                      int64
Analysis Neighborhoods         int64
risk\_category\_cat               int8
dtype: object
\end{Verbatim}
\end{tcolorbox}
        
    \begin{tcolorbox}[breakable, size=fbox, boxrule=1pt, pad at break*=1mm,colback=cellbackground, colframe=cellborder]
\prompt{In}{incolor}{16}{\boxspacing}
\begin{Verbatim}[commandchars=\\\{\}]
\PY{n}{target} \PY{o}{=} \PY{l+s+s2}{\PYZdq{}}\PY{l+s+s2}{Supervisor Districts}\PY{l+s+s2}{\PYZdq{}}
\end{Verbatim}
\end{tcolorbox}

    \begin{tcolorbox}[breakable, size=fbox, boxrule=1pt, pad at break*=1mm,colback=cellbackground, colframe=cellborder]
\prompt{In}{incolor}{17}{\boxspacing}
\begin{Verbatim}[commandchars=\\\{\}]
\PY{c+c1}{\PYZsh{} Масштабирование}
\PY{k+kn}{from} \PY{n+nn}{sklearn}\PY{n+nn}{.}\PY{n+nn}{preprocessing} \PY{k+kn}{import} \PY{n}{StandardScaler}

\PY{n}{scaler} \PY{o}{=} \PY{n}{StandardScaler}\PY{p}{(}\PY{p}{)}
\PY{k}{for} \PY{n}{col} \PY{o+ow}{in} \PY{n}{data2}\PY{o}{.}\PY{n}{columns}\PY{p}{:}
  \PY{k}{if} \PY{n}{col} \PY{o}{!=} \PY{n}{target}\PY{p}{:}
    \PY{n}{data2}\PY{p}{[}\PY{n}{col}\PY{p}{]} \PY{o}{=} \PY{n}{scaler}\PY{o}{.}\PY{n}{fit\PYZus{}transform}\PY{p}{(}\PY{n}{data2}\PY{p}{[}\PY{p}{[}\PY{n}{col}\PY{p}{]}\PY{p}{]}\PY{p}{)}
\end{Verbatim}
\end{tcolorbox}

    \begin{tcolorbox}[breakable, size=fbox, boxrule=1pt, pad at break*=1mm,colback=cellbackground, colframe=cellborder]
\prompt{In}{incolor}{18}{\boxspacing}
\begin{Verbatim}[commandchars=\\\{\}]
\PY{n}{fig}\PY{p}{,} \PY{n}{ax} \PY{o}{=} \PY{n}{plt}\PY{o}{.}\PY{n}{subplots}\PY{p}{(}\PY{n}{figsize}\PY{o}{=}\PY{p}{(}\PY{l+m+mi}{15}\PY{p}{,}\PY{l+m+mi}{7}\PY{p}{)}\PY{p}{)}
\PY{n}{sns}\PY{o}{.}\PY{n}{heatmap}\PY{p}{(}\PY{n}{data2}\PY{o}{.}\PY{n}{corr}\PY{p}{(}\PY{n}{method}\PY{o}{=}\PY{l+s+s1}{\PYZsq{}}\PY{l+s+s1}{pearson}\PY{l+s+s1}{\PYZsq{}}\PY{p}{)}\PY{p}{,} \PY{n}{ax}\PY{o}{=}\PY{n}{ax}\PY{p}{,} \PY{n}{annot}\PY{o}{=}\PY{k+kc}{True}\PY{p}{,} \PY{n}{fmt}\PY{o}{=}\PY{l+s+s1}{\PYZsq{}}\PY{l+s+s1}{.2f}\PY{l+s+s1}{\PYZsq{}}\PY{p}{)}
\end{Verbatim}
\end{tcolorbox}

            \begin{tcolorbox}[breakable, size=fbox, boxrule=.5pt, pad at break*=1mm, opacityfill=0]
\prompt{Out}{outcolor}{18}{\boxspacing}
\begin{Verbatim}[commandchars=\\\{\}]
<AxesSubplot:>
\end{Verbatim}
\end{tcolorbox}
        
    \begin{center}
    \adjustimage{max size={0.9\linewidth}{0.9\paperheight}}{output_22_1.png}
    \end{center}
    { \hspace*{\fill} \\}
    
    \begin{tcolorbox}[breakable, size=fbox, boxrule=1pt, pad at break*=1mm,colback=cellbackground, colframe=cellborder]
\prompt{In}{incolor}{19}{\boxspacing}
\begin{Verbatim}[commandchars=\\\{\}]
\PY{k+kn}{from} \PY{n+nn}{sklearn}\PY{n+nn}{.}\PY{n+nn}{model\PYZus{}selection} \PY{k+kn}{import} \PY{n}{train\PYZus{}test\PYZus{}split}

\PY{n}{feature\PYZus{}cols} \PY{o}{=} \PY{p}{[}\PY{l+s+s2}{\PYZdq{}}\PY{l+s+s2}{business\PYZus{}latitude}\PY{l+s+s2}{\PYZdq{}}\PY{p}{,} \PY{l+s+s2}{\PYZdq{}}\PY{l+s+s2}{business\PYZus{}longitude}\PY{l+s+s2}{\PYZdq{}}\PY{p}{,} \PY{l+s+s2}{\PYZdq{}}\PY{l+s+s2}{Zip Codes}\PY{l+s+s2}{\PYZdq{}}\PY{p}{]}

\PY{n}{X\PYZus{}train}\PY{p}{,} \PY{n}{X\PYZus{}test}\PY{p}{,} \PY{n}{y\PYZus{}train}\PY{p}{,} \PY{n}{y\PYZus{}test} \PY{o}{=} \PY{n}{train\PYZus{}test\PYZus{}split}\PY{p}{(}
  \PY{n}{data2}\PY{p}{[}\PY{n}{feature\PYZus{}cols}\PY{p}{]}\PY{p}{,}
  \PY{n}{data2}\PY{p}{[}\PY{n}{target}\PY{p}{]}\PY{p}{,}
  \PY{n}{test\PYZus{}size}\PY{o}{=}\PY{l+m+mf}{0.3}\PY{p}{,}
  \PY{n}{random\PYZus{}state}\PY{o}{=}\PY{l+m+mi}{1}\PY{p}{,}
\PY{p}{)}
\end{Verbatim}
\end{tcolorbox}

    \hypertarget{ux43bux438ux43dux435ux439ux43dux430ux44f-ux440ux435ux433ux440ux435ux435ux441ux441ux438ux44f}{%
\subsubsection{Линейная
регреессия}\label{ux43bux438ux43dux435ux439ux43dux430ux44f-ux440ux435ux433ux440ux435ux435ux441ux441ux438ux44f}}

    \begin{tcolorbox}[breakable, size=fbox, boxrule=1pt, pad at break*=1mm,colback=cellbackground, colframe=cellborder]
\prompt{In}{incolor}{20}{\boxspacing}
\begin{Verbatim}[commandchars=\\\{\}]
\PY{k+kn}{from} \PY{n+nn}{sklearn}\PY{n+nn}{.}\PY{n+nn}{linear\PYZus{}model} \PY{k+kn}{import} \PY{n}{LinearRegression}

\PY{n}{linreg} \PY{o}{=} \PY{n}{LinearRegression}\PY{p}{(}\PY{p}{)}\PY{o}{.}\PY{n}{fit}\PY{p}{(}\PY{n}{X\PYZus{}train}\PY{p}{,} \PY{n}{y\PYZus{}train}\PY{p}{)}
\end{Verbatim}
\end{tcolorbox}

    \begin{tcolorbox}[breakable, size=fbox, boxrule=1pt, pad at break*=1mm,colback=cellbackground, colframe=cellborder]
\prompt{In}{incolor}{21}{\boxspacing}
\begin{Verbatim}[commandchars=\\\{\}]
\PY{k+kn}{from} \PY{n+nn}{sklearn}\PY{n+nn}{.}\PY{n+nn}{metrics} \PY{k+kn}{import} \PY{n}{r2\PYZus{}score}\PY{p}{,} \PY{n}{mean\PYZus{}absolute\PYZus{}error}

\PY{n}{linreg\PYZus{}predict} \PY{o}{=} \PY{n}{linreg}\PY{o}{.}\PY{n}{predict}\PY{p}{(}\PY{n}{X\PYZus{}test}\PY{p}{)}
\PY{n}{r2\PYZus{}score}\PY{p}{(}\PY{n}{y\PYZus{}test}\PY{p}{,} \PY{n}{linreg\PYZus{}predict}\PY{p}{)}\PY{p}{,} \PYZbs{}
  \PY{n}{mean\PYZus{}absolute\PYZus{}error}\PY{p}{(}\PY{n}{y\PYZus{}test}\PY{p}{,} \PY{n}{linreg\PYZus{}predict}\PY{p}{)}
\end{Verbatim}
\end{tcolorbox}

            \begin{tcolorbox}[breakable, size=fbox, boxrule=.5pt, pad at break*=1mm, opacityfill=0]
\prompt{Out}{outcolor}{21}{\boxspacing}
\begin{Verbatim}[commandchars=\\\{\}]
(0.32634156097620515, 1.8639363772620643)
\end{Verbatim}
\end{tcolorbox}
        
    \hypertarget{ux433ux440ux430ux434ux438ux435ux43dux442ux43dux44bux439-ux431ux443ux441ux442ux438ux43dux433}{%
\subsubsection{Градиентный
бустинг}\label{ux433ux440ux430ux434ux438ux435ux43dux442ux43dux44bux439-ux431ux443ux441ux442ux438ux43dux433}}

    \begin{tcolorbox}[breakable, size=fbox, boxrule=1pt, pad at break*=1mm,colback=cellbackground, colframe=cellborder]
\prompt{In}{incolor}{22}{\boxspacing}
\begin{Verbatim}[commandchars=\\\{\}]
\PY{k+kn}{from} \PY{n+nn}{sklearn}\PY{n+nn}{.}\PY{n+nn}{ensemble} \PY{k+kn}{import} \PY{n}{GradientBoostingRegressor}

\PY{n}{gboostreg} \PY{o}{=} \PY{n}{GradientBoostingRegressor}\PY{p}{(}\PY{n}{random\PYZus{}state}\PY{o}{=}\PY{l+m+mi}{10}\PY{p}{)}\PY{o}{.}\PY{n}{fit}\PY{p}{(}\PY{n}{X\PYZus{}train}\PY{p}{,} \PY{n}{y\PYZus{}train}\PY{p}{)}
\end{Verbatim}
\end{tcolorbox}

    \begin{tcolorbox}[breakable, size=fbox, boxrule=1pt, pad at break*=1mm,colback=cellbackground, colframe=cellborder]
\prompt{In}{incolor}{23}{\boxspacing}
\begin{Verbatim}[commandchars=\\\{\}]
\PY{n}{gboostreg\PYZus{}predict} \PY{o}{=} \PY{n}{gboostreg}\PY{o}{.}\PY{n}{predict}\PY{p}{(}\PY{n}{X\PYZus{}test}\PY{p}{)}
\PY{n}{r2\PYZus{}score}\PY{p}{(}\PY{n}{y\PYZus{}test}\PY{p}{,} \PY{n}{gboostreg\PYZus{}predict}\PY{p}{)}\PY{p}{,} \PYZbs{}
  \PY{n}{mean\PYZus{}absolute\PYZus{}error}\PY{p}{(}\PY{n}{y\PYZus{}test}\PY{p}{,} \PY{n}{gboostreg\PYZus{}predict}\PY{p}{)}
\end{Verbatim}
\end{tcolorbox}

            \begin{tcolorbox}[breakable, size=fbox, boxrule=.5pt, pad at break*=1mm, opacityfill=0]
\prompt{Out}{outcolor}{23}{\boxspacing}
\begin{Verbatim}[commandchars=\\\{\}]
(0.9155392208894986, 0.4158270828303186)
\end{Verbatim}
\end{tcolorbox}
        
    \hypertarget{ux432ux44bux432ux43eux434}{%
\subsection{Вывод}\label{ux432ux44bux432ux43eux434}}

    Как видно по тепловой карте, данные плохо коррелируют друг с другом.
Поэтому для построения модели был выбрал целевой признак ``Supervisor
Districts'', а в качестве ключевых признаков -
{[}``business\_latitude'', ``business\_longitude'', ``Zip Codes''{]}.
Как видно по оценкам, модель линейной регрессии недообучается, а модель
градиентного бустинга хорошо обучается. Вторая модель имеет высокую
оценку r2(близкую к 1) и низкую абсолютную ошибку(\textless1, что для
целочисленного признака дает хороший результат).
    
    
    
\end{document}
