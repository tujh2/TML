\documentclass[11pt]{report}
\usepackage[T2A,T1]{fontenc}
\usepackage[utf8]{inputenc}
\usepackage{blindtext}
\usepackage[russian]{babel}
\usepackage{mwe}
\usepackage{graphbox}
\usepackage[document]{ragged2e}
\usepackage[margin=50pt]{geometry}
\usepackage{longtable}
\usepackage{fontspec}
\usepackage{float}
\usepackage{titlesec}
\usepackage{setspace}
\usepackage{minted}
\usepackage{titlesec}
\usepackage{titletoc}

\usepackage{hyperref}
\hypersetup{
    colorlinks,
    citecolor=black,
    filecolor=black,
    linkcolor=black,
    urlcolor=black
}

\setstretch{1}
\graphicspath{{./images/}}
\setmainfont[Scale=0.8]{LiberationSerif}
\setmonofont[Scale=0.8]{Hack}
\titleformat{\chapter}{\normalfont\LARGE\bfseries}{\thechapter}{1em}{}
\titleclass{\chapter}{straight}
\titlespacing{\chapter}{0pt}{0pt}{5pt}[25pt]


    \usepackage[breakable]{tcolorbox}
    \usepackage{parskip} % Stop auto-indenting (to mimic markdown behaviour)
    
    \usepackage{iftex}
    \ifPDFTeX
    	\usepackage[T1]{fontenc}
    	\usepackage{mathpazo}
    \else
    	\usepackage{fontspec}
    \fi

    % Basic figure setup, for now with no caption control since it's done
    % automatically by Pandoc (which extracts ![](path) syntax from Markdown).
    \usepackage{graphicx}
    % Maintain compatibility with old templates. Remove in nbconvert 6.0
    \let\Oldincludegraphics\includegraphics
    % Ensure that by default, figures have no caption (until we provide a
    % proper Figure object with a Caption API and a way to capture that
    % in the conversion process - todo).
    \usepackage{caption}
    \DeclareCaptionFormat{nocaption}{}
    \captionsetup{format=nocaption,aboveskip=0pt,belowskip=0pt}

    \usepackage{float}
    \floatplacement{figure}{H} % forces figures to be placed at the correct location
    \usepackage{xcolor} % Allow colors to be defined
    \usepackage{enumerate} % Needed for markdown enumerations to work
    \usepackage{geometry} % Used to adjust the document margins
    \usepackage{amsmath} % Equations
    \usepackage{amssymb} % Equations
    \usepackage{textcomp} % defines textquotesingle
    % Hack from http://tex.stackexchange.com/a/47451/13684:
    \AtBeginDocument{%
        \def\PYZsq{\textquotesingle}% Upright quotes in Pygmentized code
    }
    \usepackage{upquote} % Upright quotes for verbatim code
    \usepackage{eurosym} % defines \euro
    \usepackage[mathletters]{ucs} % Extended unicode (utf-8) support
    \usepackage{fancyvrb} % verbatim replacement that allows latex
    \usepackage{grffile} % extends the file name processing of package graphics 
                         % to support a larger range
    \makeatletter % fix for old versions of grffile with XeLaTeX
    \@ifpackagelater{grffile}{2019/11/01}
    {
      % Do nothing on new versions
    }
    {
      \def\Gread@@xetex#1{%
        \IfFileExists{"\Gin@base".bb}%
        {\Gread@eps{\Gin@base.bb}}%
        {\Gread@@xetex@aux#1}%
      }
    }
    \makeatother
    \usepackage[Export]{adjustbox} % Used to constrain images to a maximum size
    \adjustboxset{max size={0.9\linewidth}{0.9\paperheight}}

    % The hyperref package gives us a pdf with properly built
    % internal navigation ('pdf bookmarks' for the table of contents,
    % internal cross-reference links, web links for URLs, etc.)
    \usepackage{hyperref}
    % The default LaTeX title has an obnoxious amount of whitespace. By default,
    % titling removes some of it. It also provides customization options.
    \usepackage{titling}
    \usepackage{longtable} % longtable support required by pandoc >1.10
    \usepackage{booktabs}  % table support for pandoc > 1.12.2
    \usepackage[inline]{enumitem} % IRkernel/repr support (it uses the enumerate* environment)
    \usepackage[normalem]{ulem} % ulem is needed to support strikethroughs (\sout)
                                % normalem makes italics be italics, not underlines
    \usepackage{mathrsfs}
    

    
    % Colors for the hyperref package
    \definecolor{urlcolor}{rgb}{0,.145,.698}
    \definecolor{linkcolor}{rgb}{.71,0.21,0.01}
    \definecolor{citecolor}{rgb}{.12,.54,.11}

    % ANSI colors
    \definecolor{ansi-black}{HTML}{3E424D}
    \definecolor{ansi-black-intense}{HTML}{282C36}
    \definecolor{ansi-red}{HTML}{E75C58}
    \definecolor{ansi-red-intense}{HTML}{B22B31}
    \definecolor{ansi-green}{HTML}{00A250}
    \definecolor{ansi-green-intense}{HTML}{007427}
    \definecolor{ansi-yellow}{HTML}{DDB62B}
    \definecolor{ansi-yellow-intense}{HTML}{B27D12}
    \definecolor{ansi-blue}{HTML}{208FFB}
    \definecolor{ansi-blue-intense}{HTML}{0065CA}
    \definecolor{ansi-magenta}{HTML}{D160C4}
    \definecolor{ansi-magenta-intense}{HTML}{A03196}
    \definecolor{ansi-cyan}{HTML}{60C6C8}
    \definecolor{ansi-cyan-intense}{HTML}{258F8F}
    \definecolor{ansi-white}{HTML}{C5C1B4}
    \definecolor{ansi-white-intense}{HTML}{A1A6B2}
    \definecolor{ansi-default-inverse-fg}{HTML}{FFFFFF}
    \definecolor{ansi-default-inverse-bg}{HTML}{000000}

    % common color for the border for error outputs.
    \definecolor{outerrorbackground}{HTML}{FFDFDF}

    % commands and environments needed by pandoc snippets
    % extracted from the output of `pandoc -s`
    \providecommand{\tightlist}{%
      \setlength{\itemsep}{0pt}\setlength{\parskip}{0pt}}
    \DefineVerbatimEnvironment{Highlighting}{Verbatim}{commandchars=\\\{\}}
    % Add ',fontsize=\small' for more characters per line
    \newenvironment{Shaded}{}{}
    \newcommand{\KeywordTok}[1]{\textcolor[rgb]{0.00,0.44,0.13}{\textbf{{#1}}}}
    \newcommand{\DataTypeTok}[1]{\textcolor[rgb]{0.56,0.13,0.00}{{#1}}}
    \newcommand{\DecValTok}[1]{\textcolor[rgb]{0.25,0.63,0.44}{{#1}}}
    \newcommand{\BaseNTok}[1]{\textcolor[rgb]{0.25,0.63,0.44}{{#1}}}
    \newcommand{\FloatTok}[1]{\textcolor[rgb]{0.25,0.63,0.44}{{#1}}}
    \newcommand{\CharTok}[1]{\textcolor[rgb]{0.25,0.44,0.63}{{#1}}}
    \newcommand{\StringTok}[1]{\textcolor[rgb]{0.25,0.44,0.63}{{#1}}}
    \newcommand{\CommentTok}[1]{\textcolor[rgb]{0.38,0.63,0.69}{\textit{{#1}}}}
    \newcommand{\OtherTok}[1]{\textcolor[rgb]{0.00,0.44,0.13}{{#1}}}
    \newcommand{\AlertTok}[1]{\textcolor[rgb]{1.00,0.00,0.00}{\textbf{{#1}}}}
    \newcommand{\FunctionTok}[1]{\textcolor[rgb]{0.02,0.16,0.49}{{#1}}}
    \newcommand{\RegionMarkerTok}[1]{{#1}}
    \newcommand{\ErrorTok}[1]{\textcolor[rgb]{1.00,0.00,0.00}{\textbf{{#1}}}}
    \newcommand{\NormalTok}[1]{{#1}}
    
    % Additional commands for more recent versions of Pandoc
    \newcommand{\ConstantTok}[1]{\textcolor[rgb]{0.53,0.00,0.00}{{#1}}}
    \newcommand{\SpecialCharTok}[1]{\textcolor[rgb]{0.25,0.44,0.63}{{#1}}}
    \newcommand{\VerbatimStringTok}[1]{\textcolor[rgb]{0.25,0.44,0.63}{{#1}}}
    \newcommand{\SpecialStringTok}[1]{\textcolor[rgb]{0.73,0.40,0.53}{{#1}}}
    \newcommand{\ImportTok}[1]{{#1}}
    \newcommand{\DocumentationTok}[1]{\textcolor[rgb]{0.73,0.13,0.13}{\textit{{#1}}}}
    \newcommand{\AnnotationTok}[1]{\textcolor[rgb]{0.38,0.63,0.69}{\textbf{\textit{{#1}}}}}
    \newcommand{\CommentVarTok}[1]{\textcolor[rgb]{0.38,0.63,0.69}{\textbf{\textit{{#1}}}}}
    \newcommand{\VariableTok}[1]{\textcolor[rgb]{0.10,0.09,0.49}{{#1}}}
    \newcommand{\ControlFlowTok}[1]{\textcolor[rgb]{0.00,0.44,0.13}{\textbf{{#1}}}}
    \newcommand{\OperatorTok}[1]{\textcolor[rgb]{0.40,0.40,0.40}{{#1}}}
    \newcommand{\BuiltInTok}[1]{{#1}}
    \newcommand{\ExtensionTok}[1]{{#1}}
    \newcommand{\PreprocessorTok}[1]{\textcolor[rgb]{0.74,0.48,0.00}{{#1}}}
    \newcommand{\AttributeTok}[1]{\textcolor[rgb]{0.49,0.56,0.16}{{#1}}}
    \newcommand{\InformationTok}[1]{\textcolor[rgb]{0.38,0.63,0.69}{\textbf{\textit{{#1}}}}}
    \newcommand{\WarningTok}[1]{\textcolor[rgb]{0.38,0.63,0.69}{\textbf{\textit{{#1}}}}}
    
    
    % Define a nice break command that doesn't care if a line doesn't already
    % exist.
    \def\br{\hspace*{\fill} \\* }
    % Math Jax compatibility definitions
    \def\gt{>}
    \def\lt{<}
    \let\Oldtex\TeX
    \let\Oldlatex\LaTeX
    \renewcommand{\TeX}{\textrm{\Oldtex}}
    \renewcommand{\LaTeX}{\textrm{\Oldlatex}}
    % Document parameters
    % Document title
    \title{coursework}
    
    
    
    
    
% Pygments definitions
\makeatletter
\def\PY@reset{\let\PY@it=\relax \let\PY@bf=\relax%
    \let\PY@ul=\relax \let\PY@tc=\relax%
    \let\PY@bc=\relax \let\PY@ff=\relax}
\def\PY@tok#1{\csname PY@tok@#1\endcsname}
\def\PY@toks#1+{\ifx\relax#1\empty\else%
    \PY@tok{#1}\expandafter\PY@toks\fi}
\def\PY@do#1{\PY@bc{\PY@tc{\PY@ul{%
    \PY@it{\PY@bf{\PY@ff{#1}}}}}}}
\def\PY#1#2{\PY@reset\PY@toks#1+\relax+\PY@do{#2}}

\@namedef{PY@tok@w}{\def\PY@tc##1{\textcolor[rgb]{0.73,0.73,0.73}{##1}}}
\@namedef{PY@tok@c}{\let\PY@it=\textit\def\PY@tc##1{\textcolor[rgb]{0.25,0.50,0.50}{##1}}}
\@namedef{PY@tok@cp}{\def\PY@tc##1{\textcolor[rgb]{0.74,0.48,0.00}{##1}}}
\@namedef{PY@tok@k}{\let\PY@bf=\textbf\def\PY@tc##1{\textcolor[rgb]{0.00,0.50,0.00}{##1}}}
\@namedef{PY@tok@kp}{\def\PY@tc##1{\textcolor[rgb]{0.00,0.50,0.00}{##1}}}
\@namedef{PY@tok@kt}{\def\PY@tc##1{\textcolor[rgb]{0.69,0.00,0.25}{##1}}}
\@namedef{PY@tok@o}{\def\PY@tc##1{\textcolor[rgb]{0.40,0.40,0.40}{##1}}}
\@namedef{PY@tok@ow}{\let\PY@bf=\textbf\def\PY@tc##1{\textcolor[rgb]{0.67,0.13,1.00}{##1}}}
\@namedef{PY@tok@nb}{\def\PY@tc##1{\textcolor[rgb]{0.00,0.50,0.00}{##1}}}
\@namedef{PY@tok@nf}{\def\PY@tc##1{\textcolor[rgb]{0.00,0.00,1.00}{##1}}}
\@namedef{PY@tok@nc}{\let\PY@bf=\textbf\def\PY@tc##1{\textcolor[rgb]{0.00,0.00,1.00}{##1}}}
\@namedef{PY@tok@nn}{\let\PY@bf=\textbf\def\PY@tc##1{\textcolor[rgb]{0.00,0.00,1.00}{##1}}}
\@namedef{PY@tok@ne}{\let\PY@bf=\textbf\def\PY@tc##1{\textcolor[rgb]{0.82,0.25,0.23}{##1}}}
\@namedef{PY@tok@nv}{\def\PY@tc##1{\textcolor[rgb]{0.10,0.09,0.49}{##1}}}
\@namedef{PY@tok@no}{\def\PY@tc##1{\textcolor[rgb]{0.53,0.00,0.00}{##1}}}
\@namedef{PY@tok@nl}{\def\PY@tc##1{\textcolor[rgb]{0.63,0.63,0.00}{##1}}}
\@namedef{PY@tok@ni}{\let\PY@bf=\textbf\def\PY@tc##1{\textcolor[rgb]{0.60,0.60,0.60}{##1}}}
\@namedef{PY@tok@na}{\def\PY@tc##1{\textcolor[rgb]{0.49,0.56,0.16}{##1}}}
\@namedef{PY@tok@nt}{\let\PY@bf=\textbf\def\PY@tc##1{\textcolor[rgb]{0.00,0.50,0.00}{##1}}}
\@namedef{PY@tok@nd}{\def\PY@tc##1{\textcolor[rgb]{0.67,0.13,1.00}{##1}}}
\@namedef{PY@tok@s}{\def\PY@tc##1{\textcolor[rgb]{0.73,0.13,0.13}{##1}}}
\@namedef{PY@tok@sd}{\let\PY@it=\textit\def\PY@tc##1{\textcolor[rgb]{0.73,0.13,0.13}{##1}}}
\@namedef{PY@tok@si}{\let\PY@bf=\textbf\def\PY@tc##1{\textcolor[rgb]{0.73,0.40,0.53}{##1}}}
\@namedef{PY@tok@se}{\let\PY@bf=\textbf\def\PY@tc##1{\textcolor[rgb]{0.73,0.40,0.13}{##1}}}
\@namedef{PY@tok@sr}{\def\PY@tc##1{\textcolor[rgb]{0.73,0.40,0.53}{##1}}}
\@namedef{PY@tok@ss}{\def\PY@tc##1{\textcolor[rgb]{0.10,0.09,0.49}{##1}}}
\@namedef{PY@tok@sx}{\def\PY@tc##1{\textcolor[rgb]{0.00,0.50,0.00}{##1}}}
\@namedef{PY@tok@m}{\def\PY@tc##1{\textcolor[rgb]{0.40,0.40,0.40}{##1}}}
\@namedef{PY@tok@gh}{\let\PY@bf=\textbf\def\PY@tc##1{\textcolor[rgb]{0.00,0.00,0.50}{##1}}}
\@namedef{PY@tok@gu}{\let\PY@bf=\textbf\def\PY@tc##1{\textcolor[rgb]{0.50,0.00,0.50}{##1}}}
\@namedef{PY@tok@gd}{\def\PY@tc##1{\textcolor[rgb]{0.63,0.00,0.00}{##1}}}
\@namedef{PY@tok@gi}{\def\PY@tc##1{\textcolor[rgb]{0.00,0.63,0.00}{##1}}}
\@namedef{PY@tok@gr}{\def\PY@tc##1{\textcolor[rgb]{1.00,0.00,0.00}{##1}}}
\@namedef{PY@tok@ge}{\let\PY@it=\textit}
\@namedef{PY@tok@gs}{\let\PY@bf=\textbf}
\@namedef{PY@tok@gp}{\let\PY@bf=\textbf\def\PY@tc##1{\textcolor[rgb]{0.00,0.00,0.50}{##1}}}
\@namedef{PY@tok@go}{\def\PY@tc##1{\textcolor[rgb]{0.53,0.53,0.53}{##1}}}
\@namedef{PY@tok@gt}{\def\PY@tc##1{\textcolor[rgb]{0.00,0.27,0.87}{##1}}}
\@namedef{PY@tok@err}{\def\PY@bc##1{{\setlength{\fboxsep}{\string -\fboxrule}\fcolorbox[rgb]{1.00,0.00,0.00}{1,1,1}{\strut ##1}}}}
\@namedef{PY@tok@kc}{\let\PY@bf=\textbf\def\PY@tc##1{\textcolor[rgb]{0.00,0.50,0.00}{##1}}}
\@namedef{PY@tok@kd}{\let\PY@bf=\textbf\def\PY@tc##1{\textcolor[rgb]{0.00,0.50,0.00}{##1}}}
\@namedef{PY@tok@kn}{\let\PY@bf=\textbf\def\PY@tc##1{\textcolor[rgb]{0.00,0.50,0.00}{##1}}}
\@namedef{PY@tok@kr}{\let\PY@bf=\textbf\def\PY@tc##1{\textcolor[rgb]{0.00,0.50,0.00}{##1}}}
\@namedef{PY@tok@bp}{\def\PY@tc##1{\textcolor[rgb]{0.00,0.50,0.00}{##1}}}
\@namedef{PY@tok@fm}{\def\PY@tc##1{\textcolor[rgb]{0.00,0.00,1.00}{##1}}}
\@namedef{PY@tok@vc}{\def\PY@tc##1{\textcolor[rgb]{0.10,0.09,0.49}{##1}}}
\@namedef{PY@tok@vg}{\def\PY@tc##1{\textcolor[rgb]{0.10,0.09,0.49}{##1}}}
\@namedef{PY@tok@vi}{\def\PY@tc##1{\textcolor[rgb]{0.10,0.09,0.49}{##1}}}
\@namedef{PY@tok@vm}{\def\PY@tc##1{\textcolor[rgb]{0.10,0.09,0.49}{##1}}}
\@namedef{PY@tok@sa}{\def\PY@tc##1{\textcolor[rgb]{0.73,0.13,0.13}{##1}}}
\@namedef{PY@tok@sb}{\def\PY@tc##1{\textcolor[rgb]{0.73,0.13,0.13}{##1}}}
\@namedef{PY@tok@sc}{\def\PY@tc##1{\textcolor[rgb]{0.73,0.13,0.13}{##1}}}
\@namedef{PY@tok@dl}{\def\PY@tc##1{\textcolor[rgb]{0.73,0.13,0.13}{##1}}}
\@namedef{PY@tok@s2}{\def\PY@tc##1{\textcolor[rgb]{0.73,0.13,0.13}{##1}}}
\@namedef{PY@tok@sh}{\def\PY@tc##1{\textcolor[rgb]{0.73,0.13,0.13}{##1}}}
\@namedef{PY@tok@s1}{\def\PY@tc##1{\textcolor[rgb]{0.73,0.13,0.13}{##1}}}
\@namedef{PY@tok@mb}{\def\PY@tc##1{\textcolor[rgb]{0.40,0.40,0.40}{##1}}}
\@namedef{PY@tok@mf}{\def\PY@tc##1{\textcolor[rgb]{0.40,0.40,0.40}{##1}}}
\@namedef{PY@tok@mh}{\def\PY@tc##1{\textcolor[rgb]{0.40,0.40,0.40}{##1}}}
\@namedef{PY@tok@mi}{\def\PY@tc##1{\textcolor[rgb]{0.40,0.40,0.40}{##1}}}
\@namedef{PY@tok@il}{\def\PY@tc##1{\textcolor[rgb]{0.40,0.40,0.40}{##1}}}
\@namedef{PY@tok@mo}{\def\PY@tc##1{\textcolor[rgb]{0.40,0.40,0.40}{##1}}}
\@namedef{PY@tok@ch}{\let\PY@it=\textit\def\PY@tc##1{\textcolor[rgb]{0.25,0.50,0.50}{##1}}}
\@namedef{PY@tok@cm}{\let\PY@it=\textit\def\PY@tc##1{\textcolor[rgb]{0.25,0.50,0.50}{##1}}}
\@namedef{PY@tok@cpf}{\let\PY@it=\textit\def\PY@tc##1{\textcolor[rgb]{0.25,0.50,0.50}{##1}}}
\@namedef{PY@tok@c1}{\let\PY@it=\textit\def\PY@tc##1{\textcolor[rgb]{0.25,0.50,0.50}{##1}}}
\@namedef{PY@tok@cs}{\let\PY@it=\textit\def\PY@tc##1{\textcolor[rgb]{0.25,0.50,0.50}{##1}}}

\def\PYZbs{\char`\\}
\def\PYZus{\char`\_}
\def\PYZob{\char`\{}
\def\PYZcb{\char`\}}
\def\PYZca{\char`\^}
\def\PYZam{\char`\&}
\def\PYZlt{\char`\<}
\def\PYZgt{\char`\>}
\def\PYZsh{\char`\#}
\def\PYZpc{\char`\%}
\def\PYZdl{\char`\$}
\def\PYZhy{\char`\-}
\def\PYZsq{\char`\'}
\def\PYZdq{\char`\"}
\def\PYZti{\char`\~}
% for compatibility with earlier versions
\def\PYZat{@}
\def\PYZlb{[}
\def\PYZrb{]}
\makeatother


    % For linebreaks inside Verbatim environment from package fancyvrb. 
    \makeatletter
        \newbox\Wrappedcontinuationbox 
        \newbox\Wrappedvisiblespacebox 
        \newcommand*\Wrappedvisiblespace {\textcolor{red}{\textvisiblespace}} 
        \newcommand*\Wrappedcontinuationsymbol {\textcolor{red}{\llap{\tiny$\m@th\hookrightarrow$}}} 
        \newcommand*\Wrappedcontinuationindent {3ex } 
        \newcommand*\Wrappedafterbreak {\kern\Wrappedcontinuationindent\copy\Wrappedcontinuationbox} 
        % Take advantage of the already applied Pygments mark-up to insert 
        % potential linebreaks for TeX processing. 
        %        {, <, #, %, $, ' and ": go to next line. 
        %        _, }, ^, &, >, - and ~: stay at end of broken line. 
        % Use of \textquotesingle for straight quote. 
        \newcommand*\Wrappedbreaksatspecials {% 
            \def\PYGZus{\discretionary{\char`\_}{\Wrappedafterbreak}{\char`\_}}% 
            \def\PYGZob{\discretionary{}{\Wrappedafterbreak\char`\{}{\char`\{}}% 
            \def\PYGZcb{\discretionary{\char`\}}{\Wrappedafterbreak}{\char`\}}}% 
            \def\PYGZca{\discretionary{\char`\^}{\Wrappedafterbreak}{\char`\^}}% 
            \def\PYGZam{\discretionary{\char`\&}{\Wrappedafterbreak}{\char`\&}}% 
            \def\PYGZlt{\discretionary{}{\Wrappedafterbreak\char`\<}{\char`\<}}% 
            \def\PYGZgt{\discretionary{\char`\>}{\Wrappedafterbreak}{\char`\>}}% 
            \def\PYGZsh{\discretionary{}{\Wrappedafterbreak\char`\#}{\char`\#}}% 
            \def\PYGZpc{\discretionary{}{\Wrappedafterbreak\char`\%}{\char`\%}}% 
            \def\PYGZdl{\discretionary{}{\Wrappedafterbreak\char`\$}{\char`\$}}% 
            \def\PYGZhy{\discretionary{\char`\-}{\Wrappedafterbreak}{\char`\-}}% 
            \def\PYGZsq{\discretionary{}{\Wrappedafterbreak\textquotesingle}{\textquotesingle}}% 
            \def\PYGZdq{\discretionary{}{\Wrappedafterbreak\char`\"}{\char`\"}}% 
            \def\PYGZti{\discretionary{\char`\~}{\Wrappedafterbreak}{\char`\~}}% 
        } 
        % Some characters . , ; ? ! / are not pygmentized. 
        % This macro makes them "active" and they will insert potential linebreaks 
        \newcommand*\Wrappedbreaksatpunct {% 
            \lccode`\~`\.\lowercase{\def~}{\discretionary{\hbox{\char`\.}}{\Wrappedafterbreak}{\hbox{\char`\.}}}% 
            \lccode`\~`\,\lowercase{\def~}{\discretionary{\hbox{\char`\,}}{\Wrappedafterbreak}{\hbox{\char`\,}}}% 
            \lccode`\~`\;\lowercase{\def~}{\discretionary{\hbox{\char`\;}}{\Wrappedafterbreak}{\hbox{\char`\;}}}% 
            \lccode`\~`\:\lowercase{\def~}{\discretionary{\hbox{\char`\:}}{\Wrappedafterbreak}{\hbox{\char`\:}}}% 
            \lccode`\~`\?\lowercase{\def~}{\discretionary{\hbox{\char`\?}}{\Wrappedafterbreak}{\hbox{\char`\?}}}% 
            \lccode`\~`\!\lowercase{\def~}{\discretionary{\hbox{\char`\!}}{\Wrappedafterbreak}{\hbox{\char`\!}}}% 
            \lccode`\~`\/\lowercase{\def~}{\discretionary{\hbox{\char`\/}}{\Wrappedafterbreak}{\hbox{\char`\/}}}% 
            \catcode`\.\active
            \catcode`\,\active 
            \catcode`\;\active
            \catcode`\:\active
            \catcode`\?\active
            \catcode`\!\active
            \catcode`\/\active 
            \lccode`\~`\~ 	
        }
    \makeatother

    \let\OriginalVerbatim=\Verbatim
    \makeatletter
    \renewcommand{\Verbatim}[1][1]{%
        %\parskip\z@skip
        \sbox\Wrappedcontinuationbox {\Wrappedcontinuationsymbol}%
        \sbox\Wrappedvisiblespacebox {\FV@SetupFont\Wrappedvisiblespace}%
        \def\FancyVerbFormatLine ##1{\hsize\linewidth
            \vtop{\raggedright\hyphenpenalty\z@\exhyphenpenalty\z@
                \doublehyphendemerits\z@\finalhyphendemerits\z@
                \strut ##1\strut}%
        }%
        % If the linebreak is at a space, the latter will be displayed as visible
        % space at end of first line, and a continuation symbol starts next line.
        % Stretch/shrink are however usually zero for typewriter font.
        \def\FV@Space {%
            \nobreak\hskip\z@ plus\fontdimen3\font minus\fontdimen4\font
            \discretionary{\copy\Wrappedvisiblespacebox}{\Wrappedafterbreak}
            {\kern\fontdimen2\font}%
        }%
        
        % Allow breaks at special characters using \PYG... macros.
        \Wrappedbreaksatspecials
        % Breaks at punctuation characters . , ; ? ! and / need catcode=\active 	
        \OriginalVerbatim[#1,codes*=\Wrappedbreaksatpunct]%
    }
    \makeatother

    % Exact colors from NB
    \definecolor{incolor}{HTML}{303F9F}
    \definecolor{outcolor}{HTML}{D84315}
    \definecolor{cellborder}{HTML}{CFCFCF}
    \definecolor{cellbackground}{HTML}{F7F7F7}
    
    % prompt
    \makeatletter
    \newcommand{\boxspacing}{\kern\kvtcb@left@rule\kern\kvtcb@boxsep}
    \makeatother
    \newcommand{\prompt}[4]{
        {\ttfamily\llap{{\color{#2}[#3]:\hspace{3pt}#4}}\vspace{-\baselineskip}}
    }
    
    % Prevent overflowing lines due to hard-to-break entities
    \sloppy 
    % Slightly bigger margins than the latex defaults
    
    \geometry{verbose,tmargin=1in,bmargin=1in,lmargin=1in,rmargin=1in}
    
    
\hypersetup{colorlinks=true, linkcolor=black}
\begin{document}
	\tableofcontents   
    % Setup hyperref package
    \hypersetup{
      breaklinks=true,  % so long urls are correctly broken across lines
      colorlinks=true,
      urlcolor=urlcolor,
      linkcolor=linkcolor,
      citecolor=citecolor,
      }
		\pagebreak
	  \chapter{Введение}
	  \qquad Курсовой проект – самостоятельная часть учебной дисциплины «Технологии машинного обучения» – учебная и практическая исследовательская студенческая работа, направленная на решение комплексной задачи машинного обучения. Результатом курсового проекта является отчет, содержащий описания моделей, тексты программ и результаты экспериментов. \\
	\qquad Курсовой проект опирается на знания, умения и владения, полученные студентом в рамках лекций и лабораторных работ по дисциплине. \\
	\qquad В рамках курсового проекта я должен провести решение задачи машинного обучения на основе материалов дисциплины.

\chapter{Поиск и выбор набора данных для построения моделей машинного
обучения. На основе выбранного набора данных студент должен построить
модели машинного обучения для решения задачи
регрессии.}

    В качестве набора данных мы будем использовать набор данных по продажам
домов в США за Май 2014 - Май 2015 -
https://www.kaggle.com/harlfoxem/housesalesprediction?select=kc\_house\_data.csv

Датасет состоит из 1 файла: kc\_house\_data.csv

Файл описывает следующие колонки:

id - уникальный id продажи дома\\
date - дата продажи дома\\
price - цена продажи дома\\
bedrooms - кол-во спальных комнат\\
bathrooms - кол-во ванных комнат (где .5 считается за комнату с туалетом
без душа)\\
sqft\_living - жилая площадь(кв. футах)\\
sqft\_lot - площать земли\\
floors - кол-во этажей\\
waterfront - рядом с водой (1 или 0)\\
view - хороший вид (от 0 или 4)\\
condition - условия апартаметов (от 1 до 5)\\
grade - оценка (от 1 до 13)\\
sqft\_above - площадь пространства, выше земли\\
sqft\_basement - площадь пространства, ниже земли(подвал)\\
yr\_built - год постройки\\
yr\_renovated - год реновации\\
zipcode - почтовый код\\
lat - долгота (координаты)\\
long - широта (координаты)\\
sqft\_living15 - площадь внутренней жилой площади для ближайших 15
соседей\\
sqft\_lot15 - площадь земли для ближайших 15 соседей

В рассматриваемом примере будем решать задачу регрессии - целевой
признак `price' - цена

\section{Импорт библиотек}

    \begin{tcolorbox}[breakable, size=fbox, boxrule=1pt, pad at break*=1mm,colback=cellbackground, colframe=cellborder]
\prompt{In}{incolor}{1}{\boxspacing}
\begin{Verbatim}[commandchars=\\\{\}]
\PY{k+kn}{import} \PY{n+nn}{numpy} \PY{k}{as} \PY{n+nn}{np}
\PY{k+kn}{import} \PY{n+nn}{pandas} \PY{k}{as} \PY{n+nn}{pd}
\PY{k+kn}{import} \PY{n+nn}{seaborn} \PY{k}{as} \PY{n+nn}{sns}
\PY{k+kn}{import} \PY{n+nn}{matplotlib}\PY{n+nn}{.}\PY{n+nn}{pyplot} \PY{k}{as} \PY{n+nn}{plt}
\PY{k+kn}{from} \PY{n+nn}{typing} \PY{k+kn}{import} \PY{n}{Dict}\PY{p}{,} \PY{n}{Tuple}
\PY{k+kn}{from} \PY{n+nn}{IPython}\PY{n+nn}{.}\PY{n+nn}{display} \PY{k+kn}{import} \PY{n}{Image}

\PY{k+kn}{from} \PY{n+nn}{sklearn}\PY{n+nn}{.}\PY{n+nn}{preprocessing} \PY{k+kn}{import} \PY{n}{MinMaxScaler}
\PY{k+kn}{from} \PY{n+nn}{sklearn}\PY{n+nn}{.}\PY{n+nn}{linear\PYZus{}model} \PY{k+kn}{import} \PY{n}{LinearRegression}\PY{p}{,} \PY{n}{LogisticRegression}
\PY{k+kn}{from} \PY{n+nn}{sklearn}\PY{n+nn}{.}\PY{n+nn}{model\PYZus{}selection} \PY{k+kn}{import} \PY{n}{train\PYZus{}test\PYZus{}split}
\PY{k+kn}{from} \PY{n+nn}{sklearn}\PY{n+nn}{.}\PY{n+nn}{neighbors} \PY{k+kn}{import} \PY{n}{KNeighborsRegressor}\PY{p}{,} \PY{n}{KNeighborsClassifier}
\PY{k+kn}{from} \PY{n+nn}{sklearn}\PY{n+nn}{.}\PY{n+nn}{metrics} \PY{k+kn}{import} \PY{n}{accuracy\PYZus{}score}\PY{p}{,} \PY{n}{balanced\PYZus{}accuracy\PYZus{}score}
\PY{k+kn}{from} \PY{n+nn}{sklearn}\PY{n+nn}{.}\PY{n+nn}{metrics} \PY{k+kn}{import} \PY{n}{precision\PYZus{}score}\PY{p}{,} \PY{n}{recall\PYZus{}score}\PY{p}{,} \PY{n}{f1\PYZus{}score}\PY{p}{,} \PY{n}{classification\PYZus{}report}
\PY{k+kn}{from} \PY{n+nn}{sklearn}\PY{n+nn}{.}\PY{n+nn}{metrics} \PY{k+kn}{import} \PY{n}{confusion\PYZus{}matrix}
\PY{k+kn}{from} \PY{n+nn}{sklearn}\PY{n+nn}{.}\PY{n+nn}{metrics} \PY{k+kn}{import} \PY{n}{plot\PYZus{}confusion\PYZus{}matrix}
\PY{k+kn}{from} \PY{n+nn}{sklearn}\PY{n+nn}{.}\PY{n+nn}{model\PYZus{}selection} \PY{k+kn}{import} \PY{n}{GridSearchCV}
\PY{k+kn}{from} \PY{n+nn}{sklearn}\PY{n+nn}{.}\PY{n+nn}{metrics} \PY{k+kn}{import} \PY{n}{mean\PYZus{}absolute\PYZus{}error}\PY{p}{,} \PY{n}{mean\PYZus{}squared\PYZus{}error}\PY{p}{,} \PY{n}{mean\PYZus{}squared\PYZus{}log\PYZus{}error}\PY{p}{,} \PY{n}{median\PYZus{}absolute\PYZus{}error}\PY{p}{,} \PY{n}{r2\PYZus{}score} 
\PY{k+kn}{from} \PY{n+nn}{sklearn}\PY{n+nn}{.}\PY{n+nn}{metrics} \PY{k+kn}{import} \PY{n}{roc\PYZus{}curve}\PY{p}{,} \PY{n}{roc\PYZus{}auc\PYZus{}score}
\PY{k+kn}{from} \PY{n+nn}{sklearn}\PY{n+nn}{.}\PY{n+nn}{svm} \PY{k+kn}{import} \PY{n}{SVC}\PY{p}{,} \PY{n}{NuSVC}\PY{p}{,} \PY{n}{LinearSVC}\PY{p}{,} \PY{n}{OneClassSVM}\PY{p}{,} \PY{n}{SVR}\PY{p}{,} \PY{n}{NuSVR}\PY{p}{,} \PY{n}{LinearSVR}
\PY{k+kn}{from} \PY{n+nn}{sklearn}\PY{n+nn}{.}\PY{n+nn}{tree} \PY{k+kn}{import} \PY{n}{DecisionTreeClassifier}\PY{p}{,} \PY{n}{DecisionTreeRegressor}\PY{p}{,} \PY{n}{export\PYZus{}graphviz}
\PY{k+kn}{from} \PY{n+nn}{sklearn}\PY{n+nn}{.}\PY{n+nn}{ensemble} \PY{k+kn}{import} \PY{n}{RandomForestClassifier}\PY{p}{,} \PY{n}{RandomForestRegressor}
\PY{k+kn}{from} \PY{n+nn}{sklearn}\PY{n+nn}{.}\PY{n+nn}{ensemble} \PY{k+kn}{import} \PY{n}{ExtraTreesClassifier}\PY{p}{,} \PY{n}{ExtraTreesRegressor}
\PY{k+kn}{from} \PY{n+nn}{sklearn}\PY{n+nn}{.}\PY{n+nn}{ensemble} \PY{k+kn}{import} \PY{n}{GradientBoostingClassifier}\PY{p}{,} \PY{n}{GradientBoostingRegressor}
\end{Verbatim}
\end{tcolorbox}

\section{Загрузка данных}

    Загрузим файлы датасета в помощью библиотеки Pandas.

Файл представляет собой данные в формате CSV. Разделитель - `,'

    \begin{tcolorbox}[breakable, size=fbox, boxrule=1pt, pad at break*=1mm,colback=cellbackground, colframe=cellborder]
\prompt{In}{incolor}{2}{\boxspacing}
\begin{Verbatim}[commandchars=\\\{\}]
\PY{n}{data} \PY{o}{=} \PY{n}{pd}\PY{o}{.}\PY{n}{read\PYZus{}csv}\PY{p}{(}\PY{l+s+s2}{\PYZdq{}}\PY{l+s+s2}{data/kc\PYZus{}house\PYZus{}data.csv}\PY{l+s+s2}{\PYZdq{}}\PY{p}{,} \PY{n}{sep}\PY{o}{=}\PY{l+s+s1}{\PYZsq{}}\PY{l+s+s1}{,}\PY{l+s+s1}{\PYZsq{}}\PY{p}{)}
\end{Verbatim}
\end{tcolorbox}

\chapter{Проведение разведочного анализа данных. Построение графиков,
необходимых для понимания структуры данных. Анализ и заполнение
пропусков в
данных.}

    \begin{tcolorbox}[breakable, size=fbox, boxrule=1pt, pad at break*=1mm,colback=cellbackground, colframe=cellborder]
\prompt{In}{incolor}{3}{\boxspacing}
\begin{Verbatim}[commandchars=\\\{\}]
\PY{n}{data}\PY{o}{.}\PY{n}{head}\PY{p}{(}\PY{p}{)}
\end{Verbatim}
\end{tcolorbox}

            \begin{tcolorbox}[breakable, size=fbox, boxrule=.5pt, pad at break*=1mm, opacityfill=0]
\prompt{Out}{outcolor}{3}{\boxspacing}
\begin{Verbatim}[commandchars=\\\{\}]
           id             date     price  bedrooms  bathrooms  sqft\_living  \textbackslash{}
0  7129300520  20141013T000000  221900.0         3       1.00         1180
1  6414100192  20141209T000000  538000.0         3       2.25         2570
2  5631500400  20150225T000000  180000.0         2       1.00          770
3  2487200875  20141209T000000  604000.0         4       3.00         1960
4  1954400510  20150218T000000  510000.0         3       2.00         1680

   sqft\_lot  floors  waterfront  view  {\ldots}  grade  sqft\_above  sqft\_basement  \textbackslash{}
0      5650     1.0           0     0  {\ldots}      7        1180              0
1      7242     2.0           0     0  {\ldots}      7        2170            400
2     10000     1.0           0     0  {\ldots}      6         770              0
3      5000     1.0           0     0  {\ldots}      7        1050            910
4      8080     1.0           0     0  {\ldots}      8        1680              0

   yr\_built  yr\_renovated  zipcode      lat     long  sqft\_living15  \textbackslash{}
0      1955             0    98178  47.5112 -122.257           1340
1      1951          1991    98125  47.7210 -122.319           1690
2      1933             0    98028  47.7379 -122.233           2720
3      1965             0    98136  47.5208 -122.393           1360
4      1987             0    98074  47.6168 -122.045           1800

   sqft\_lot15
0        5650
1        7639
2        8062
3        5000
4        7503

[5 rows x 21 columns]
\end{Verbatim}
\end{tcolorbox}
        
    \begin{tcolorbox}[breakable, size=fbox, boxrule=1pt, pad at break*=1mm,colback=cellbackground, colframe=cellborder]
\prompt{In}{incolor}{4}{\boxspacing}
\begin{Verbatim}[commandchars=\\\{\}]
\PY{n}{data}\PY{o}{.}\PY{n}{shape}
\end{Verbatim}
\end{tcolorbox}

            \begin{tcolorbox}[breakable, size=fbox, boxrule=.5pt, pad at break*=1mm, opacityfill=0]
\prompt{Out}{outcolor}{4}{\boxspacing}
\begin{Verbatim}[commandchars=\\\{\}]
(21613, 21)
\end{Verbatim}
\end{tcolorbox}
        
    \begin{tcolorbox}[breakable, size=fbox, boxrule=1pt, pad at break*=1mm,colback=cellbackground, colframe=cellborder]
\prompt{In}{incolor}{5}{\boxspacing}
\begin{Verbatim}[commandchars=\\\{\}]
\PY{n}{data}\PY{o}{.}\PY{n}{dtypes}
\end{Verbatim}
\end{tcolorbox}

            \begin{tcolorbox}[breakable, size=fbox, boxrule=.5pt, pad at break*=1mm, opacityfill=0]
\prompt{Out}{outcolor}{5}{\boxspacing}
\begin{Verbatim}[commandchars=\\\{\}]
id                 int64
date              object
price            float64
bedrooms           int64
bathrooms        float64
sqft\_living        int64
sqft\_lot           int64
floors           float64
waterfront         int64
view               int64
condition          int64
grade              int64
sqft\_above         int64
sqft\_basement      int64
yr\_built           int64
yr\_renovated       int64
zipcode            int64
lat              float64
long             float64
sqft\_living15      int64
sqft\_lot15         int64
dtype: object
\end{Verbatim}
\end{tcolorbox}
        
    \begin{tcolorbox}[breakable, size=fbox, boxrule=1pt, pad at break*=1mm,colback=cellbackground, colframe=cellborder]
\prompt{In}{incolor}{6}{\boxspacing}
\begin{Verbatim}[commandchars=\\\{\}]
\PY{n}{data}\PY{o}{.}\PY{n}{isnull}\PY{p}{(}\PY{p}{)}\PY{o}{.}\PY{n}{sum}\PY{p}{(}\PY{p}{)}
\end{Verbatim}
\end{tcolorbox}

            \begin{tcolorbox}[breakable, size=fbox, boxrule=.5pt, pad at break*=1mm, opacityfill=0]
\prompt{Out}{outcolor}{6}{\boxspacing}
\begin{Verbatim}[commandchars=\\\{\}]
id               0
date             0
price            0
bedrooms         0
bathrooms        0
sqft\_living      0
sqft\_lot         0
floors           0
waterfront       0
view             0
condition        0
grade            0
sqft\_above       0
sqft\_basement    0
yr\_built         0
yr\_renovated     0
zipcode          0
lat              0
long             0
sqft\_living15    0
sqft\_lot15       0
dtype: int64
\end{Verbatim}
\end{tcolorbox}
        
    \textbf{Вывод: исходный набор данных данных не содержит пропусков}

\section{Построение графиков для понимания структуры данных}

    Сразу удалим ненужные столбцы - уникальный id и уникальную дату.

    \begin{tcolorbox}[breakable, size=fbox, boxrule=1pt, pad at break*=1mm,colback=cellbackground, colframe=cellborder]
\prompt{In}{incolor}{7}{\boxspacing}
\begin{Verbatim}[commandchars=\\\{\}]
\PY{n}{sns}\PY{o}{.}\PY{n}{pairplot}\PY{p}{(}\PY{n}{data}\PY{p}{)}
\end{Verbatim}
\end{tcolorbox}

            \begin{tcolorbox}[breakable, size=fbox, boxrule=.5pt, pad at break*=1mm, opacityfill=0]
\prompt{Out}{outcolor}{7}{\boxspacing}
\begin{Verbatim}[commandchars=\\\{\}]
<seaborn.axisgrid.PairGrid at 0x7efce06511f0>
\end{Verbatim}
\end{tcolorbox}
        
    \begin{center}
    \adjustimage{max size={0.9\linewidth}{0.9\paperheight}}{output_15_1.png}
    \end{center}
    { \hspace*{\fill} \\}
    
    \begin{tcolorbox}[breakable, size=fbox, boxrule=1pt, pad at break*=1mm,colback=cellbackground, colframe=cellborder]
\prompt{In}{incolor}{8}{\boxspacing}
\begin{Verbatim}[commandchars=\\\{\}]
\PY{n}{data}\PY{o}{.}\PY{n}{columns}
\end{Verbatim}
\end{tcolorbox}

            \begin{tcolorbox}[breakable, size=fbox, boxrule=.5pt, pad at break*=1mm, opacityfill=0]
\prompt{Out}{outcolor}{8}{\boxspacing}
\begin{Verbatim}[commandchars=\\\{\}]
Index(['id', 'date', 'price', 'bedrooms', 'bathrooms', 'sqft\_living',
       'sqft\_lot', 'floors', 'waterfront', 'view', 'condition', 'grade',
       'sqft\_above', 'sqft\_basement', 'yr\_built', 'yr\_renovated', 'zipcode',
       'lat', 'long', 'sqft\_living15', 'sqft\_lot15'],
      dtype='object')
\end{Verbatim}
\end{tcolorbox}
        
    \begin{tcolorbox}[breakable, size=fbox, boxrule=1pt, pad at break*=1mm,colback=cellbackground, colframe=cellborder]
\prompt{In}{incolor}{9}{\boxspacing}
\begin{Verbatim}[commandchars=\\\{\}]
\PY{c+c1}{\PYZsh{} Скрипичные диаграммы для числовых колонок}
\PY{k}{for} \PY{n}{col} \PY{o+ow}{in} \PY{p}{[}\PY{l+s+s1}{\PYZsq{}}\PY{l+s+s1}{price}\PY{l+s+s1}{\PYZsq{}}\PY{p}{,} \PY{l+s+s1}{\PYZsq{}}\PY{l+s+s1}{bedrooms}\PY{l+s+s1}{\PYZsq{}}\PY{p}{,} \PY{l+s+s1}{\PYZsq{}}\PY{l+s+s1}{bathrooms}\PY{l+s+s1}{\PYZsq{}}\PY{p}{,} \PY{l+s+s1}{\PYZsq{}}\PY{l+s+s1}{sqft\PYZus{}living}\PY{l+s+s1}{\PYZsq{}}\PY{p}{,}
       \PY{l+s+s1}{\PYZsq{}}\PY{l+s+s1}{sqft\PYZus{}lot}\PY{l+s+s1}{\PYZsq{}}\PY{p}{,} \PY{l+s+s1}{\PYZsq{}}\PY{l+s+s1}{floors}\PY{l+s+s1}{\PYZsq{}}\PY{p}{,} \PY{l+s+s1}{\PYZsq{}}\PY{l+s+s1}{waterfront}\PY{l+s+s1}{\PYZsq{}}\PY{p}{,} \PY{l+s+s1}{\PYZsq{}}\PY{l+s+s1}{view}\PY{l+s+s1}{\PYZsq{}}\PY{p}{,} \PY{l+s+s1}{\PYZsq{}}\PY{l+s+s1}{condition}\PY{l+s+s1}{\PYZsq{}}\PY{p}{,} \PY{l+s+s1}{\PYZsq{}}\PY{l+s+s1}{grade}\PY{l+s+s1}{\PYZsq{}}\PY{p}{,}
       \PY{l+s+s1}{\PYZsq{}}\PY{l+s+s1}{sqft\PYZus{}above}\PY{l+s+s1}{\PYZsq{}}\PY{p}{,} \PY{l+s+s1}{\PYZsq{}}\PY{l+s+s1}{sqft\PYZus{}basement}\PY{l+s+s1}{\PYZsq{}}\PY{p}{,} \PY{l+s+s1}{\PYZsq{}}\PY{l+s+s1}{yr\PYZus{}built}\PY{l+s+s1}{\PYZsq{}}\PY{p}{,} \PY{l+s+s1}{\PYZsq{}}\PY{l+s+s1}{yr\PYZus{}renovated}\PY{l+s+s1}{\PYZsq{}}\PY{p}{,} \PY{l+s+s1}{\PYZsq{}}\PY{l+s+s1}{zipcode}\PY{l+s+s1}{\PYZsq{}}\PY{p}{,}
       \PY{l+s+s1}{\PYZsq{}}\PY{l+s+s1}{lat}\PY{l+s+s1}{\PYZsq{}}\PY{p}{,} \PY{l+s+s1}{\PYZsq{}}\PY{l+s+s1}{long}\PY{l+s+s1}{\PYZsq{}}\PY{p}{,} \PY{l+s+s1}{\PYZsq{}}\PY{l+s+s1}{sqft\PYZus{}living15}\PY{l+s+s1}{\PYZsq{}}\PY{p}{,} \PY{l+s+s1}{\PYZsq{}}\PY{l+s+s1}{sqft\PYZus{}lot15}\PY{l+s+s1}{\PYZsq{}}\PY{p}{]}\PY{p}{:}
    \PY{n}{sns}\PY{o}{.}\PY{n}{violinplot}\PY{p}{(}\PY{n}{x}\PY{o}{=}\PY{n}{data}\PY{p}{[}\PY{n}{col}\PY{p}{]}\PY{p}{)}
    \PY{n}{plt}\PY{o}{.}\PY{n}{show}\PY{p}{(}\PY{p}{)}
\end{Verbatim}
\end{tcolorbox}

    \begin{center}
    \adjustimage{max size={0.9\linewidth}{0.9\paperheight}}{output_17_0.png}
    \end{center}
    { \hspace*{\fill} \\}
    
    \begin{center}
    \adjustimage{max size={0.9\linewidth}{0.9\paperheight}}{output_17_1.png}
    \end{center}
    { \hspace*{\fill} \\}
    
    \begin{center}
    \adjustimage{max size={0.9\linewidth}{0.9\paperheight}}{output_17_2.png}
    \end{center}
    { \hspace*{\fill} \\}
    
    \begin{center}
    \adjustimage{max size={0.9\linewidth}{0.9\paperheight}}{output_17_3.png}
    \end{center}
    { \hspace*{\fill} \\}
    
    \begin{center}
    \adjustimage{max size={0.9\linewidth}{0.9\paperheight}}{output_17_4.png}
    \end{center}
    { \hspace*{\fill} \\}
    
    \begin{center}
    \adjustimage{max size={0.9\linewidth}{0.9\paperheight}}{output_17_5.png}
    \end{center}
    { \hspace*{\fill} \\}
    
    \begin{center}
    \adjustimage{max size={0.9\linewidth}{0.9\paperheight}}{output_17_6.png}
    \end{center}
    { \hspace*{\fill} \\}
    
    \begin{center}
    \adjustimage{max size={0.9\linewidth}{0.9\paperheight}}{output_17_7.png}
    \end{center}
    { \hspace*{\fill} \\}
    
    \begin{center}
    \adjustimage{max size={0.9\linewidth}{0.9\paperheight}}{output_17_8.png}
    \end{center}
    { \hspace*{\fill} \\}
    
    \begin{center}
    \adjustimage{max size={0.9\linewidth}{0.9\paperheight}}{output_17_9.png}
    \end{center}
    { \hspace*{\fill} \\}
    
    \begin{center}
    \adjustimage{max size={0.9\linewidth}{0.9\paperheight}}{output_17_10.png}
    \end{center}
    { \hspace*{\fill} \\}
    
    \begin{center}
    \adjustimage{max size={0.9\linewidth}{0.9\paperheight}}{output_17_11.png}
    \end{center}
    { \hspace*{\fill} \\}
    
    \begin{center}
    \adjustimage{max size={0.9\linewidth}{0.9\paperheight}}{output_17_12.png}
    \end{center}
    { \hspace*{\fill} \\}
    
    \begin{center}
    \adjustimage{max size={0.9\linewidth}{0.9\paperheight}}{output_17_13.png}
    \end{center}
    { \hspace*{\fill} \\}
    
    \begin{center}
    \adjustimage{max size={0.9\linewidth}{0.9\paperheight}}{output_17_14.png}
    \end{center}
    { \hspace*{\fill} \\}
    
    \begin{center}
    \adjustimage{max size={0.9\linewidth}{0.9\paperheight}}{output_17_15.png}
    \end{center}
    { \hspace*{\fill} \\}
    
    \begin{center}
    \adjustimage{max size={0.9\linewidth}{0.9\paperheight}}{output_17_16.png}
    \end{center}
    { \hspace*{\fill} \\}
    
    \begin{center}
    \adjustimage{max size={0.9\linewidth}{0.9\paperheight}}{output_17_17.png}
    \end{center}
    { \hspace*{\fill} \\}
    
    \begin{center}
    \adjustimage{max size={0.9\linewidth}{0.9\paperheight}}{output_17_18.png}
    \end{center}
    { \hspace*{\fill} \\}
    
\chapter{Выбор признаков, подходящих для построения моделей.
Кодирование категориальных признаков. Масштабирование данных.
Формирование вспомогательных признаков, улучшающих качество
моделей.}

    \begin{tcolorbox}[breakable, size=fbox, boxrule=1pt, pad at break*=1mm,colback=cellbackground, colframe=cellborder]
\prompt{In}{incolor}{10}{\boxspacing}
\begin{Verbatim}[commandchars=\\\{\}]
\PY{n}{data}\PY{o}{.}\PY{n}{dtypes}
\end{Verbatim}
\end{tcolorbox}

            \begin{tcolorbox}[breakable, size=fbox, boxrule=.5pt, pad at break*=1mm, opacityfill=0]
\prompt{Out}{outcolor}{10}{\boxspacing}
\begin{Verbatim}[commandchars=\\\{\}]
id                 int64
date              object
price            float64
bedrooms           int64
bathrooms        float64
sqft\_living        int64
sqft\_lot           int64
floors           float64
waterfront         int64
view               int64
condition          int64
grade              int64
sqft\_above         int64
sqft\_basement      int64
yr\_built           int64
yr\_renovated       int64
zipcode            int64
lat              float64
long             float64
sqft\_living15      int64
sqft\_lot15         int64
dtype: object
\end{Verbatim}
\end{tcolorbox}
        
    Пока для построения модели будем использовать все признаки, кроме
`date', т.к. не рассматриваем нашу модель как временную, и `id', т.к. он
содержит уникальынй id покупки, который мы не рассматриваем в модели
вообще.

    \begin{tcolorbox}[breakable, size=fbox, boxrule=1pt, pad at break*=1mm,colback=cellbackground, colframe=cellborder]
\prompt{In}{incolor}{11}{\boxspacing}
\begin{Verbatim}[commandchars=\\\{\}]
\PY{n}{data}\PY{o}{.}\PY{n}{drop}\PY{p}{(}\PY{p}{[}\PY{l+s+s1}{\PYZsq{}}\PY{l+s+s1}{id}\PY{l+s+s1}{\PYZsq{}}\PY{p}{,} \PY{l+s+s1}{\PYZsq{}}\PY{l+s+s1}{date}\PY{l+s+s1}{\PYZsq{}}\PY{p}{]}\PY{p}{,} \PY{n}{axis}\PY{o}{=}\PY{l+m+mi}{1}\PY{p}{,} \PY{n}{inplace}\PY{o}{=}\PY{k+kc}{True}\PY{p}{)}
\end{Verbatim}
\end{tcolorbox}

    Категориальные признаки отсутствуют, их кодирования не требуется(date не
рассматриваем).

Вспомогательные признаки для улучшения качества моделей в данном примере
мы строить не будем.

Выполним масштабирование данных(построим корреляционную карту, чтобы
сравнить её с полученной в корреляционном анализе).

    \begin{tcolorbox}[breakable, size=fbox, boxrule=1pt, pad at break*=1mm,colback=cellbackground, colframe=cellborder]
\prompt{In}{incolor}{12}{\boxspacing}
\begin{Verbatim}[commandchars=\\\{\}]
\PY{n}{fig}\PY{p}{,} \PY{n}{ax} \PY{o}{=} \PY{n}{plt}\PY{o}{.}\PY{n}{subplots}\PY{p}{(}\PY{n}{figsize}\PY{o}{=}\PY{p}{(}\PY{l+m+mi}{15}\PY{p}{,}\PY{l+m+mi}{7}\PY{p}{)}\PY{p}{)}
\PY{n}{sns}\PY{o}{.}\PY{n}{heatmap}\PY{p}{(}\PY{n}{data}\PY{o}{.}\PY{n}{corr}\PY{p}{(}\PY{n}{method}\PY{o}{=}\PY{l+s+s1}{\PYZsq{}}\PY{l+s+s1}{pearson}\PY{l+s+s1}{\PYZsq{}}\PY{p}{)}\PY{p}{,} \PY{n}{ax}\PY{o}{=}\PY{n}{ax}\PY{p}{,} \PY{n}{annot}\PY{o}{=}\PY{k+kc}{True}\PY{p}{,} \PY{n}{fmt}\PY{o}{=}\PY{l+s+s1}{\PYZsq{}}\PY{l+s+s1}{.2f}\PY{l+s+s1}{\PYZsq{}}\PY{p}{)}
\end{Verbatim}
\end{tcolorbox}

            \begin{tcolorbox}[breakable, size=fbox, boxrule=.5pt, pad at break*=1mm, opacityfill=0]
\prompt{Out}{outcolor}{12}{\boxspacing}
\begin{Verbatim}[commandchars=\\\{\}]
<AxesSubplot:>
\end{Verbatim}
\end{tcolorbox}
        
    \begin{center}
    \adjustimage{max size={0.9\linewidth}{0.9\paperheight}}{output_23_1.png}
    \end{center}
    { \hspace*{\fill} \\}
    
    \begin{tcolorbox}[breakable, size=fbox, boxrule=1pt, pad at break*=1mm,colback=cellbackground, colframe=cellborder]
\prompt{In}{incolor}{13}{\boxspacing}
\begin{Verbatim}[commandchars=\\\{\}]
\PY{c+c1}{\PYZsh{} Числовые колонки для масштабирования}
\PY{n}{scale\PYZus{}cols} \PY{o}{=} \PY{p}{[}\PY{l+s+s1}{\PYZsq{}}\PY{l+s+s1}{sqft\PYZus{}living}\PY{l+s+s1}{\PYZsq{}}\PY{p}{,} \PY{l+s+s1}{\PYZsq{}}\PY{l+s+s1}{sqft\PYZus{}lot}\PY{l+s+s1}{\PYZsq{}}\PY{p}{,} \PY{l+s+s1}{\PYZsq{}}\PY{l+s+s1}{sqft\PYZus{}above}\PY{l+s+s1}{\PYZsq{}}\PY{p}{,}
              \PY{l+s+s1}{\PYZsq{}}\PY{l+s+s1}{sqft\PYZus{}basement}\PY{l+s+s1}{\PYZsq{}}\PY{p}{,} \PY{l+s+s1}{\PYZsq{}}\PY{l+s+s1}{lat}\PY{l+s+s1}{\PYZsq{}}\PY{p}{,} \PY{l+s+s1}{\PYZsq{}}\PY{l+s+s1}{long}\PY{l+s+s1}{\PYZsq{}}\PY{p}{,}
              \PY{l+s+s1}{\PYZsq{}}\PY{l+s+s1}{sqft\PYZus{}living15}\PY{l+s+s1}{\PYZsq{}}\PY{p}{,} \PY{l+s+s1}{\PYZsq{}}\PY{l+s+s1}{sqft\PYZus{}lot15}\PY{l+s+s1}{\PYZsq{}}\PY{p}{,} \PY{l+s+s1}{\PYZsq{}}\PY{l+s+s1}{bedrooms}\PY{l+s+s1}{\PYZsq{}}\PY{p}{,} 
              \PY{l+s+s1}{\PYZsq{}}\PY{l+s+s1}{bathrooms}\PY{l+s+s1}{\PYZsq{}}\PY{p}{,} \PY{l+s+s1}{\PYZsq{}}\PY{l+s+s1}{view}\PY{l+s+s1}{\PYZsq{}}\PY{p}{,} \PY{l+s+s1}{\PYZsq{}}\PY{l+s+s1}{grade}\PY{l+s+s1}{\PYZsq{}}\PY{p}{,} \PY{l+s+s1}{\PYZsq{}}\PY{l+s+s1}{floors}\PY{l+s+s1}{\PYZsq{}}\PY{p}{,} \PY{l+s+s1}{\PYZsq{}}\PY{l+s+s1}{yr\PYZus{}renovated}\PY{l+s+s1}{\PYZsq{}}\PY{p}{,} \PY{l+s+s1}{\PYZsq{}}\PY{l+s+s1}{yr\PYZus{}built}\PY{l+s+s1}{\PYZsq{}}\PY{p}{,} \PY{l+s+s1}{\PYZsq{}}\PY{l+s+s1}{condition}\PY{l+s+s1}{\PYZsq{}}
             \PY{p}{]}
\end{Verbatim}
\end{tcolorbox}

    \begin{tcolorbox}[breakable, size=fbox, boxrule=1pt, pad at break*=1mm,colback=cellbackground, colframe=cellborder]
\prompt{In}{incolor}{14}{\boxspacing}
\begin{Verbatim}[commandchars=\\\{\}]
\PY{n}{sc} \PY{o}{=} \PY{n}{MinMaxScaler}\PY{p}{(}\PY{p}{)}
\PY{n}{sc\PYZus{}data} \PY{o}{=} \PY{n}{sc}\PY{o}{.}\PY{n}{fit\PYZus{}transform}\PY{p}{(}\PY{n}{data}\PY{p}{[}\PY{n}{scale\PYZus{}cols}\PY{p}{]}\PY{p}{)}
\end{Verbatim}
\end{tcolorbox}

    \begin{tcolorbox}[breakable, size=fbox, boxrule=1pt, pad at break*=1mm,colback=cellbackground, colframe=cellborder]
\prompt{In}{incolor}{15}{\boxspacing}
\begin{Verbatim}[commandchars=\\\{\}]
\PY{c+c1}{\PYZsh{} Добавим масштабированные данные в набор данных}
\PY{k}{for} \PY{n}{i} \PY{o+ow}{in} \PY{n+nb}{range}\PY{p}{(}\PY{n+nb}{len}\PY{p}{(}\PY{n}{scale\PYZus{}cols}\PY{p}{)}\PY{p}{)}\PY{p}{:}
    \PY{n}{col} \PY{o}{=} \PY{n}{scale\PYZus{}cols}\PY{p}{[}\PY{n}{i}\PY{p}{]}
    \PY{n}{new\PYZus{}col\PYZus{}name} \PY{o}{=} \PY{n}{col} \PY{o}{+} \PY{l+s+s1}{\PYZsq{}}\PY{l+s+s1}{\PYZus{}scaled}\PY{l+s+s1}{\PYZsq{}}
    \PY{n}{data}\PY{p}{[}\PY{n}{new\PYZus{}col\PYZus{}name}\PY{p}{]} \PY{o}{=} \PY{n}{sc\PYZus{}data}\PY{p}{[}\PY{p}{:}\PY{p}{,}\PY{n}{i}\PY{p}{]}
\end{Verbatim}
\end{tcolorbox}

    \begin{tcolorbox}[breakable, size=fbox, boxrule=1pt, pad at break*=1mm,colback=cellbackground, colframe=cellborder]
\prompt{In}{incolor}{16}{\boxspacing}
\begin{Verbatim}[commandchars=\\\{\}]
\PY{n}{data}\PY{o}{.}\PY{n}{head}\PY{p}{(}\PY{p}{)}
\end{Verbatim}
\end{tcolorbox}

            \begin{tcolorbox}[breakable, size=fbox, boxrule=.5pt, pad at break*=1mm, opacityfill=0]
\prompt{Out}{outcolor}{16}{\boxspacing}
\begin{Verbatim}[commandchars=\\\{\}]
      price  bedrooms  bathrooms  sqft\_living  sqft\_lot  floors  waterfront  \textbackslash{}
0  221900.0         3       1.00         1180      5650     1.0           0
1  538000.0         3       2.25         2570      7242     2.0           0
2  180000.0         2       1.00          770     10000     1.0           0
3  604000.0         4       3.00         1960      5000     1.0           0
4  510000.0         3       2.00         1680      8080     1.0           0

   view  condition  grade  {\ldots}  sqft\_living15\_scaled  sqft\_lot15\_scaled  \textbackslash{}
0     0          3      7  {\ldots}              0.161934           0.005742
1     0          3      7  {\ldots}              0.222165           0.008027
2     0          3      6  {\ldots}              0.399415           0.008513
3     0          5      7  {\ldots}              0.165376           0.004996
4     0          3      8  {\ldots}              0.241094           0.007871

   bedrooms\_scaled  bathrooms\_scaled  view\_scaled  grade\_scaled  \textbackslash{}
0         0.090909           0.12500          0.0      0.500000
1         0.090909           0.28125          0.0      0.500000
2         0.060606           0.12500          0.0      0.416667
3         0.121212           0.37500          0.0      0.500000
4         0.090909           0.25000          0.0      0.583333

   floors\_scaled  yr\_renovated\_scaled  yr\_built\_scaled  condition\_scaled
0            0.0             0.000000         0.478261               0.5
1            0.4             0.988089         0.443478               0.5
2            0.0             0.000000         0.286957               0.5
3            0.0             0.000000         0.565217               1.0
4            0.0             0.000000         0.756522               0.5

[5 rows x 35 columns]
\end{Verbatim}
\end{tcolorbox}
        
    \begin{tcolorbox}[breakable, size=fbox, boxrule=1pt, pad at break*=1mm,colback=cellbackground, colframe=cellborder]
\prompt{In}{incolor}{17}{\boxspacing}
\begin{Verbatim}[commandchars=\\\{\}]
\PY{c+c1}{\PYZsh{} Проверим, что масштабирование не повлияло на распределение данных}
\PY{k}{for} \PY{n}{col} \PY{o+ow}{in} \PY{n}{scale\PYZus{}cols}\PY{p}{:}
    \PY{n}{col\PYZus{}scaled} \PY{o}{=} \PY{n}{col} \PY{o}{+} \PY{l+s+s1}{\PYZsq{}}\PY{l+s+s1}{\PYZus{}scaled}\PY{l+s+s1}{\PYZsq{}}

    \PY{n}{fig}\PY{p}{,} \PY{n}{ax} \PY{o}{=} \PY{n}{plt}\PY{o}{.}\PY{n}{subplots}\PY{p}{(}\PY{l+m+mi}{1}\PY{p}{,} \PY{l+m+mi}{2}\PY{p}{,} \PY{n}{figsize}\PY{o}{=}\PY{p}{(}\PY{l+m+mi}{8}\PY{p}{,}\PY{l+m+mi}{3}\PY{p}{)}\PY{p}{)}
    \PY{n}{ax}\PY{p}{[}\PY{l+m+mi}{0}\PY{p}{]}\PY{o}{.}\PY{n}{hist}\PY{p}{(}\PY{n}{data}\PY{p}{[}\PY{n}{col}\PY{p}{]}\PY{p}{,} \PY{l+m+mi}{50}\PY{p}{)}
    \PY{n}{ax}\PY{p}{[}\PY{l+m+mi}{1}\PY{p}{]}\PY{o}{.}\PY{n}{hist}\PY{p}{(}\PY{n}{data}\PY{p}{[}\PY{n}{col\PYZus{}scaled}\PY{p}{]}\PY{p}{,} \PY{l+m+mi}{50}\PY{p}{)}
    \PY{n}{ax}\PY{p}{[}\PY{l+m+mi}{0}\PY{p}{]}\PY{o}{.}\PY{n}{title}\PY{o}{.}\PY{n}{set\PYZus{}text}\PY{p}{(}\PY{n}{col}\PY{p}{)}
    \PY{n}{ax}\PY{p}{[}\PY{l+m+mi}{1}\PY{p}{]}\PY{o}{.}\PY{n}{title}\PY{o}{.}\PY{n}{set\PYZus{}text}\PY{p}{(}\PY{n}{col\PYZus{}scaled}\PY{p}{)}
    \PY{n}{plt}\PY{o}{.}\PY{n}{show}\PY{p}{(}\PY{p}{)}
\end{Verbatim}
\end{tcolorbox}

    \begin{center}
    \adjustimage{max size={0.9\linewidth}{0.9\paperheight}}{output_28_0.png}
    \end{center}
    { \hspace*{\fill} \\}
    
    \begin{center}
    \adjustimage{max size={0.9\linewidth}{0.9\paperheight}}{output_28_1.png}
    \end{center}
    { \hspace*{\fill} \\}
    
    \begin{center}
    \adjustimage{max size={0.9\linewidth}{0.9\paperheight}}{output_28_2.png}
    \end{center}
    { \hspace*{\fill} \\}
    
    \begin{center}
    \adjustimage{max size={0.9\linewidth}{0.9\paperheight}}{output_28_3.png}
    \end{center}
    { \hspace*{\fill} \\}
    
    \begin{center}
    \adjustimage{max size={0.9\linewidth}{0.9\paperheight}}{output_28_4.png}
    \end{center}
    { \hspace*{\fill} \\}
    
    \begin{center}
    \adjustimage{max size={0.9\linewidth}{0.9\paperheight}}{output_28_5.png}
    \end{center}
    { \hspace*{\fill} \\}
    
    \begin{center}
    \adjustimage{max size={0.9\linewidth}{0.9\paperheight}}{output_28_6.png}
    \end{center}
    { \hspace*{\fill} \\}
    
    \begin{center}
    \adjustimage{max size={0.9\linewidth}{0.9\paperheight}}{output_28_7.png}
    \end{center}
    { \hspace*{\fill} \\}
    
    \begin{center}
    \adjustimage{max size={0.9\linewidth}{0.9\paperheight}}{output_28_8.png}
    \end{center}
    { \hspace*{\fill} \\}
    
    \begin{center}
    \adjustimage{max size={0.9\linewidth}{0.9\paperheight}}{output_28_9.png}
    \end{center}
    { \hspace*{\fill} \\}
    
    \begin{center}
    \adjustimage{max size={0.9\linewidth}{0.9\paperheight}}{output_28_10.png}
    \end{center}
    { \hspace*{\fill} \\}
    
    \begin{center}
    \adjustimage{max size={0.9\linewidth}{0.9\paperheight}}{output_28_11.png}
    \end{center}
    { \hspace*{\fill} \\}
    
    \begin{center}
    \adjustimage{max size={0.9\linewidth}{0.9\paperheight}}{output_28_12.png}
    \end{center}
    { \hspace*{\fill} \\}
    
    \begin{center}
    \adjustimage{max size={0.9\linewidth}{0.9\paperheight}}{output_28_13.png}
    \end{center}
    { \hspace*{\fill} \\}
    
    \begin{center}
    \adjustimage{max size={0.9\linewidth}{0.9\paperheight}}{output_28_14.png}
    \end{center}
    { \hspace*{\fill} \\}
    
    \begin{center}
    \adjustimage{max size={0.9\linewidth}{0.9\paperheight}}{output_28_15.png}
    \end{center}
    { \hspace*{\fill} \\}
    
    Удалим стаыре столбцы.

    \begin{tcolorbox}[breakable, size=fbox, boxrule=1pt, pad at break*=1mm,colback=cellbackground, colframe=cellborder]
\prompt{In}{incolor}{18}{\boxspacing}
\begin{Verbatim}[commandchars=\\\{\}]
\PY{n}{data}\PY{o}{.}\PY{n}{drop}\PY{p}{(}\PY{n}{scale\PYZus{}cols}\PY{p}{,} \PY{n}{axis}\PY{o}{=}\PY{l+m+mi}{1}\PY{p}{,} \PY{n}{inplace}\PY{o}{=}\PY{k+kc}{True}\PY{p}{)}
\end{Verbatim}
\end{tcolorbox}

    \begin{tcolorbox}[breakable, size=fbox, boxrule=1pt, pad at break*=1mm,colback=cellbackground, colframe=cellborder]
\prompt{In}{incolor}{19}{\boxspacing}
\begin{Verbatim}[commandchars=\\\{\}]
\PY{n}{data}\PY{o}{.}\PY{n}{head}\PY{p}{(}\PY{p}{)}
\end{Verbatim}
\end{tcolorbox}

            \begin{tcolorbox}[breakable, size=fbox, boxrule=.5pt, pad at break*=1mm, opacityfill=0]
\prompt{Out}{outcolor}{19}{\boxspacing}
\begin{Verbatim}[commandchars=\\\{\}]
      price  waterfront  zipcode  sqft\_living\_scaled  sqft\_lot\_scaled  \textbackslash{}
0  221900.0           0    98178            0.067170         0.003108
1  538000.0           0    98125            0.172075         0.004072
2  180000.0           0    98028            0.036226         0.005743
3  604000.0           0    98136            0.126038         0.002714
4  510000.0           0    98074            0.104906         0.004579

   sqft\_above\_scaled  sqft\_basement\_scaled  lat\_scaled  long\_scaled  \textbackslash{}
0           0.097588              0.000000    0.571498     0.217608
1           0.206140              0.082988    0.908959     0.166113
2           0.052632              0.000000    0.936143     0.237542
3           0.083333              0.188797    0.586939     0.104651
4           0.152412              0.000000    0.741354     0.393688

   sqft\_living15\_scaled  sqft\_lot15\_scaled  bedrooms\_scaled  bathrooms\_scaled  \textbackslash{}
0              0.161934           0.005742         0.090909           0.12500
1              0.222165           0.008027         0.090909           0.28125
2              0.399415           0.008513         0.060606           0.12500
3              0.165376           0.004996         0.121212           0.37500
4              0.241094           0.007871         0.090909           0.25000

   view\_scaled  grade\_scaled  floors\_scaled  yr\_renovated\_scaled  \textbackslash{}
0          0.0      0.500000            0.0             0.000000
1          0.0      0.500000            0.4             0.988089
2          0.0      0.416667            0.0             0.000000
3          0.0      0.500000            0.0             0.000000
4          0.0      0.583333            0.0             0.000000

   yr\_built\_scaled  condition\_scaled
0         0.478261               0.5
1         0.443478               0.5
2         0.286957               0.5
3         0.565217               1.0
4         0.756522               0.5
\end{Verbatim}
\end{tcolorbox}
        
\chapter{Проведение корреляционного анализа данных. Формирование
промежуточных выводов о возможности построения моделей машинного
обучения}

    \begin{tcolorbox}[breakable, size=fbox, boxrule=1pt, pad at break*=1mm,colback=cellbackground, colframe=cellborder]
\prompt{In}{incolor}{20}{\boxspacing}
\begin{Verbatim}[commandchars=\\\{\}]
\PY{n}{fig}\PY{p}{,} \PY{n}{ax} \PY{o}{=} \PY{n}{plt}\PY{o}{.}\PY{n}{subplots}\PY{p}{(}\PY{n}{figsize}\PY{o}{=}\PY{p}{(}\PY{l+m+mi}{15}\PY{p}{,}\PY{l+m+mi}{7}\PY{p}{)}\PY{p}{)}
\PY{n}{sns}\PY{o}{.}\PY{n}{heatmap}\PY{p}{(}\PY{n}{data}\PY{o}{.}\PY{n}{corr}\PY{p}{(}\PY{n}{method}\PY{o}{=}\PY{l+s+s1}{\PYZsq{}}\PY{l+s+s1}{pearson}\PY{l+s+s1}{\PYZsq{}}\PY{p}{)}\PY{p}{,} \PY{n}{ax}\PY{o}{=}\PY{n}{ax}\PY{p}{,} \PY{n}{annot}\PY{o}{=}\PY{k+kc}{True}\PY{p}{,} \PY{n}{fmt}\PY{o}{=}\PY{l+s+s1}{\PYZsq{}}\PY{l+s+s1}{.2f}\PY{l+s+s1}{\PYZsq{}}\PY{p}{)}
\end{Verbatim}
\end{tcolorbox}

            \begin{tcolorbox}[breakable, size=fbox, boxrule=.5pt, pad at break*=1mm, opacityfill=0]
\prompt{Out}{outcolor}{20}{\boxspacing}
\begin{Verbatim}[commandchars=\\\{\}]
<AxesSubplot:>
\end{Verbatim}
\end{tcolorbox}
        
    \begin{center}
    \adjustimage{max size={0.9\linewidth}{0.9\paperheight}}{output_33_1.png}
    \end{center}
    { \hspace*{\fill} \\}
    
    Удалим столбцы, которые плохо коррелируют с целевым признаком `price'

    \begin{tcolorbox}[breakable, size=fbox, boxrule=1pt, pad at break*=1mm,colback=cellbackground, colframe=cellborder]
\prompt{In}{incolor}{21}{\boxspacing}
\begin{Verbatim}[commandchars=\\\{\}]
\PY{n}{data}\PY{o}{.}\PY{n}{drop}\PY{p}{(}\PY{p}{[}\PY{l+s+s1}{\PYZsq{}}\PY{l+s+s1}{long\PYZus{}scaled}\PY{l+s+s1}{\PYZsq{}} \PY{p}{,} \PY{l+s+s1}{\PYZsq{}}\PY{l+s+s1}{zipcode}\PY{l+s+s1}{\PYZsq{}}\PY{p}{,} \PY{l+s+s1}{\PYZsq{}}\PY{l+s+s1}{yr\PYZus{}built\PYZus{}scaled}\PY{l+s+s1}{\PYZsq{}}\PY{p}{,} \PY{l+s+s1}{\PYZsq{}}\PY{l+s+s1}{condition\PYZus{}scaled}\PY{l+s+s1}{\PYZsq{}}\PY{p}{,} \PY{l+s+s1}{\PYZsq{}}\PY{l+s+s1}{sqft\PYZus{}lot\PYZus{}scaled}\PY{l+s+s1}{\PYZsq{}}\PY{p}{,} \PY{l+s+s1}{\PYZsq{}}\PY{l+s+s1}{sqft\PYZus{}lot15\PYZus{}scaled}\PY{l+s+s1}{\PYZsq{}}\PY{p}{]}\PY{p}{,} \PY{n}{axis}\PY{o}{=}\PY{l+m+mi}{1}\PY{p}{,} \PY{n}{inplace}\PY{o}{=}\PY{k+kc}{True}\PY{p}{)}
\end{Verbatim}
\end{tcolorbox}

    \begin{tcolorbox}[breakable, size=fbox, boxrule=1pt, pad at break*=1mm,colback=cellbackground, colframe=cellborder]
\prompt{In}{incolor}{22}{\boxspacing}
\begin{Verbatim}[commandchars=\\\{\}]
\PY{n}{fig}\PY{p}{,} \PY{n}{ax} \PY{o}{=} \PY{n}{plt}\PY{o}{.}\PY{n}{subplots}\PY{p}{(}\PY{n}{figsize}\PY{o}{=}\PY{p}{(}\PY{l+m+mi}{15}\PY{p}{,}\PY{l+m+mi}{7}\PY{p}{)}\PY{p}{)}
\PY{n}{sns}\PY{o}{.}\PY{n}{heatmap}\PY{p}{(}\PY{n}{data}\PY{o}{.}\PY{n}{corr}\PY{p}{(}\PY{n}{method}\PY{o}{=}\PY{l+s+s1}{\PYZsq{}}\PY{l+s+s1}{pearson}\PY{l+s+s1}{\PYZsq{}}\PY{p}{)}\PY{p}{,} \PY{n}{ax}\PY{o}{=}\PY{n}{ax}\PY{p}{,} \PY{n}{annot}\PY{o}{=}\PY{k+kc}{True}\PY{p}{,} \PY{n}{fmt}\PY{o}{=}\PY{l+s+s1}{\PYZsq{}}\PY{l+s+s1}{.2f}\PY{l+s+s1}{\PYZsq{}}\PY{p}{)}
\end{Verbatim}
\end{tcolorbox}

            \begin{tcolorbox}[breakable, size=fbox, boxrule=.5pt, pad at break*=1mm, opacityfill=0]
\prompt{Out}{outcolor}{22}{\boxspacing}
\begin{Verbatim}[commandchars=\\\{\}]
<AxesSubplot:>
\end{Verbatim}
\end{tcolorbox}
        
    \begin{center}
    \adjustimage{max size={0.9\linewidth}{0.9\paperheight}}{output_36_1.png}
    \end{center}
    { \hspace*{\fill} \\}
    
    Вывод: * Корреляционные матрицы для исходных и масштабированных данных
совпадают. * Целевой признак `price' хорошо коррелирует с
`grade\_scaled', `sqft\_living\_scaled', `sqft\_living15\_scaled',
`sqft\_above\_scaled', `bathrooms\_scaled', их точно надо оставить в
модели регрессии * Признаки `long\_scaled' , `zipcode', `yr\_built',
`condition', `sqft\_lot\_scaled', `sqft\_lot15\_scaled' почти не
коррелируют с целевым признаком, их лучше удалить из модели * Большие по
модулю значения коэффициентов корреляции свидетельствуют о значимой
корреляции между исходными признаками и целевым признаком. На основании
корреляционной матрицы можно сделать вывод о том, что данные позволяют
построить модель машинного обучения.

\chapter{Выбор метрик для последующей оценки качества
моделей.}

    В качестве метрик для решения задачи регрессии будем испсользовать: Mean
Absolute Error, Mean Squared Error и r2 оценку.

\section{Сохранение и визуализация метрик}

    Разработаем класс, который поможет нам хранить метрики, их значения и
визуализировать их.

    \begin{tcolorbox}[breakable, size=fbox, boxrule=1pt, pad at break*=1mm,colback=cellbackground, colframe=cellborder]
\prompt{In}{incolor}{23}{\boxspacing}
\begin{Verbatim}[commandchars=\\\{\}]
\PY{k}{class} \PY{n+nc}{MetricLogger}\PY{p}{:}
    
    \PY{k}{def} \PY{n+nf+fm}{\PYZus{}\PYZus{}init\PYZus{}\PYZus{}}\PY{p}{(}\PY{n+nb+bp}{self}\PY{p}{)}\PY{p}{:}
        \PY{n+nb+bp}{self}\PY{o}{.}\PY{n}{df} \PY{o}{=} \PY{n}{pd}\PY{o}{.}\PY{n}{DataFrame}\PY{p}{(}
            \PY{p}{\PYZob{}}\PY{l+s+s1}{\PYZsq{}}\PY{l+s+s1}{metric}\PY{l+s+s1}{\PYZsq{}}\PY{p}{:} \PY{n}{pd}\PY{o}{.}\PY{n}{Series}\PY{p}{(}\PY{p}{[}\PY{p}{]}\PY{p}{,} \PY{n}{dtype}\PY{o}{=}\PY{l+s+s1}{\PYZsq{}}\PY{l+s+s1}{str}\PY{l+s+s1}{\PYZsq{}}\PY{p}{)}\PY{p}{,}
            \PY{l+s+s1}{\PYZsq{}}\PY{l+s+s1}{alg}\PY{l+s+s1}{\PYZsq{}}\PY{p}{:} \PY{n}{pd}\PY{o}{.}\PY{n}{Series}\PY{p}{(}\PY{p}{[}\PY{p}{]}\PY{p}{,} \PY{n}{dtype}\PY{o}{=}\PY{l+s+s1}{\PYZsq{}}\PY{l+s+s1}{str}\PY{l+s+s1}{\PYZsq{}}\PY{p}{)}\PY{p}{,}
            \PY{l+s+s1}{\PYZsq{}}\PY{l+s+s1}{value}\PY{l+s+s1}{\PYZsq{}}\PY{p}{:} \PY{n}{pd}\PY{o}{.}\PY{n}{Series}\PY{p}{(}\PY{p}{[}\PY{p}{]}\PY{p}{,} \PY{n}{dtype}\PY{o}{=}\PY{l+s+s1}{\PYZsq{}}\PY{l+s+s1}{float}\PY{l+s+s1}{\PYZsq{}}\PY{p}{)}\PY{p}{\PYZcb{}}\PY{p}{)}

    \PY{k}{def} \PY{n+nf}{add}\PY{p}{(}\PY{n+nb+bp}{self}\PY{p}{,} \PY{n}{metric}\PY{p}{,} \PY{n}{alg}\PY{p}{,} \PY{n}{value}\PY{p}{)}\PY{p}{:}
        \PY{l+s+sd}{\PYZdq{}\PYZdq{}\PYZdq{}}
\PY{l+s+sd}{        Добавление значения}
\PY{l+s+sd}{        \PYZdq{}\PYZdq{}\PYZdq{}}
        \PY{c+c1}{\PYZsh{} Удаление значения если оно уже было ранее добавлено}
        \PY{n+nb+bp}{self}\PY{o}{.}\PY{n}{df}\PY{o}{.}\PY{n}{drop}\PY{p}{(}\PY{n+nb+bp}{self}\PY{o}{.}\PY{n}{df}\PY{p}{[}\PY{p}{(}\PY{n+nb+bp}{self}\PY{o}{.}\PY{n}{df}\PY{p}{[}\PY{l+s+s1}{\PYZsq{}}\PY{l+s+s1}{metric}\PY{l+s+s1}{\PYZsq{}}\PY{p}{]}\PY{o}{==}\PY{n}{metric}\PY{p}{)}\PY{o}{\PYZam{}}\PY{p}{(}\PY{n+nb+bp}{self}\PY{o}{.}\PY{n}{df}\PY{p}{[}\PY{l+s+s1}{\PYZsq{}}\PY{l+s+s1}{alg}\PY{l+s+s1}{\PYZsq{}}\PY{p}{]}\PY{o}{==}\PY{n}{alg}\PY{p}{)}\PY{p}{]}\PY{o}{.}\PY{n}{index}\PY{p}{,} \PY{n}{inplace} \PY{o}{=} \PY{k+kc}{True}\PY{p}{)}
        \PY{c+c1}{\PYZsh{} Добавление нового значения}
        \PY{n}{temp} \PY{o}{=} \PY{p}{[}\PY{p}{\PYZob{}}\PY{l+s+s1}{\PYZsq{}}\PY{l+s+s1}{metric}\PY{l+s+s1}{\PYZsq{}}\PY{p}{:}\PY{n}{metric}\PY{p}{,} \PY{l+s+s1}{\PYZsq{}}\PY{l+s+s1}{alg}\PY{l+s+s1}{\PYZsq{}}\PY{p}{:}\PY{n}{alg}\PY{p}{,} \PY{l+s+s1}{\PYZsq{}}\PY{l+s+s1}{value}\PY{l+s+s1}{\PYZsq{}}\PY{p}{:}\PY{n}{value}\PY{p}{\PYZcb{}}\PY{p}{]}
        \PY{n+nb+bp}{self}\PY{o}{.}\PY{n}{df} \PY{o}{=} \PY{n+nb+bp}{self}\PY{o}{.}\PY{n}{df}\PY{o}{.}\PY{n}{append}\PY{p}{(}\PY{n}{temp}\PY{p}{,} \PY{n}{ignore\PYZus{}index}\PY{o}{=}\PY{k+kc}{True}\PY{p}{)}

    \PY{k}{def} \PY{n+nf}{get\PYZus{}data\PYZus{}for\PYZus{}metric}\PY{p}{(}\PY{n+nb+bp}{self}\PY{p}{,} \PY{n}{metric}\PY{p}{,} \PY{n}{ascending}\PY{o}{=}\PY{k+kc}{True}\PY{p}{)}\PY{p}{:}
        \PY{l+s+sd}{\PYZdq{}\PYZdq{}\PYZdq{}}
\PY{l+s+sd}{        Формирование данных с фильтром по метрике}
\PY{l+s+sd}{        \PYZdq{}\PYZdq{}\PYZdq{}}
        \PY{n}{temp\PYZus{}data} \PY{o}{=} \PY{n+nb+bp}{self}\PY{o}{.}\PY{n}{df}\PY{p}{[}\PY{n+nb+bp}{self}\PY{o}{.}\PY{n}{df}\PY{p}{[}\PY{l+s+s1}{\PYZsq{}}\PY{l+s+s1}{metric}\PY{l+s+s1}{\PYZsq{}}\PY{p}{]}\PY{o}{==}\PY{n}{metric}\PY{p}{]}
        \PY{n}{temp\PYZus{}data\PYZus{}2} \PY{o}{=} \PY{n}{temp\PYZus{}data}\PY{o}{.}\PY{n}{sort\PYZus{}values}\PY{p}{(}\PY{n}{by}\PY{o}{=}\PY{l+s+s1}{\PYZsq{}}\PY{l+s+s1}{value}\PY{l+s+s1}{\PYZsq{}}\PY{p}{,} \PY{n}{ascending}\PY{o}{=}\PY{n}{ascending}\PY{p}{)}
        \PY{k}{return} \PY{n}{temp\PYZus{}data\PYZus{}2}\PY{p}{[}\PY{l+s+s1}{\PYZsq{}}\PY{l+s+s1}{alg}\PY{l+s+s1}{\PYZsq{}}\PY{p}{]}\PY{o}{.}\PY{n}{values}\PY{p}{,} \PY{n}{temp\PYZus{}data\PYZus{}2}\PY{p}{[}\PY{l+s+s1}{\PYZsq{}}\PY{l+s+s1}{value}\PY{l+s+s1}{\PYZsq{}}\PY{p}{]}\PY{o}{.}\PY{n}{values}
    
    \PY{k}{def} \PY{n+nf}{plot}\PY{p}{(}\PY{n+nb+bp}{self}\PY{p}{,} \PY{n}{str\PYZus{}header}\PY{p}{,} \PY{n}{metric}\PY{p}{,} \PY{n}{ascending}\PY{o}{=}\PY{k+kc}{True}\PY{p}{,} \PY{n}{figsize}\PY{o}{=}\PY{p}{(}\PY{l+m+mi}{5}\PY{p}{,} \PY{l+m+mi}{5}\PY{p}{)}\PY{p}{)}\PY{p}{:}
        \PY{l+s+sd}{\PYZdq{}\PYZdq{}\PYZdq{}}
\PY{l+s+sd}{        Вывод графика}
\PY{l+s+sd}{        \PYZdq{}\PYZdq{}\PYZdq{}}
        \PY{n}{array\PYZus{}labels}\PY{p}{,} \PY{n}{array\PYZus{}metric} \PY{o}{=} \PY{n+nb+bp}{self}\PY{o}{.}\PY{n}{get\PYZus{}data\PYZus{}for\PYZus{}metric}\PY{p}{(}\PY{n}{metric}\PY{p}{,} \PY{n}{ascending}\PY{p}{)}
        \PY{n}{fig}\PY{p}{,} \PY{n}{ax1} \PY{o}{=} \PY{n}{plt}\PY{o}{.}\PY{n}{subplots}\PY{p}{(}\PY{n}{figsize}\PY{o}{=}\PY{n}{figsize}\PY{p}{)}
        \PY{n}{pos} \PY{o}{=} \PY{n}{np}\PY{o}{.}\PY{n}{arange}\PY{p}{(}\PY{n+nb}{len}\PY{p}{(}\PY{n}{array\PYZus{}metric}\PY{p}{)}\PY{p}{)}
        \PY{n}{rects} \PY{o}{=} \PY{n}{ax1}\PY{o}{.}\PY{n}{barh}\PY{p}{(}\PY{n}{pos}\PY{p}{,} \PY{n}{array\PYZus{}metric}\PY{p}{,}
                         \PY{n}{align}\PY{o}{=}\PY{l+s+s1}{\PYZsq{}}\PY{l+s+s1}{center}\PY{l+s+s1}{\PYZsq{}}\PY{p}{,}
                         \PY{n}{height}\PY{o}{=}\PY{l+m+mf}{0.5}\PY{p}{,} 
                         \PY{n}{tick\PYZus{}label}\PY{o}{=}\PY{n}{array\PYZus{}labels}\PY{p}{)}
        \PY{n}{ax1}\PY{o}{.}\PY{n}{set\PYZus{}title}\PY{p}{(}\PY{n}{str\PYZus{}header}\PY{p}{)}
        \PY{k}{for} \PY{n}{a}\PY{p}{,}\PY{n}{b} \PY{o+ow}{in} \PY{n+nb}{zip}\PY{p}{(}\PY{n}{pos}\PY{p}{,} \PY{n}{array\PYZus{}metric}\PY{p}{)}\PY{p}{:}
            \PY{n}{plt}\PY{o}{.}\PY{n}{text}\PY{p}{(}\PY{l+m+mf}{0.5}\PY{p}{,} \PY{n}{a}\PY{o}{\PYZhy{}}\PY{l+m+mf}{0.05}\PY{p}{,} \PY{n+nb}{str}\PY{p}{(}\PY{n+nb}{round}\PY{p}{(}\PY{n}{b}\PY{p}{,}\PY{l+m+mi}{3}\PY{p}{)}\PY{p}{)}\PY{p}{,} \PY{n}{color}\PY{o}{=}\PY{l+s+s1}{\PYZsq{}}\PY{l+s+s1}{white}\PY{l+s+s1}{\PYZsq{}}\PY{p}{)}
        \PY{n}{plt}\PY{o}{.}\PY{n}{show}\PY{p}{(}\PY{p}{)}  
\end{Verbatim}
\end{tcolorbox}

\chapter{Выбор наиболее подходящих моделей для решения задачи
классификации или
регрессии.}

    Для задачи регрессии будем использовать следующие модели:

\begin{itemize}
\tightlist
\item
  Линейная регрессия
\item
  Метод ближайших соседей
\item
  Машина опорных векторов (линейный алгоритм)
\item
  Решающее дерево
\item
  Случайный лес
\item
  Градиентный бустинг
\end{itemize}

\chapter{Формирование обучающей и тестовой выборок на основе
исходного набора
данных}

    \begin{tcolorbox}[breakable, size=fbox, boxrule=1pt, pad at break*=1mm,colback=cellbackground, colframe=cellborder]
\prompt{In}{incolor}{24}{\boxspacing}
\begin{Verbatim}[commandchars=\\\{\}]
\PY{n}{data}\PY{o}{.}\PY{n}{columns}
\end{Verbatim}
\end{tcolorbox}

            \begin{tcolorbox}[breakable, size=fbox, boxrule=.5pt, pad at break*=1mm, opacityfill=0]
\prompt{Out}{outcolor}{24}{\boxspacing}
\begin{Verbatim}[commandchars=\\\{\}]
Index(['price', 'waterfront', 'sqft\_living\_scaled', 'sqft\_above\_scaled',
       'sqft\_basement\_scaled', 'lat\_scaled', 'sqft\_living15\_scaled',
       'bedrooms\_scaled', 'bathrooms\_scaled', 'view\_scaled', 'grade\_scaled',
       'floors\_scaled', 'yr\_renovated\_scaled'],
      dtype='object')
\end{Verbatim}
\end{tcolorbox}
        
    \begin{tcolorbox}[breakable, size=fbox, boxrule=1pt, pad at break*=1mm,colback=cellbackground, colframe=cellborder]
\prompt{In}{incolor}{25}{\boxspacing}
\begin{Verbatim}[commandchars=\\\{\}]
\PY{n}{feature\PYZus{}cols} \PY{o}{=} \PY{p}{[}
    \PY{l+s+s1}{\PYZsq{}}\PY{l+s+s1}{bedrooms\PYZus{}scaled}\PY{l+s+s1}{\PYZsq{}}\PY{p}{,} \PY{l+s+s1}{\PYZsq{}}\PY{l+s+s1}{bathrooms\PYZus{}scaled}\PY{l+s+s1}{\PYZsq{}}\PY{p}{,} \PY{l+s+s1}{\PYZsq{}}\PY{l+s+s1}{floors\PYZus{}scaled}\PY{l+s+s1}{\PYZsq{}}\PY{p}{,} \PY{l+s+s1}{\PYZsq{}}\PY{l+s+s1}{waterfront}\PY{l+s+s1}{\PYZsq{}}\PY{p}{,} \PY{l+s+s1}{\PYZsq{}}\PY{l+s+s1}{view\PYZus{}scaled}\PY{l+s+s1}{\PYZsq{}}\PY{p}{,}
    \PY{l+s+s1}{\PYZsq{}}\PY{l+s+s1}{grade\PYZus{}scaled}\PY{l+s+s1}{\PYZsq{}}\PY{p}{,} \PY{l+s+s1}{\PYZsq{}}\PY{l+s+s1}{yr\PYZus{}renovated\PYZus{}scaled}\PY{l+s+s1}{\PYZsq{}}\PY{p}{,} \PY{l+s+s1}{\PYZsq{}}\PY{l+s+s1}{sqft\PYZus{}living\PYZus{}scaled}\PY{l+s+s1}{\PYZsq{}}\PY{p}{,} \PY{l+s+s1}{\PYZsq{}}\PY{l+s+s1}{sqft\PYZus{}above\PYZus{}scaled}\PY{l+s+s1}{\PYZsq{}}\PY{p}{,}
    \PY{l+s+s1}{\PYZsq{}}\PY{l+s+s1}{sqft\PYZus{}basement\PYZus{}scaled}\PY{l+s+s1}{\PYZsq{}}\PY{p}{,} \PY{l+s+s1}{\PYZsq{}}\PY{l+s+s1}{lat\PYZus{}scaled}\PY{l+s+s1}{\PYZsq{}}\PY{p}{,} \PY{l+s+s1}{\PYZsq{}}\PY{l+s+s1}{sqft\PYZus{}living15\PYZus{}scaled}\PY{l+s+s1}{\PYZsq{}}
\PY{p}{]}
\PY{n}{data\PYZus{}X} \PY{o}{=} \PY{n}{data}\PY{o}{.}\PY{n}{loc}\PY{p}{[}\PY{p}{:}\PY{p}{,}\PY{n}{feature\PYZus{}cols}\PY{p}{]}
\PY{n}{data\PYZus{}Y} \PY{o}{=} \PY{n}{data}\PY{o}{.}\PY{n}{loc}\PY{p}{[}\PY{p}{:}\PY{p}{,} \PY{l+s+s1}{\PYZsq{}}\PY{l+s+s1}{price}\PY{l+s+s1}{\PYZsq{}}\PY{p}{]}
\PY{n}{data\PYZus{}X\PYZus{}train}\PY{p}{,} \PY{n}{data\PYZus{}X\PYZus{}test}\PY{p}{,} \PY{n}{data\PYZus{}y\PYZus{}train}\PY{p}{,} \PY{n}{data\PYZus{}y\PYZus{}test} \PY{o}{=} \PY{n}{train\PYZus{}test\PYZus{}split}\PY{p}{(}\PY{n}{data\PYZus{}X}\PY{p}{,} \PY{n}{data\PYZus{}Y}\PY{p}{,}\PY{n}{test\PYZus{}size}\PY{o}{=}\PY{l+m+mf}{0.2}\PY{p}{,} \PY{n}{random\PYZus{}state}\PY{o}{=}\PY{l+m+mi}{1}\PY{p}{)}
\end{Verbatim}
\end{tcolorbox}

\chapter{Построение базового решения (baseline) для выбранных моделей
без подбора гиперпараметров. Производится обучение моделей на основе
обучающей выборки и оценка качества моделей на основе тестовой
выборки.}

    \begin{tcolorbox}[breakable, size=fbox, boxrule=1pt, pad at break*=1mm,colback=cellbackground, colframe=cellborder]
\prompt{In}{incolor}{26}{\boxspacing}
\begin{Verbatim}[commandchars=\\\{\}]
\PY{c+c1}{\PYZsh{} Модели}
\PY{n}{regr\PYZus{}models} \PY{o}{=} \PY{p}{\PYZob{}}\PY{l+s+s1}{\PYZsq{}}\PY{l+s+s1}{LR}\PY{l+s+s1}{\PYZsq{}}\PY{p}{:} \PY{n}{LinearRegression}\PY{p}{(}\PY{p}{)}\PY{p}{,} 
               \PY{l+s+s1}{\PYZsq{}}\PY{l+s+s1}{KNN\PYZus{}1}\PY{l+s+s1}{\PYZsq{}}\PY{p}{:}\PY{n}{KNeighborsRegressor}\PY{p}{(}\PY{n}{n\PYZus{}neighbors}\PY{o}{=}\PY{l+m+mi}{1}\PY{p}{)}\PY{p}{,}
               \PY{l+s+s1}{\PYZsq{}}\PY{l+s+s1}{LinearSVR}\PY{l+s+s1}{\PYZsq{}}\PY{p}{:}\PY{n}{LinearSVR}\PY{p}{(}\PY{n}{C}\PY{o}{=}\PY{l+m+mf}{1.0}\PY{p}{)}\PY{p}{,}
               \PY{l+s+s1}{\PYZsq{}}\PY{l+s+s1}{Tree}\PY{l+s+s1}{\PYZsq{}}\PY{p}{:}\PY{n}{DecisionTreeRegressor}\PY{p}{(}\PY{p}{)}\PY{p}{,}
               \PY{l+s+s1}{\PYZsq{}}\PY{l+s+s1}{RF}\PY{l+s+s1}{\PYZsq{}}\PY{p}{:}\PY{n}{RandomForestRegressor}\PY{p}{(}\PY{p}{)}\PY{p}{,}
               \PY{l+s+s1}{\PYZsq{}}\PY{l+s+s1}{GB}\PY{l+s+s1}{\PYZsq{}}\PY{p}{:}\PY{n}{GradientBoostingRegressor}\PY{p}{(}\PY{p}{)}\PY{p}{\PYZcb{}}
\end{Verbatim}
\end{tcolorbox}

    \begin{tcolorbox}[breakable, size=fbox, boxrule=1pt, pad at break*=1mm,colback=cellbackground, colframe=cellborder]
\prompt{In}{incolor}{27}{\boxspacing}
\begin{Verbatim}[commandchars=\\\{\}]
\PY{c+c1}{\PYZsh{} Сохранение метрик}
\PY{n}{regrMetricLogger} \PY{o}{=} \PY{n}{MetricLogger}\PY{p}{(}\PY{p}{)}
\end{Verbatim}
\end{tcolorbox}

    \begin{tcolorbox}[breakable, size=fbox, boxrule=1pt, pad at break*=1mm,colback=cellbackground, colframe=cellborder]
\prompt{In}{incolor}{28}{\boxspacing}
\begin{Verbatim}[commandchars=\\\{\}]
\PY{k}{def} \PY{n+nf}{regr\PYZus{}train\PYZus{}model}\PY{p}{(}\PY{n}{model\PYZus{}name}\PY{p}{,} \PY{n}{model}\PY{p}{,} \PY{n}{regrMetricLogger}\PY{p}{)}\PY{p}{:}
    \PY{n}{model}\PY{o}{.}\PY{n}{fit}\PY{p}{(}\PY{n}{data\PYZus{}X\PYZus{}train}\PY{p}{,} \PY{n}{data\PYZus{}y\PYZus{}train}\PY{p}{)}
    \PY{n}{y\PYZus{}pred} \PY{o}{=} \PY{n}{model}\PY{o}{.}\PY{n}{predict}\PY{p}{(}\PY{n}{data\PYZus{}X\PYZus{}test}\PY{p}{)}
    
    \PY{n}{mae} \PY{o}{=} \PY{n}{mean\PYZus{}absolute\PYZus{}error}\PY{p}{(}\PY{n}{data\PYZus{}y\PYZus{}test}\PY{p}{,} \PY{n}{y\PYZus{}pred}\PY{p}{)}
    \PY{n}{mse} \PY{o}{=} \PY{n}{mean\PYZus{}squared\PYZus{}error}\PY{p}{(}\PY{n}{data\PYZus{}y\PYZus{}test}\PY{p}{,} \PY{n}{y\PYZus{}pred}\PY{p}{)}
    \PY{n}{r2} \PY{o}{=} \PY{n}{r2\PYZus{}score}\PY{p}{(}\PY{n}{data\PYZus{}y\PYZus{}test}\PY{p}{,} \PY{n}{y\PYZus{}pred}\PY{p}{)}

    \PY{n}{regrMetricLogger}\PY{o}{.}\PY{n}{add}\PY{p}{(}\PY{l+s+s1}{\PYZsq{}}\PY{l+s+s1}{MAE}\PY{l+s+s1}{\PYZsq{}}\PY{p}{,} \PY{n}{model\PYZus{}name}\PY{p}{,} \PY{n}{mae}\PY{p}{)}
    \PY{n}{regrMetricLogger}\PY{o}{.}\PY{n}{add}\PY{p}{(}\PY{l+s+s1}{\PYZsq{}}\PY{l+s+s1}{MSE}\PY{l+s+s1}{\PYZsq{}}\PY{p}{,} \PY{n}{model\PYZus{}name}\PY{p}{,} \PY{n}{mse}\PY{p}{)}
    \PY{n}{regrMetricLogger}\PY{o}{.}\PY{n}{add}\PY{p}{(}\PY{l+s+s1}{\PYZsq{}}\PY{l+s+s1}{R2}\PY{l+s+s1}{\PYZsq{}}\PY{p}{,} \PY{n}{model\PYZus{}name}\PY{p}{,} \PY{n}{r2}\PY{p}{)}    
    
    \PY{n+nb}{print}\PY{p}{(}\PY{l+s+s1}{\PYZsq{}}\PY{l+s+si}{\PYZob{}\PYZcb{}}\PY{l+s+s1}{ }\PY{l+s+se}{\PYZbs{}t}\PY{l+s+s1}{ MAE=}\PY{l+s+si}{\PYZob{}\PYZcb{}}\PY{l+s+s1}{, MSE=}\PY{l+s+si}{\PYZob{}\PYZcb{}}\PY{l+s+s1}{, R2=}\PY{l+s+si}{\PYZob{}\PYZcb{}}\PY{l+s+s1}{\PYZsq{}}\PY{o}{.}\PY{n}{format}\PY{p}{(}
        \PY{n}{model\PYZus{}name}\PY{p}{,} \PY{n+nb}{round}\PY{p}{(}\PY{n}{mae}\PY{p}{,} \PY{l+m+mi}{3}\PY{p}{)}\PY{p}{,} \PY{n+nb}{round}\PY{p}{(}\PY{n}{mse}\PY{p}{,} \PY{l+m+mi}{3}\PY{p}{)}\PY{p}{,} \PY{n+nb}{round}\PY{p}{(}\PY{n}{r2}\PY{p}{,} \PY{l+m+mi}{3}\PY{p}{)}\PY{p}{)}\PY{p}{)}
\end{Verbatim}
\end{tcolorbox}

    \begin{tcolorbox}[breakable, size=fbox, boxrule=1pt, pad at break*=1mm,colback=cellbackground, colframe=cellborder]
\prompt{In}{incolor}{29}{\boxspacing}
\begin{Verbatim}[commandchars=\\\{\}]
\PY{k}{for} \PY{n}{model\PYZus{}name}\PY{p}{,} \PY{n}{model} \PY{o+ow}{in} \PY{n}{regr\PYZus{}models}\PY{o}{.}\PY{n}{items}\PY{p}{(}\PY{p}{)}\PY{p}{:}
    \PY{n}{regr\PYZus{}train\PYZus{}model}\PY{p}{(}\PY{n}{model\PYZus{}name}\PY{p}{,} \PY{n}{model}\PY{p}{,} \PY{n}{regrMetricLogger}\PY{p}{)}
\end{Verbatim}
\end{tcolorbox}

    \begin{Verbatim}[commandchars=\\\{\}]
LR       MAE=136653.422, MSE=60599347912.745, R2=0.649
KNN\_1    MAE=117071.057, MSE=55480600711.547, R2=0.679
LinearSVR        MAE=514663.235, MSE=435404175784.022, R2=-1.523
Tree     MAE=125079.57, MSE=60957821930.478, R2=0.647
RF       MAE=91576.642, MSE=32389789387.025, R2=0.812
GB       MAE=94569.69, MSE=32432160464.488, R2=0.812
    \end{Verbatim}

\chapter{Подбор гиперпараметров для выбранных моделей. Рекомендуется
использовать методы кросс-валидации. В зависимости от используемой
библиотеки можно применять функцию GridSearchCV, использовать перебор
параметров в цикле, или использовать другие
методы.}

    \begin{tcolorbox}[breakable, size=fbox, boxrule=1pt, pad at break*=1mm,colback=cellbackground, colframe=cellborder]
\prompt{In}{incolor}{31}{\boxspacing}
\begin{Verbatim}[commandchars=\\\{\}]
\PY{n}{kn\PYZus{}n\PYZus{}range} \PY{o}{=} \PY{n}{np}\PY{o}{.}\PY{n}{array}\PY{p}{(}\PY{n+nb}{range}\PY{p}{(}\PY{l+m+mi}{1}\PY{p}{,}\PY{l+m+mi}{15}\PY{p}{,}\PY{l+m+mi}{1}\PY{p}{)}\PY{p}{)}
\PY{n}{kn\PYZus{}tuned\PYZus{}parameters} \PY{o}{=} \PY{p}{[}\PY{p}{\PYZob{}}\PY{l+s+s1}{\PYZsq{}}\PY{l+s+s1}{n\PYZus{}neighbors}\PY{l+s+s1}{\PYZsq{}}\PY{p}{:} \PY{n}{kn\PYZus{}n\PYZus{}range}\PY{p}{\PYZcb{}}\PY{p}{]}
\PY{n}{kn\PYZus{}tuned\PYZus{}parameters}
\end{Verbatim}
\end{tcolorbox}

            \begin{tcolorbox}[breakable, size=fbox, boxrule=.5pt, pad at break*=1mm, opacityfill=0]
\prompt{Out}{outcolor}{31}{\boxspacing}
\begin{Verbatim}[commandchars=\\\{\}]
[\{'n\_neighbors': array([ 1,  2,  3,  4,  5,  6,  7,  8,  9, 10, 11, 12, 13,
14])\}]
\end{Verbatim}
\end{tcolorbox}
        
    \begin{tcolorbox}[breakable, size=fbox, boxrule=1pt, pad at break*=1mm,colback=cellbackground, colframe=cellborder]
\prompt{In}{incolor}{32}{\boxspacing}
\begin{Verbatim}[commandchars=\\\{\}]
\PY{o}{\PYZpc{}\PYZpc{}time}
\PY{n}{gs\PYZus{}kn} \PY{o}{=} \PY{n}{GridSearchCV}\PY{p}{(}\PY{n}{KNeighborsRegressor}\PY{p}{(}\PY{p}{)}\PY{p}{,} \PY{n}{kn\PYZus{}tuned\PYZus{}parameters}\PY{p}{,} \PY{n}{cv}\PY{o}{=}\PY{l+m+mi}{5}\PY{p}{,} \PY{n}{scoring}\PY{o}{=}\PY{l+s+s1}{\PYZsq{}}\PY{l+s+s1}{neg\PYZus{}mean\PYZus{}squared\PYZus{}error}\PY{l+s+s1}{\PYZsq{}}\PY{p}{)}
\PY{n}{gs\PYZus{}kn}\PY{o}{.}\PY{n}{fit}\PY{p}{(}\PY{n}{data\PYZus{}X\PYZus{}train}\PY{p}{,} \PY{n}{data\PYZus{}y\PYZus{}train}\PY{p}{)}
\end{Verbatim}
\end{tcolorbox}

    \begin{Verbatim}[commandchars=\\\{\}]
CPU times: user 49.7 s, sys: 4.3 ms, total: 49.7 s
Wall time: 49.8 s
    \end{Verbatim}

            \begin{tcolorbox}[breakable, size=fbox, boxrule=.5pt, pad at break*=1mm, opacityfill=0]
\prompt{Out}{outcolor}{32}{\boxspacing}
\begin{Verbatim}[commandchars=\\\{\}]
GridSearchCV(cv=5, estimator=KNeighborsRegressor(),
             param\_grid=[\{'n\_neighbors': array([ 1,  2,  3,  4,  5,  6,  7,  8,
9, 10, 11, 12, 13, 14])\}],
             scoring='neg\_mean\_squared\_error')
\end{Verbatim}
\end{tcolorbox}
        
    \begin{tcolorbox}[breakable, size=fbox, boxrule=1pt, pad at break*=1mm,colback=cellbackground, colframe=cellborder]
\prompt{In}{incolor}{33}{\boxspacing}
\begin{Verbatim}[commandchars=\\\{\}]
\PY{c+c1}{\PYZsh{} Лучшая модель}
\PY{n}{gs\PYZus{}kn}\PY{o}{.}\PY{n}{best\PYZus{}estimator\PYZus{}}
\end{Verbatim}
\end{tcolorbox}

            \begin{tcolorbox}[breakable, size=fbox, boxrule=.5pt, pad at break*=1mm, opacityfill=0]
\prompt{Out}{outcolor}{33}{\boxspacing}
\begin{Verbatim}[commandchars=\\\{\}]
KNeighborsRegressor(n\_neighbors=10)
\end{Verbatim}
\end{tcolorbox}
        
    \begin{tcolorbox}[breakable, size=fbox, boxrule=1pt, pad at break*=1mm,colback=cellbackground, colframe=cellborder]
\prompt{In}{incolor}{37}{\boxspacing}
\begin{Verbatim}[commandchars=\\\{\}]
\PY{c+c1}{\PYZsh{} Изменение качества на тестовой выборке в зависимости от К\PYZhy{}соседей}
\PY{n}{plt}\PY{o}{.}\PY{n}{plot}\PY{p}{(}\PY{n}{kn\PYZus{}n\PYZus{}range}\PY{p}{,} \PY{n}{gs\PYZus{}kn}\PY{o}{.}\PY{n}{cv\PYZus{}results\PYZus{}}\PY{p}{[}\PY{l+s+s1}{\PYZsq{}}\PY{l+s+s1}{mean\PYZus{}test\PYZus{}score}\PY{l+s+s1}{\PYZsq{}}\PY{p}{]}\PY{p}{)}
\end{Verbatim}
\end{tcolorbox}

            \begin{tcolorbox}[breakable, size=fbox, boxrule=.5pt, pad at break*=1mm, opacityfill=0]
\prompt{Out}{outcolor}{37}{\boxspacing}
\begin{Verbatim}[commandchars=\\\{\}]
[<matplotlib.lines.Line2D at 0x7efcc11ffcd0>]
\end{Verbatim}
\end{tcolorbox}
        
    \begin{center}
    \adjustimage{max size={0.9\linewidth}{0.9\paperheight}}{output_57_1.png}
    \end{center}
    { \hspace*{\fill} \\}
    
    \begin{tcolorbox}[breakable, size=fbox, boxrule=1pt, pad at break*=1mm,colback=cellbackground, colframe=cellborder]
\prompt{In}{incolor}{43}{\boxspacing}
\begin{Verbatim}[commandchars=\\\{\}]
\PY{n}{lsvr\PYZus{}c\PYZus{}range} \PY{o}{=}  \PY{n}{np}\PY{o}{.}\PY{n}{array}\PY{p}{(}\PY{n+nb}{range}\PY{p}{(}\PY{l+m+mi}{1}\PY{p}{,}\PY{l+m+mi}{100000}\PY{p}{,} \PY{l+m+mi}{10000}\PY{p}{)}\PY{p}{)}
\PY{n}{lsvr\PYZus{}tuned\PYZus{}parameters} \PY{o}{=} \PY{p}{[}\PY{p}{\PYZob{}}\PY{l+s+s1}{\PYZsq{}}\PY{l+s+s1}{C}\PY{l+s+s1}{\PYZsq{}}\PY{p}{:} \PY{n}{lsvr\PYZus{}c\PYZus{}range}\PY{p}{\PYZcb{}}\PY{p}{]}
\PY{n}{lsvr\PYZus{}tuned\PYZus{}parameters}
\end{Verbatim}
\end{tcolorbox}

            \begin{tcolorbox}[breakable, size=fbox, boxrule=.5pt, pad at break*=1mm, opacityfill=0]
\prompt{Out}{outcolor}{43}{\boxspacing}
\begin{Verbatim}[commandchars=\\\{\}]
[\{'C': array([    1, 10001, 20001, 30001, 40001, 50001, 60001, 70001, 80001,
         90001])\}]
\end{Verbatim}
\end{tcolorbox}
        
    \begin{tcolorbox}[breakable, size=fbox, boxrule=1pt, pad at break*=1mm,colback=cellbackground, colframe=cellborder]
\prompt{In}{incolor}{39}{\boxspacing}
\begin{Verbatim}[commandchars=\\\{\}]
\PY{o}{\PYZpc{}\PYZpc{}time}
\PY{n}{gs\PYZus{}lsvr} \PY{o}{=} \PY{n}{GridSearchCV}\PY{p}{(}\PY{n}{LinearSVR}\PY{p}{(}\PY{p}{)}\PY{p}{,} \PY{n}{lsvr\PYZus{}tuned\PYZus{}parameters}\PY{p}{,} \PY{n}{cv}\PY{o}{=}\PY{l+m+mi}{5}\PY{p}{,} \PY{n}{scoring}\PY{o}{=}\PY{l+s+s1}{\PYZsq{}}\PY{l+s+s1}{neg\PYZus{}mean\PYZus{}squared\PYZus{}error}\PY{l+s+s1}{\PYZsq{}}\PY{p}{)}
\PY{n}{gs\PYZus{}lsvr}\PY{o}{.}\PY{n}{fit}\PY{p}{(}\PY{n}{data\PYZus{}X\PYZus{}train}\PY{p}{,} \PY{n}{data\PYZus{}y\PYZus{}train}\PY{p}{)}
\end{Verbatim}
\end{tcolorbox}

    \begin{Verbatim}[commandchars=\\\{\}]
CPU times: user 7.12 s, sys: 5.25 s, total: 12.4 s
Wall time: 3.48 s
    \end{Verbatim}

            \begin{tcolorbox}[breakable, size=fbox, boxrule=.5pt, pad at break*=1mm, opacityfill=0]
\prompt{Out}{outcolor}{39}{\boxspacing}
\begin{Verbatim}[commandchars=\\\{\}]
GridSearchCV(cv=5, estimator=LinearSVR(),
             param\_grid=[\{'C': array([    1, 10001, 20001, 30001, 40001, 50001,
60001, 70001, 80001,
       90001])\}],
             scoring='neg\_mean\_squared\_error')
\end{Verbatim}
\end{tcolorbox}
        
    \begin{tcolorbox}[breakable, size=fbox, boxrule=1pt, pad at break*=1mm,colback=cellbackground, colframe=cellborder]
\prompt{In}{incolor}{40}{\boxspacing}
\begin{Verbatim}[commandchars=\\\{\}]
\PY{c+c1}{\PYZsh{} Лучшая модель}
\PY{n}{gs\PYZus{}lsvr}\PY{o}{.}\PY{n}{best\PYZus{}estimator\PYZus{}}
\end{Verbatim}
\end{tcolorbox}

            \begin{tcolorbox}[breakable, size=fbox, boxrule=.5pt, pad at break*=1mm, opacityfill=0]
\prompt{Out}{outcolor}{40}{\boxspacing}
\begin{Verbatim}[commandchars=\\\{\}]
LinearSVR(C=90001)
\end{Verbatim}
\end{tcolorbox}
        
    \begin{tcolorbox}[breakable, size=fbox, boxrule=1pt, pad at break*=1mm,colback=cellbackground, colframe=cellborder]
\prompt{In}{incolor}{45}{\boxspacing}
\begin{Verbatim}[commandchars=\\\{\}]
\PY{c+c1}{\PYZsh{} Изменение качества на тестовой выборке в зависимости от C}
\PY{n}{plt}\PY{o}{.}\PY{n}{plot}\PY{p}{(}\PY{n}{lsvr\PYZus{}c\PYZus{}range}\PY{p}{,} \PY{n}{gs\PYZus{}lsvr}\PY{o}{.}\PY{n}{cv\PYZus{}results\PYZus{}}\PY{p}{[}\PY{l+s+s1}{\PYZsq{}}\PY{l+s+s1}{mean\PYZus{}test\PYZus{}score}\PY{l+s+s1}{\PYZsq{}}\PY{p}{]}\PY{p}{)}
\end{Verbatim}
\end{tcolorbox}

            \begin{tcolorbox}[breakable, size=fbox, boxrule=.5pt, pad at break*=1mm, opacityfill=0]
\prompt{Out}{outcolor}{45}{\boxspacing}
\begin{Verbatim}[commandchars=\\\{\}]
[<matplotlib.lines.Line2D at 0x7efcc03a24f0>]
\end{Verbatim}
\end{tcolorbox}
        
    \begin{center}
    \adjustimage{max size={0.9\linewidth}{0.9\paperheight}}{output_61_1.png}
    \end{center}
    { \hspace*{\fill} \\}
    
    \begin{tcolorbox}[breakable, size=fbox, boxrule=1pt, pad at break*=1mm,colback=cellbackground, colframe=cellborder]
\prompt{In}{incolor}{87}{\boxspacing}
\begin{Verbatim}[commandchars=\\\{\}]
\PY{n}{tree\PYZus{}depth\PYZus{}range} \PY{o}{=} \PY{n}{np}\PY{o}{.}\PY{n}{array}\PY{p}{(}\PY{n+nb}{range}\PY{p}{(}\PY{l+m+mi}{3}\PY{p}{,}\PY{l+m+mi}{11}\PY{p}{,}\PY{l+m+mi}{1}\PY{p}{)}\PY{p}{)}
\PY{n}{tree\PYZus{}tuned\PYZus{}parameters} \PY{o}{=} \PY{p}{[}\PY{p}{\PYZob{}}\PY{l+s+s1}{\PYZsq{}}\PY{l+s+s1}{max\PYZus{}depth}\PY{l+s+s1}{\PYZsq{}}\PY{p}{:} \PY{n}{tree\PYZus{}depth\PYZus{}range}\PY{p}{\PYZcb{}}\PY{p}{]}
\PY{n}{tree\PYZus{}tuned\PYZus{}parameters}
\end{Verbatim}
\end{tcolorbox}

            \begin{tcolorbox}[breakable, size=fbox, boxrule=.5pt, pad at break*=1mm, opacityfill=0]
\prompt{Out}{outcolor}{87}{\boxspacing}
\begin{Verbatim}[commandchars=\\\{\}]
[\{'max\_depth': array([ 3,  4,  5,  6,  7,  8,  9, 10])\}]
\end{Verbatim}
\end{tcolorbox}
        
    \begin{tcolorbox}[breakable, size=fbox, boxrule=1pt, pad at break*=1mm,colback=cellbackground, colframe=cellborder]
\prompt{In}{incolor}{88}{\boxspacing}
\begin{Verbatim}[commandchars=\\\{\}]
\PY{o}{\PYZpc{}\PYZpc{}time}
\PY{n}{gs\PYZus{}tree} \PY{o}{=} \PY{n}{GridSearchCV}\PY{p}{(}\PY{n}{DecisionTreeRegressor}\PY{p}{(}\PY{p}{)}\PY{p}{,} \PY{n}{tree\PYZus{}tuned\PYZus{}parameters}\PY{p}{,} \PY{n}{cv}\PY{o}{=}\PY{l+m+mi}{5}\PY{p}{,} \PY{n}{scoring}\PY{o}{=}\PY{l+s+s1}{\PYZsq{}}\PY{l+s+s1}{neg\PYZus{}mean\PYZus{}squared\PYZus{}error}\PY{l+s+s1}{\PYZsq{}}\PY{p}{)}
\PY{n}{gs\PYZus{}tree}\PY{o}{.}\PY{n}{fit}\PY{p}{(}\PY{n}{data\PYZus{}X\PYZus{}train}\PY{p}{,} \PY{n}{data\PYZus{}y\PYZus{}train}\PY{p}{)}
\end{Verbatim}
\end{tcolorbox}

    \begin{Verbatim}[commandchars=\\\{\}]
CPU times: user 2.05 s, sys: 21 µs, total: 2.05 s
Wall time: 2.09 s
    \end{Verbatim}

            \begin{tcolorbox}[breakable, size=fbox, boxrule=.5pt, pad at break*=1mm, opacityfill=0]
\prompt{Out}{outcolor}{88}{\boxspacing}
\begin{Verbatim}[commandchars=\\\{\}]
GridSearchCV(cv=5, estimator=DecisionTreeRegressor(),
             param\_grid=[\{'max\_depth': array([ 3,  4,  5,  6,  7,  8,  9,
10])\}],
             scoring='neg\_mean\_squared\_error')
\end{Verbatim}
\end{tcolorbox}
        
    \begin{tcolorbox}[breakable, size=fbox, boxrule=1pt, pad at break*=1mm,colback=cellbackground, colframe=cellborder]
\prompt{In}{incolor}{49}{\boxspacing}
\begin{Verbatim}[commandchars=\\\{\}]
\PY{c+c1}{\PYZsh{} Лучшая модель}
\PY{n}{gs\PYZus{}tree}\PY{o}{.}\PY{n}{best\PYZus{}estimator\PYZus{}}
\end{Verbatim}
\end{tcolorbox}

            \begin{tcolorbox}[breakable, size=fbox, boxrule=.5pt, pad at break*=1mm, opacityfill=0]
\prompt{Out}{outcolor}{49}{\boxspacing}
\begin{Verbatim}[commandchars=\\\{\}]
DecisionTreeRegressor(max\_depth=9)
\end{Verbatim}
\end{tcolorbox}
        
    \begin{tcolorbox}[breakable, size=fbox, boxrule=1pt, pad at break*=1mm,colback=cellbackground, colframe=cellborder]
\prompt{In}{incolor}{50}{\boxspacing}
\begin{Verbatim}[commandchars=\\\{\}]
\PY{c+c1}{\PYZsh{} Изменение качества на тестовой выборке в зависимости от глубины дерева}
\PY{n}{plt}\PY{o}{.}\PY{n}{plot}\PY{p}{(}\PY{n}{tree\PYZus{}depth\PYZus{}range}\PY{p}{,} \PY{n}{gs\PYZus{}tree}\PY{o}{.}\PY{n}{cv\PYZus{}results\PYZus{}}\PY{p}{[}\PY{l+s+s1}{\PYZsq{}}\PY{l+s+s1}{mean\PYZus{}test\PYZus{}score}\PY{l+s+s1}{\PYZsq{}}\PY{p}{]}\PY{p}{)}
\end{Verbatim}
\end{tcolorbox}

            \begin{tcolorbox}[breakable, size=fbox, boxrule=.5pt, pad at break*=1mm, opacityfill=0]
\prompt{Out}{outcolor}{50}{\boxspacing}
\begin{Verbatim}[commandchars=\\\{\}]
[<matplotlib.lines.Line2D at 0x7efcbfa012b0>]
\end{Verbatim}
\end{tcolorbox}
        
    \begin{center}
    \adjustimage{max size={0.9\linewidth}{0.9\paperheight}}{output_65_1.png}
    \end{center}
    { \hspace*{\fill} \\}
    
    \begin{tcolorbox}[breakable, size=fbox, boxrule=1pt, pad at break*=1mm,colback=cellbackground, colframe=cellborder]
\prompt{In}{incolor}{73}{\boxspacing}
\begin{Verbatim}[commandchars=\\\{\}]
\PY{n}{rf\PYZus{}n\PYZus{}range} \PY{o}{=} \PY{n}{np}\PY{o}{.}\PY{n}{array}\PY{p}{(}\PY{n+nb}{range}\PY{p}{(}\PY{l+m+mi}{200}\PY{p}{,}\PY{l+m+mi}{500}\PY{p}{,}\PY{l+m+mi}{100}\PY{p}{)}\PY{p}{)}
\PY{n}{rf\PYZus{}tuned\PYZus{}parameters} \PY{o}{=} \PY{p}{[}\PY{p}{\PYZob{}}\PY{l+s+s1}{\PYZsq{}}\PY{l+s+s1}{n\PYZus{}estimators}\PY{l+s+s1}{\PYZsq{}}\PY{p}{:} \PY{n}{rf\PYZus{}n\PYZus{}range}\PY{p}{\PYZcb{}}\PY{p}{]}
\PY{n}{rf\PYZus{}tuned\PYZus{}parameters}
\end{Verbatim}
\end{tcolorbox}

            \begin{tcolorbox}[breakable, size=fbox, boxrule=.5pt, pad at break*=1mm, opacityfill=0]
\prompt{Out}{outcolor}{73}{\boxspacing}
\begin{Verbatim}[commandchars=\\\{\}]
[\{'n\_estimators': array([200, 300, 400])\}]
\end{Verbatim}
\end{tcolorbox}
        
    \begin{tcolorbox}[breakable, size=fbox, boxrule=1pt, pad at break*=1mm,colback=cellbackground, colframe=cellborder]
\prompt{In}{incolor}{80}{\boxspacing}
\begin{Verbatim}[commandchars=\\\{\}]
\PY{o}{\PYZpc{}\PYZpc{}time}
\PY{n}{gs\PYZus{}rf} \PY{o}{=} \PY{n}{GridSearchCV}\PY{p}{(}\PY{n}{RandomForestRegressor}\PY{p}{(}\PY{p}{)}\PY{p}{,} \PY{n}{rf\PYZus{}tuned\PYZus{}parameters}\PY{p}{,} \PY{n}{cv}\PY{o}{=}\PY{l+m+mi}{5}\PY{p}{,} \PY{n}{scoring}\PY{o}{=}\PY{l+s+s1}{\PYZsq{}}\PY{l+s+s1}{neg\PYZus{}mean\PYZus{}squared\PYZus{}error}\PY{l+s+s1}{\PYZsq{}}\PY{p}{,} \PY{n}{n\PYZus{}jobs}\PY{o}{=}\PY{o}{\PYZhy{}}\PY{l+m+mi}{1}\PY{p}{)}
\PY{n}{gs\PYZus{}rf}\PY{o}{.}\PY{n}{fit}\PY{p}{(}\PY{n}{data\PYZus{}X\PYZus{}train}\PY{p}{,} \PY{n}{data\PYZus{}y\PYZus{}train}\PY{p}{)}
\end{Verbatim}
\end{tcolorbox}

    \begin{Verbatim}[commandchars=\\\{\}]
CPU times: user 19.2 s, sys: 251 ms, total: 19.4 s
Wall time: 2min 49s
    \end{Verbatim}

            \begin{tcolorbox}[breakable, size=fbox, boxrule=.5pt, pad at break*=1mm, opacityfill=0]
\prompt{Out}{outcolor}{80}{\boxspacing}
\begin{Verbatim}[commandchars=\\\{\}]
GridSearchCV(cv=5, estimator=RandomForestRegressor(), n\_jobs=-1,
             param\_grid=[\{'n\_estimators': array([200, 300, 400])\}],
             scoring='neg\_mean\_squared\_error')
\end{Verbatim}
\end{tcolorbox}
        
    \begin{tcolorbox}[breakable, size=fbox, boxrule=1pt, pad at break*=1mm,colback=cellbackground, colframe=cellborder]
\prompt{In}{incolor}{81}{\boxspacing}
\begin{Verbatim}[commandchars=\\\{\}]
\PY{c+c1}{\PYZsh{} Лучшая модель}
\PY{n}{gs\PYZus{}rf}\PY{o}{.}\PY{n}{best\PYZus{}estimator\PYZus{}}
\end{Verbatim}
\end{tcolorbox}

            \begin{tcolorbox}[breakable, size=fbox, boxrule=.5pt, pad at break*=1mm, opacityfill=0]
\prompt{Out}{outcolor}{81}{\boxspacing}
\begin{Verbatim}[commandchars=\\\{\}]
RandomForestRegressor(n\_estimators=200)
\end{Verbatim}
\end{tcolorbox}
        
    \begin{tcolorbox}[breakable, size=fbox, boxrule=1pt, pad at break*=1mm,colback=cellbackground, colframe=cellborder]
\prompt{In}{incolor}{82}{\boxspacing}
\begin{Verbatim}[commandchars=\\\{\}]
\PY{c+c1}{\PYZsh{} Изменение качества на тестовой выборке в зависимости от n\PYZus{}estimators}
\PY{n}{plt}\PY{o}{.}\PY{n}{plot}\PY{p}{(}\PY{n}{rf\PYZus{}n\PYZus{}range}\PY{p}{,} \PY{n}{gs\PYZus{}rf}\PY{o}{.}\PY{n}{cv\PYZus{}results\PYZus{}}\PY{p}{[}\PY{l+s+s1}{\PYZsq{}}\PY{l+s+s1}{mean\PYZus{}test\PYZus{}score}\PY{l+s+s1}{\PYZsq{}}\PY{p}{]}\PY{p}{)}
\end{Verbatim}
\end{tcolorbox}

            \begin{tcolorbox}[breakable, size=fbox, boxrule=.5pt, pad at break*=1mm, opacityfill=0]
\prompt{Out}{outcolor}{82}{\boxspacing}
\begin{Verbatim}[commandchars=\\\{\}]
[<matplotlib.lines.Line2D at 0x7efcc88bf910>]
\end{Verbatim}
\end{tcolorbox}
        
    \begin{center}
    \adjustimage{max size={0.9\linewidth}{0.9\paperheight}}{output_69_1.png}
    \end{center}
    { \hspace*{\fill} \\}
    
    \begin{tcolorbox}[breakable, size=fbox, boxrule=1pt, pad at break*=1mm,colback=cellbackground, colframe=cellborder]
\prompt{In}{incolor}{83}{\boxspacing}
\begin{Verbatim}[commandchars=\\\{\}]
\PY{o}{\PYZpc{}\PYZpc{}time}
\PY{n}{gs\PYZus{}gb} \PY{o}{=} \PY{n}{GridSearchCV}\PY{p}{(}\PY{n}{GradientBoostingRegressor}\PY{p}{(}\PY{p}{)}\PY{p}{,} \PY{n}{rf\PYZus{}tuned\PYZus{}parameters}\PY{p}{,} \PY{n}{cv}\PY{o}{=}\PY{l+m+mi}{5}\PY{p}{,} \PY{n}{scoring}\PY{o}{=}\PY{l+s+s1}{\PYZsq{}}\PY{l+s+s1}{neg\PYZus{}mean\PYZus{}squared\PYZus{}error}\PY{l+s+s1}{\PYZsq{}}\PY{p}{,} \PY{n}{n\PYZus{}jobs}\PY{o}{=}\PY{o}{\PYZhy{}}\PY{l+m+mi}{1}\PY{p}{)}
\PY{n}{gs\PYZus{}gb}\PY{o}{.}\PY{n}{fit}\PY{p}{(}\PY{n}{data\PYZus{}X\PYZus{}train}\PY{p}{,} \PY{n}{data\PYZus{}y\PYZus{}train}\PY{p}{)}
\end{Verbatim}
\end{tcolorbox}

    \begin{Verbatim}[commandchars=\\\{\}]
CPU times: user 9.77 s, sys: 6.56 ms, total: 9.78 s
Wall time: 45.2 s
    \end{Verbatim}

            \begin{tcolorbox}[breakable, size=fbox, boxrule=.5pt, pad at break*=1mm, opacityfill=0]
\prompt{Out}{outcolor}{83}{\boxspacing}
\begin{Verbatim}[commandchars=\\\{\}]
GridSearchCV(cv=5, estimator=GradientBoostingRegressor(), n\_jobs=-1,
             param\_grid=[\{'n\_estimators': array([200, 300, 400])\}],
             scoring='neg\_mean\_squared\_error')
\end{Verbatim}
\end{tcolorbox}
        
    \begin{tcolorbox}[breakable, size=fbox, boxrule=1pt, pad at break*=1mm,colback=cellbackground, colframe=cellborder]
\prompt{In}{incolor}{84}{\boxspacing}
\begin{Verbatim}[commandchars=\\\{\}]
\PY{c+c1}{\PYZsh{} Лучшая модель}
\PY{n}{gs\PYZus{}gb}\PY{o}{.}\PY{n}{best\PYZus{}estimator\PYZus{}}
\end{Verbatim}
\end{tcolorbox}

            \begin{tcolorbox}[breakable, size=fbox, boxrule=.5pt, pad at break*=1mm, opacityfill=0]
\prompt{Out}{outcolor}{84}{\boxspacing}
\begin{Verbatim}[commandchars=\\\{\}]
GradientBoostingRegressor(n\_estimators=400)
\end{Verbatim}
\end{tcolorbox}
        
    \begin{tcolorbox}[breakable, size=fbox, boxrule=1pt, pad at break*=1mm,colback=cellbackground, colframe=cellborder]
\prompt{In}{incolor}{85}{\boxspacing}
\begin{Verbatim}[commandchars=\\\{\}]
\PY{c+c1}{\PYZsh{} Изменение качества на тестовой выборке в зависимости от n\PYZus{}estimators}
\PY{n}{plt}\PY{o}{.}\PY{n}{plot}\PY{p}{(}\PY{n}{rf\PYZus{}n\PYZus{}range}\PY{p}{,} \PY{n}{gs\PYZus{}gb}\PY{o}{.}\PY{n}{cv\PYZus{}results\PYZus{}}\PY{p}{[}\PY{l+s+s1}{\PYZsq{}}\PY{l+s+s1}{mean\PYZus{}test\PYZus{}score}\PY{l+s+s1}{\PYZsq{}}\PY{p}{]}\PY{p}{)}
\end{Verbatim}
\end{tcolorbox}

            \begin{tcolorbox}[breakable, size=fbox, boxrule=.5pt, pad at break*=1mm, opacityfill=0]
\prompt{Out}{outcolor}{85}{\boxspacing}
\begin{Verbatim}[commandchars=\\\{\}]
[<matplotlib.lines.Line2D at 0x7efcc8db0820>]
\end{Verbatim}
\end{tcolorbox}
        
    \begin{center}
    \adjustimage{max size={0.9\linewidth}{0.9\paperheight}}{output_72_1.png}
    \end{center}
    { \hspace*{\fill} \\}
    
\chapter{Повторение пункта 8 для найденных оптимальных значений
гиперпараметров. Сравнение качества полученных моделей с качеством
baseline-моделей.}

    \begin{tcolorbox}[breakable, size=fbox, boxrule=1pt, pad at break*=1mm,colback=cellbackground, colframe=cellborder]
\prompt{In}{incolor}{89}{\boxspacing}
\begin{Verbatim}[commandchars=\\\{\}]
\PY{n}{regr\PYZus{}models\PYZus{}grid} \PY{o}{=} \PY{p}{\PYZob{}} 
    \PY{l+s+s1}{\PYZsq{}}\PY{l+s+s1}{KNN(best)}\PY{l+s+s1}{\PYZsq{}}\PY{p}{:}\PY{n}{gs\PYZus{}kn}\PY{o}{.}\PY{n}{best\PYZus{}estimator\PYZus{}}\PY{p}{,}
    \PY{l+s+s1}{\PYZsq{}}\PY{l+s+s1}{Tree(best)}\PY{l+s+s1}{\PYZsq{}}\PY{p}{:}\PY{n}{gs\PYZus{}tree}\PY{o}{.}\PY{n}{best\PYZus{}estimator\PYZus{}}\PY{p}{,}
    \PY{l+s+s1}{\PYZsq{}}\PY{l+s+s1}{LinearSVR(best)}\PY{l+s+s1}{\PYZsq{}}\PY{p}{:}\PY{n}{gs\PYZus{}lsvr}\PY{o}{.}\PY{n}{best\PYZus{}estimator\PYZus{}}\PY{p}{,}
    \PY{l+s+s1}{\PYZsq{}}\PY{l+s+s1}{RF(best)}\PY{l+s+s1}{\PYZsq{}}\PY{p}{:}\PY{n}{gs\PYZus{}rf}\PY{o}{.}\PY{n}{best\PYZus{}estimator\PYZus{}}\PY{p}{,}
    \PY{l+s+s1}{\PYZsq{}}\PY{l+s+s1}{GB(best)}\PY{l+s+s1}{\PYZsq{}}\PY{p}{:}\PY{n}{gs\PYZus{}gb}\PY{o}{.}\PY{n}{best\PYZus{}estimator\PYZus{}}
                   \PY{p}{\PYZcb{}}
\end{Verbatim}
\end{tcolorbox}

    \begin{tcolorbox}[breakable, size=fbox, boxrule=1pt, pad at break*=1mm,colback=cellbackground, colframe=cellborder]
\prompt{In}{incolor}{90}{\boxspacing}
\begin{Verbatim}[commandchars=\\\{\}]
\PY{k}{for} \PY{n}{model\PYZus{}name}\PY{p}{,} \PY{n}{model} \PY{o+ow}{in} \PY{n}{regr\PYZus{}models\PYZus{}grid}\PY{o}{.}\PY{n}{items}\PY{p}{(}\PY{p}{)}\PY{p}{:}
    \PY{n}{regr\PYZus{}train\PYZus{}model}\PY{p}{(}\PY{n}{model\PYZus{}name}\PY{p}{,} \PY{n}{model}\PY{p}{,} \PY{n}{regrMetricLogger}\PY{p}{)}
\end{Verbatim}
\end{tcolorbox}

    \begin{Verbatim}[commandchars=\\\{\}]
KNN(best)        MAE=96657.319, MSE=47779449527.494, R2=0.723
Tree(best)       MAE=108504.681, MSE=42722488692.848, R2=0.752
LinearSVR(best)          MAE=129267.13, MSE=71875871300.374, R2=0.584
RF(best)         MAE=91524.206, MSE=33475421722.717, R2=0.806
GB(best)         MAE=92487.687, MSE=30236231505.354, R2=0.825
    \end{Verbatim}

\chapter{Формирование выводов о качестве построенных моделей на
основе выбранных метрик. Результаты сравнения качества рекомендуется
отобразить в виде графиков и сделать выводы в форме текстового описания.
Рекомендуется построение графиков обучения и валидации, влияния значений
гиперпарметров на качество моделей и
т.д.}

    \begin{tcolorbox}[breakable, size=fbox, boxrule=1pt, pad at break*=1mm,colback=cellbackground, colframe=cellborder]
\prompt{In}{incolor}{91}{\boxspacing}
\begin{Verbatim}[commandchars=\\\{\}]
\PY{c+c1}{\PYZsh{} Метрики качества модели}
\PY{n}{regr\PYZus{}metrics} \PY{o}{=} \PY{n}{regrMetricLogger}\PY{o}{.}\PY{n}{df}\PY{p}{[}\PY{l+s+s1}{\PYZsq{}}\PY{l+s+s1}{metric}\PY{l+s+s1}{\PYZsq{}}\PY{p}{]}\PY{o}{.}\PY{n}{unique}\PY{p}{(}\PY{p}{)}
\PY{n}{regr\PYZus{}metrics}
\end{Verbatim}
\end{tcolorbox}

            \begin{tcolorbox}[breakable, size=fbox, boxrule=.5pt, pad at break*=1mm, opacityfill=0]
\prompt{Out}{outcolor}{91}{\boxspacing}
\begin{Verbatim}[commandchars=\\\{\}]
array(['MAE', 'MSE', 'R2'], dtype=object)
\end{Verbatim}
\end{tcolorbox}
        
    \begin{tcolorbox}[breakable, size=fbox, boxrule=1pt, pad at break*=1mm,colback=cellbackground, colframe=cellborder]
\prompt{In}{incolor}{92}{\boxspacing}
\begin{Verbatim}[commandchars=\\\{\}]
\PY{n}{regrMetricLogger}\PY{o}{.}\PY{n}{plot}\PY{p}{(}\PY{l+s+s1}{\PYZsq{}}\PY{l+s+s1}{Метрика: }\PY{l+s+s1}{\PYZsq{}} \PY{o}{+} \PY{l+s+s1}{\PYZsq{}}\PY{l+s+s1}{MAE}\PY{l+s+s1}{\PYZsq{}}\PY{p}{,} \PY{l+s+s1}{\PYZsq{}}\PY{l+s+s1}{MAE}\PY{l+s+s1}{\PYZsq{}}\PY{p}{,} \PY{n}{ascending}\PY{o}{=}\PY{k+kc}{False}\PY{p}{,} \PY{n}{figsize}\PY{o}{=}\PY{p}{(}\PY{l+m+mi}{7}\PY{p}{,} \PY{l+m+mi}{6}\PY{p}{)}\PY{p}{)}
\end{Verbatim}
\end{tcolorbox}

    \begin{center}
    \adjustimage{max size={0.9\linewidth}{0.9\paperheight}}{output_78_0.png}
    \end{center}
    { \hspace*{\fill} \\}
    
    \begin{tcolorbox}[breakable, size=fbox, boxrule=1pt, pad at break*=1mm,colback=cellbackground, colframe=cellborder]
\prompt{In}{incolor}{93}{\boxspacing}
\begin{Verbatim}[commandchars=\\\{\}]
\PY{n}{regrMetricLogger}\PY{o}{.}\PY{n}{plot}\PY{p}{(}\PY{l+s+s1}{\PYZsq{}}\PY{l+s+s1}{Метрика: }\PY{l+s+s1}{\PYZsq{}} \PY{o}{+} \PY{l+s+s1}{\PYZsq{}}\PY{l+s+s1}{MSE}\PY{l+s+s1}{\PYZsq{}}\PY{p}{,} \PY{l+s+s1}{\PYZsq{}}\PY{l+s+s1}{MSE}\PY{l+s+s1}{\PYZsq{}}\PY{p}{,} \PY{n}{ascending}\PY{o}{=}\PY{k+kc}{False}\PY{p}{,} \PY{n}{figsize}\PY{o}{=}\PY{p}{(}\PY{l+m+mi}{7}\PY{p}{,} \PY{l+m+mi}{6}\PY{p}{)}\PY{p}{)}
\end{Verbatim}
\end{tcolorbox}

    \begin{center}
    \adjustimage{max size={0.9\linewidth}{0.9\paperheight}}{output_79_0.png}
    \end{center}
    { \hspace*{\fill} \\}
    
    \begin{tcolorbox}[breakable, size=fbox, boxrule=1pt, pad at break*=1mm,colback=cellbackground, colframe=cellborder]
\prompt{In}{incolor}{94}{\boxspacing}
\begin{Verbatim}[commandchars=\\\{\}]
\PY{n}{regrMetricLogger}\PY{o}{.}\PY{n}{plot}\PY{p}{(}\PY{l+s+s1}{\PYZsq{}}\PY{l+s+s1}{Метрика: }\PY{l+s+s1}{\PYZsq{}} \PY{o}{+} \PY{l+s+s1}{\PYZsq{}}\PY{l+s+s1}{R2}\PY{l+s+s1}{\PYZsq{}}\PY{p}{,} \PY{l+s+s1}{\PYZsq{}}\PY{l+s+s1}{R2}\PY{l+s+s1}{\PYZsq{}}\PY{p}{,} \PY{n}{ascending}\PY{o}{=}\PY{k+kc}{True}\PY{p}{,} \PY{n}{figsize}\PY{o}{=}\PY{p}{(}\PY{l+m+mi}{7}\PY{p}{,} \PY{l+m+mi}{6}\PY{p}{)}\PY{p}{)}
\end{Verbatim}
\end{tcolorbox}

    \begin{center}
    \adjustimage{max size={0.9\linewidth}{0.9\paperheight}}{output_80_0.png}
    \end{center}
    { \hspace*{\fill} \\}
    
\chapter{AutoML}

    \begin{tcolorbox}[breakable, size=fbox, boxrule=1pt, pad at break*=1mm,colback=cellbackground, colframe=cellborder]
\prompt{In}{incolor}{76}{\boxspacing}
\begin{Verbatim}[commandchars=\\\{\}]
\PY{k+kn}{import} \PY{n+nn}{autosklearn}\PY{n+nn}{.}\PY{n+nn}{regression}
\end{Verbatim}
\end{tcolorbox}

    \begin{tcolorbox}[breakable, size=fbox, boxrule=1pt, pad at break*=1mm,colback=cellbackground, colframe=cellborder]
\prompt{In}{incolor}{78}{\boxspacing}
\begin{Verbatim}[commandchars=\\\{\}]
\PY{n}{automl} \PY{o}{=} \PY{n}{autosklearn}\PY{o}{.}\PY{n}{regression}\PY{o}{.}\PY{n}{AutoSklearnRegressor}\PY{p}{(}\PY{n}{time\PYZus{}left\PYZus{}for\PYZus{}this\PYZus{}task}\PY{o}{=}\PY{l+m+mi}{120}\PY{p}{,} \PY{n}{per\PYZus{}run\PYZus{}time\PYZus{}limit}\PY{o}{=}\PY{l+m+mi}{30}\PY{p}{)}
\PY{n}{automl}\PY{o}{.}\PY{n}{fit}\PY{p}{(}\PY{n}{data\PYZus{}X\PYZus{}train}\PY{p}{,} \PY{n}{data\PYZus{}y\PYZus{}train}\PY{p}{)}
\end{Verbatim}
\end{tcolorbox}

            \begin{tcolorbox}[breakable, size=fbox, boxrule=.5pt, pad at break*=1mm, opacityfill=0]
\prompt{Out}{outcolor}{78}{\boxspacing}
\begin{Verbatim}[commandchars=\\\{\}]
AutoSklearnRegressor(per\_run\_time\_limit=30, time\_left\_for\_this\_task=120)
\end{Verbatim}
\end{tcolorbox}
        
    \begin{tcolorbox}[breakable, size=fbox, boxrule=1pt, pad at break*=1mm,colback=cellbackground, colframe=cellborder]
\prompt{In}{incolor}{79}{\boxspacing}
\begin{Verbatim}[commandchars=\\\{\}]
\PY{n}{predictions} \PY{o}{=} \PY{n}{automl}\PY{o}{.}\PY{n}{predict}\PY{p}{(}\PY{n}{data\PYZus{}X\PYZus{}test}\PY{p}{)}
\PY{n+nb}{print}\PY{p}{(}\PY{l+s+s2}{\PYZdq{}}\PY{l+s+s2}{AutoML R2 score:}\PY{l+s+s2}{\PYZdq{}}\PY{p}{,} \PY{n}{r2\PYZus{}score}\PY{p}{(}\PY{n}{data\PYZus{}y\PYZus{}test}\PY{p}{,} \PY{n}{predictions}\PY{p}{)}\PY{p}{)}
\end{Verbatim}
\end{tcolorbox}

    \begin{Verbatim}[commandchars=\\\{\}]
AutoML R2 score: 0.808234600610789
    \end{Verbatim}

    \begin{tcolorbox}[breakable, size=fbox, boxrule=1pt, pad at break*=1mm,colback=cellbackground, colframe=cellborder]
\prompt{In}{incolor}{95}{\boxspacing}
\begin{Verbatim}[commandchars=\\\{\}]
\PY{n+nb}{print}\PY{p}{(}\PY{n}{automl}\PY{o}{.}\PY{n}{show\PYZus{}models}\PY{p}{(}\PY{p}{)}\PY{p}{)}
\end{Verbatim}
\end{tcolorbox}

    \begin{Verbatim}[commandchars=\\\{\}]
[(0.680000, SimpleRegressionPipeline(\{'data\_preprocessing:categorical\_transforme
r:categorical\_encoding:\_\_choice\_\_': 'no\_encoding',
'data\_preprocessing:categorical\_transformer:category\_coalescence:\_\_choice\_\_':
'no\_coalescense',
'data\_preprocessing:numerical\_transformer:imputation:strategy': 'mean',
'data\_preprocessing:numerical\_transformer:rescaling:\_\_choice\_\_':
'robust\_scaler', 'feature\_preprocessor:\_\_choice\_\_': 'feature\_agglomeration',
'regressor:\_\_choice\_\_': 'gradient\_boosting',
'data\_preprocessing:numerical\_transformer:rescaling:robust\_scaler:q\_max':
0.7727512096172742,
'data\_preprocessing:numerical\_transformer:rescaling:robust\_scaler:q\_min':
0.22461598115758682, 'feature\_preprocessor:feature\_agglomeration:affinity':
'manhattan', 'feature\_preprocessor:feature\_agglomeration:linkage': 'complete',
'feature\_preprocessor:feature\_agglomeration:n\_clusters': 21,
'feature\_preprocessor:feature\_agglomeration:pooling\_func': 'max',
'regressor:gradient\_boosting:early\_stop': 'train',
'regressor:gradient\_boosting:l2\_regularization': 2.208787572338781e-05,
'regressor:gradient\_boosting:learning\_rate': 0.036087332404571744,
'regressor:gradient\_boosting:loss': 'least\_squares',
'regressor:gradient\_boosting:max\_bins': 255,
'regressor:gradient\_boosting:max\_depth': 'None',
'regressor:gradient\_boosting:max\_leaf\_nodes': 64,
'regressor:gradient\_boosting:min\_samples\_leaf': 3,
'regressor:gradient\_boosting:scoring': 'loss',
'regressor:gradient\_boosting:tol': 1e-07,
'regressor:gradient\_boosting:n\_iter\_no\_change': 18\},
dataset\_properties=\{
  'task': 4,
  'sparse': False,
  'multioutput': False,
  'target\_type': 'regression',
  'signed': False\})),
(0.220000, SimpleRegressionPipeline(\{'data\_preprocessing:categorical\_transformer
:categorical\_encoding:\_\_choice\_\_': 'one\_hot\_encoding',
'data\_preprocessing:categorical\_transformer:category\_coalescence:\_\_choice\_\_':
'minority\_coalescer',
'data\_preprocessing:numerical\_transformer:imputation:strategy': 'mean',
'data\_preprocessing:numerical\_transformer:rescaling:\_\_choice\_\_': 'none',
'feature\_preprocessor:\_\_choice\_\_': 'polynomial', 'regressor:\_\_choice\_\_':
'gradient\_boosting', 'data\_preprocessing:categorical\_transformer:category\_coales
cence:minority\_coalescer:minimum\_fraction': 0.010000000000000004,
'feature\_preprocessor:polynomial:degree': 3,
'feature\_preprocessor:polynomial:include\_bias': 'True',
'feature\_preprocessor:polynomial:interaction\_only': 'True',
'regressor:gradient\_boosting:early\_stop': 'train',
'regressor:gradient\_boosting:l2\_regularization': 6.085630700044881e-10,
'regressor:gradient\_boosting:learning\_rate': 0.12392806728650493,
'regressor:gradient\_boosting:loss': 'least\_squares',
'regressor:gradient\_boosting:max\_bins': 255,
'regressor:gradient\_boosting:max\_depth': 'None',
'regressor:gradient\_boosting:max\_leaf\_nodes': 31,
'regressor:gradient\_boosting:min\_samples\_leaf': 25,
'regressor:gradient\_boosting:scoring': 'loss',
'regressor:gradient\_boosting:tol': 1e-07,
'regressor:gradient\_boosting:n\_iter\_no\_change': 7\},
dataset\_properties=\{
  'task': 4,
  'sparse': False,
  'multioutput': False,
  'target\_type': 'regression',
  'signed': False\})),
(0.100000, SimpleRegressionPipeline(\{'data\_preprocessing:categorical\_transformer
:categorical\_encoding:\_\_choice\_\_': 'no\_encoding',
'data\_preprocessing:categorical\_transformer:category\_coalescence:\_\_choice\_\_':
'minority\_coalescer',
'data\_preprocessing:numerical\_transformer:imputation:strategy': 'mean',
'data\_preprocessing:numerical\_transformer:rescaling:\_\_choice\_\_': 'none',
'feature\_preprocessor:\_\_choice\_\_': 'polynomial', 'regressor:\_\_choice\_\_':
'ard\_regression', 'data\_preprocessing:categorical\_transformer:category\_coalescen
ce:minority\_coalescer:minimum\_fraction': 0.00829519231049576,
'feature\_preprocessor:polynomial:degree': 2,
'feature\_preprocessor:polynomial:include\_bias': 'False',
'feature\_preprocessor:polynomial:interaction\_only': 'False',
'regressor:ard\_regression:alpha\_1': 4.7044575285722365e-05,
'regressor:ard\_regression:alpha\_2': 0.000629863807127318,
'regressor:ard\_regression:fit\_intercept': 'True',
'regressor:ard\_regression:lambda\_1': 7.584067704707025e-10,
'regressor:ard\_regression:lambda\_2': 3.923255608410879e-08,
'regressor:ard\_regression:n\_iter': 300,
'regressor:ard\_regression:threshold\_lambda': 4052.403778957396,
'regressor:ard\_regression:tol': 0.009359388994186051\},
dataset\_properties=\{
  'task': 4,
  'sparse': False,
  'multioutput': False,
  'target\_type': 'regression',
  'signed': False\})),
]
    \end{Verbatim}


\chapter{Код для макета WEB-версии}
\inputminted[tabsize=4, linenos, breaklines]{python}{app.py}

\chapter{Заключение}

\qquad В ходе работы я исследовал несколько разных моделей машинного обучения(в т.ч. используя автоматический пакет AutoML) и создал макет веб-приложения, предназначенного для анализа данных. \\
\qquad Лучше всего себя показали ансамблевые модели градиентного бустинга и
случаного леса. Самый лучший результат показывает модель градиентного
бустинга(при n\_estimators=400). Без подбора параметров модель
случайного леса дает лучший результат. Худшей моделью оказалась
LinearSVR, без подбора параметров дает вообще отрицательный результат по
оценке r2. AutoML также выбрал модель градиентного бустинга как
оптимальную. \\

\chapter{Список использованных источников}
\begin{itemize}
		\item Конспекты лекций - https://github.com/ugapanyuk/ml\_course\_2021/wiki/COURSE\_TMO
		\item Документация библиотеки skit-learn - https://scikit-learn.org/stable/user\_guide.html
		\item Документация библиотеки streamlit - https://docs.streamlit.io/en/stable/
		\item Использованный набор данных для обучения - https://www.kaggle.com/harlfoxem/housesalesprediction?select=kc\_house\_data.csv
		\item Jupyter Lab - https://jupyterlab.readthedocs.io/en/stable/index.html
\end{itemize}

    
\end{document}
